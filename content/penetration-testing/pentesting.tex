\section{Penetration Testing}
Penetration testing (pentesting for short) is an simulated attack to test the security and discover vulnerabilities. The goal of a penetration test is to find the weak points in a system to be able to fix and secure them, before a malicious actor finds them.

Companies, like Secura, perform penetration tests for clients (i.e.\ other organizations). The result of such a penetration test is a report detailing the weaknesses of the client's system. This gives the client insight in how they should secure their systems and what weaknesses an attacker might actually target.

These penetration tests are performed in phases (called a kill chain):
\begin{enumerate}
    \item Reconnaissance: Gather data about the target system. This can be actively gathered (i.e.\ with interaction with the target) or passively gathered (i.e.\ without interaction with the target).
    \item Exploitation: The data that has been gathered is used to identify weak spots and vulnerabilities. These are attacked and exploited to gain (unprivileged) access.
    \item Post-exploitation: After successful exploitation and gaining a foothold, a persistent foothold is established.
    \item Exfiltration: Once a persistent foothold has been established, sensitive data from the system needs to be retrieved/downloaded.
    \item Cleanup: Once the attack is successful, all traces of the attack should be wiped clean.
\end{enumerate}

\hfill

There are many types of penetration tests. Most tests differ in what information about the system the assessor gets before the assessment starts or what kind of system is being tested. These are some common assessments that companies, like Secura, perform:
\begin{itemize}
    \item Black Box: The assessor does not get any information about the system they are going to test.
    \item Grey Box: The assessor gets some information (e.g.\ credentials) about the system.
    \item White Box/Crystal Box:  The assessor gets all information about the system and it's internal working.
    \item Infrastructure Research: An assessment of the infrastructure of a system (e.g.\ a network or a server), without having knowledge of the internal workings (i.e.\ black box).
    \item Configuration Research: A white box infrastructure research. The assessor gets information on the complete infrastructure of a system.
    \item Internal Assessment: An assessment of the internal network of a system. Most of the time the assessment has a clear goal (e.g.\ finding certain sensitive information).
    \item Red Teaming: An security assessment with a specific goal that takes weeks or months. The focus heavily lies on stealth.
    \item Social Engineering: An assessment of the security of the people interacting with a system. For example, sending phishing mails.
    \item Code Reviews: Reviewing the source code of a system.
    \item Design Review: Reviewing the design of a system, by looking at architecture design descriptions and interviewing engineers and developers. This is possible to do before the system is actually build.
\end{itemize}
