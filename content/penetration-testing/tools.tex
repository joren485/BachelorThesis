\section{Automation Tools}\label{section:tools}
Most security assessments are time restricted. Large, complex systems need to be assessed in a short amount of time. There are tools that automate part of the assessment process. Instead of taking every step manually, these tools scan systems automatically and systematically to find known vulnerabilities and possible weak spots in a system. In this section we will discuss the tools that help us automate part of the manual steps of \autoref{section:identify-vulnerabilities} to identify and exploit the vulnerabilities we discussed in \autoref{chapter:vulnerabilities}.

As we will see, most tools have a specific focus (e.g.\ a single vulnerability or part of a system). This makes it is harder to separate them into groups that correspond to the attacker models of \autoref{chapter:attack-surface-models}. We instead separate them into groups that match their purpose: host configuration scanners (\autoref{tools:host-scanners}), Docker image analysis tools (\autoref{subsection:image-analysis-tools}) and exploitation tools (\autoref{subsection:offensive-tools}).

\subsection{Host Configuration Scanners}\label{tools:host-scanners}
The tools described in this section are run on a host running Docker (see \autoref{subsection:testing-host}). They check for issues in the configuration of Docker, available images and available containers.

\subsubsection{Docker Bench for Security}\label{tools:bench}
Docker itself has released a scanner (called Docker Bench for Security\footnote{\url{https://github.com/docker/docker-bench-security}}) that is based on the CIS Docker Benchmark. It is meant to run on a host running Docker. The scanner checks whether the Docker configuration, images and containers on the host follow every guideline in the CIS Docker Benchmark. Some guidelines might be irrelevant to every host (e.g.\ guidelines relating to Docker Swarm). These are skipped by Docker Bench for Security.

\medskip

Docker Bench for Security solves the biggest problem of the CIS Docker Benchmark: its length. The CIS Docker Benchmark is a long document, which makes it hard to use (as discussed in \autoref{futurework:CIS}). Because Docker Bench for Security automatically checks all guidelines, we only need to look at the guidelines that do not pass the check. This makes it a helpful tool during a security assessment.

\subsubsection{Dockscan}
Dockscan\footnote{\url{https://github.com/kost/dockscan}} checks a host and the running containers for misconfigurations (not every misconfiguration is security related). It is quite old (the last change is made in august 2016\footnote{\url{https://github.com/kost/dockscan/commit/0a21d64}}) and as such less useful during a penetration test. Dockscan scans for the following misconfigurations:
\begin{itemize}
    \item The number of changed but not persistent files of running containers.
    \item Empty passwords in containers (similar to \autoref{subsection:CVE-2019-5021}).
    \item The number of processes running inside a container.
    \item Whether a SSH server is running inside a container.
    \item If a non-stable version of Docker is installed.
    \item The use of insecure registries.
    \item Whether memory limits have been set for containers.
    \item Whether IPv4 traffic forwarding is enabled in Docker.
    \item Whether a mirror registry is used.
    \item Whether the AUFS storage driver is used.
\end{itemize}

\subsection{Docker Image Analysis Tools}\label{subsection:image-analysis-tools}
Most Docker security analysis tools focus on static analysis of Docker images. They look for software and libraries inside the images and match these against known vulnerability databases. Some also look for sensitive information (e.g.\ passwords) that might be stored inside the image. \autoref{appendix:static-analysis-list} lists the available Docker image analysis tools.

Although these tools are more useful from a defensive perspective (e.g.\ scanning images for problems before they are deployed), they might reveal vulnerabilities or sensitive information during a penetration test.

\subsection{Exploitation Tools}\label{subsection:offensive-tools}
There are tools that specifically focus on the exploitation of vulnerabilities. These tools focus on escaping containers or escalating privileges on the host. They can be useful during a penetration test, because they will automate exploitation of specific vulnerabilities.

\subsubsection{Break Out of the Box}
Break Out of the Box\footnote{\url{https://github.com/brompwnie/botb}} (BOtB) is a tool that identifies and exploits common container escape vulnerabilities. It is able to do the following escapes:

\begin{itemize}
    \item If BOtB finds the Docker socket mounted inside the container (which we manually do in \autoref{subsubsection:searching-socket}), BOtB can escape the container using the same technique we discuss in \autoref{subsection:api}.

    \item BOtB is able to escape containers using CVE--2019--5736 (see \autoref{CVE-2019-5736}).

    \item BOtB is able to identify sensitive information in environment variables (see \autoref{pentest:container:env-vars}).

    \item If the container is running in privileged mode, BOtB tries to escape using the same vulnerability\footnote{It should be noted that privileged mode is not needed for this container escape to work (as discussed in \autoref{CAP_SYS_ADMIN}).} we looked at in \autoref{CAP_SYS_ADMIN}~\cite{TrailOfBits-Docker-Escape}.
\end{itemize}

\subsubsection{Metasploit}
Metasploit\footnote{\url{https://www.metasploit.com/}} is an exploitation framework (not only for Docker). It has some modules specific to Docker:

\begin{itemize}
    \item The ``Linux Gather Container Detection'' module checks whether it is running inside a container (similar to the checks we look at in \autoref{subsection:detection})~\cite{Metasploit-Linux-Gather-Container-Detection}.
    \item The ``Multi Gather Docker Credentials Collection'' module collects all \lstinline{.docker/config.json} files in the home directories of users (see \autoref{config-files:docker-config-json})~\cite{Docker-Credentials-Metasploit}.
    \item The ``Unprotected TCP Socket Exploit'' module gets \lstinline{root} access to a remote host which exposes its Docker API over TCP by creating a container with the host filesystem mounted as a volume (see \autoref{subsection:api} and specifically \autoref{subsubsection:remote-access})~\cite{Metasploit-Unprotected-TCP-Socket}.
\end{itemize}

\subsubsection{Harpoon}
Harpoon\footnote{\url{https://github.com/ProfessionallyEvil/harpoon}} is a tool that can identify whether it is running inside a container by looking at the \lstinline{cgroup} (see \autoref{subsubsection:detection:cgroup}) and tries to find and escape using a mounted Docker socket (see \autoref{subsection:api}).
