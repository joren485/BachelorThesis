\chapter{Penetration Testing of Docker}\label{chapter:pentesting}
In \autoref{chapter:vulnerabilities} we discuss specific vulnerabilities. Before we can exploit those vulnerabilities, we first need to perform reconnaissance on the target system to gather data. This data can then be used to identify weak spots and vulnerabilities. This chapter will focus on gathering that interesting data and identifying those vulnerabilities.

In \autoref{section:identify-vulnerabilities} we focus on how to do this manually for both perspectives of \autoref{chapter:attack-surface-models}. In \autoref{section:tools} we will look at available tools that will help us automate part of assessments.

In \autoref{chapter:checklist} we will combine the information from \autoref{chapter:vulnerabilities} and this chapter into a checklist.

\section{Manually Identifying Vulnerabilities}\label{section:identify-vulnerabilities}
In this section we will discuss how we can manually identify the vulnerabilities we looked at in \autoref{chapter:vulnerabilities} once we have access to a system. This section is split into three parts, that correspond to the attacker models of \autoref{chapter:attack-surface-models}.

In \autoref{subsection:detection} we look at techniques to identify which attacker model is relevant during an assessment. This means we will discuss techniques to identify whether we are inside a container or on a host.

The second part (\autoref{subsection:testing-container}) corresponds directly to container escapes (\autoref{subsection:container-escape}). We take the perspective of a process inside a container and look how we could perform a container escape attack.

The third part (\autoref{subsection:testing-host}) corresponds directly to Docker daemon attacks (\autoref{attacker-model:daemon-attacks}). We take the perspective of an (unprivileged) process on a host with Docker installed on it and look how we could perform a Docker daemon attack.

\medskip

We will mostly focus on the misconfigurations (\autoref{section:misconfigurations}), because although the security related bugs (\autoref{section:bugs}) might have a high impact, they are all mitigated with one simple line of advice: ``Keep your systems up to date''. Checking whether a system is vulnerable to a known bug is also a lot easier than checking whether a system is vulnerable because of misconfiguration, because all Docker bugs are dependent on the version of Docker being out of date (i.e.\ the Docker version tells us what Docker is vulnerable to).

\subsection{Detect If We Are Running in a Container}\label{subsection:detection}
In most security assessments and penetration tests it will be clear what kind of system (i.e.\ running inside a container or not) we are attacking. In some cases, however, it might not be. A good example of this is getting remote code execution on a system during a black box penetration test. In that case, we might get a reverse shell and are able to execute commands, but do not know anything about the systems' internal workings. In such a case it is important to know if we are running in a Docker container or not.

\medskip

In this section, we will look at steps that show us whether we are in a Docker container. These steps are in order of ease and certainty. If we know we are inside a container, we can look for vulnerabilities inside the container (see \autoref{subsection:testing-container}). If we know we are not running inside a container, we can look for vulnerabilities on the host (see \autoref{subsection:testing-host}).

\subsubsection{\texorpdfstring{\lstinline{/.dockerenv}}{/.dockerenv}}
\lstinline{/.dockerenv} is a file that is present in all Docker containers. It was used in the past by LXC to load the environment variables in the container. Currently it is always empty, because LXC is not used anymore. However, it is still (officially) used to identify containers\cite{Metasploit-Linux-Gather-Container-Detection}\cite{Removed-Dockerinit-Reference}.

\subsubsection{Control Group}\label{subsubsection:detection:cgroup}
To limit the resources of containers, Docker creates control groups for each container and a parent control group called \lstinline{docker}. If a process is started in a Docker container, that process will have to be in the control group of that container. We can verify this by looking at the \lstinline{cgroup} of the initial process (\lstinline{/proc/1/cgroups})~\cite{Metasploit-Linux-Gather-Container-Detection}.

\begin{lstlisting}[caption={Process control group inside container.\protect\footnotemark},captionpos=b]
(cont)# cat /proc/1/cgroup
12:hugetlb:/docker/0c7a3b8...
11:blkio:/docker/0c7a3b8...
...
\end{lstlisting}
\footnotetext{Long lines have been abbreviated with ``$\ldots$''.}

If we look at a host, we do not see the same \lstinline{/docker/} parent control group.
\begin{lstlisting}[caption={Process control groups on the host.},captionpos=b]
(cont)# cat /proc/1/cgroup
12:hugetlb:/
11:blkio:/
...
\end{lstlisting}

In some systems that are using Docker (e.g.\ orchestration software), the parent control group has another name (e.g. \lstinline{kubepod} for Kubernetes).

\subsubsection{Running Processes}
Containers are made to run one process. Host systems run many processes. All these processes have one root process (with process id 1) to start all necessary processes. On most Linux systems that process is either \lstinline{init} or \lstinline{systemd}. You would never see \lstinline{init} or \lstinline{systemd} in a container, because the container only runs one process not a full operating system. That is why the amount of processes and the process with pid 1 is a good indicator whether we are running in a container.

\subsubsection{Available Libraries and Binaries}\label{subsubsection:binaries}
Docker images are made as small as possible. Many processes do not need a fully operational Linux system, they need only part of it. That is why developers often remove libraries and binaries that are not needed for their specific application from their Docker images. If we see a lot of missing packages, binaries or libraries it is a good indicator that we are running in a container.

\medskip

The \lstinline{sudo} package is a good example of this. This package is crucial on many Linux distributions, because it enables a way for non-\lstinline{root} users to execute commands as \lstinline{root}. However, in a Docker container the \lstinline{sudo} package does not make a lot of sense. If a process needs to run something as \lstinline{root}, the process should be run as \lstinline{root} in the container. That is why \lstinline{sudo} is often not installed in Docker images.


\subsection{Testing from Container}
If we have code execution inside of a container, we are going to focus on escaping. Because the Docker daemon runs as \lstinline{root}, we will most likely get \lstinline{root} access to the host if we escape. We will take a look at steps we can take to identify the container operating system, the container image, the host operating system and weak spots in the container.

\hfill

Many Docker images are stripped from unnecessary tools, binaries and libraries to make the image smaller. However, we might need those tools during a penetration test. If we are \lstinline{root} in a container, we are most likely able to install the necessary tooling. If we only have access to a non-\lstinline{root} user, it might not be possible to install anything. In that case, we will have to work with what is available to us or find a way to get statically compiled binaries inside the container.

\subsubsection{Identifying Users}
The first step we should take is to see if we are a privileged user and identify other users. We can see our current user by using \lstinline{id} and see all users by looking at \lstinline{/etc/passwd}.
\begin{lstlisting}[caption={Current and all user enumeration},captionpos=b]
(cont)# id
uid=0(root) gid=0(root) groups=0(root)
(cont)# cat /etc/passwd
root:x:0:0:root:/root:/bin/bash
...
test:x:1000:1000:,,,:/home/test:/bin/bash
\end{lstlisting}

Wee see that our current user is \lstinline{root} (the user id is 0) and that there are two users (besides the default users in Linux). By default, containers run as \lstinline{root}. Which is great from an attacker perspective, because it allows us full access to everything inside the container. A well configured container most likely does not run as \lstinline{root} (see \autoref{subsection:non-root-user}).

\subsubsection{Identifying Docker Image}

The next step is to identify the operating system (and maybe the Docker Image) of the container.

All modern Linux distributions have a file \lstinline{/etc/os-release}\footnote{Although this file is introduced by \lstinline{systemd}, even systems that explicitly do not use \lstinline{systemd} contain it.} that contains information about the running operating system.
\begin{lstlisting}
(host)$ docker run -it --rm centos:latest cat /etc/os-release
...
PRETTY_NAME="CentOS Linux 8 (Core)"
...
\end{lstlisting}

\hfill

To get a better idea of what a container is supposed to do, we can look at the processes. Because containers should only have a singular task (e.g.\ running a database), they should only have one running process. 

\begin{lstlisting}
(host)$ docker run --rm --env MYSQL_RANDOM_ROOT_PASSWORD=true --name=database mariadb:latest
...
(host)$ docker exec database ps -A -o pid,cmd
PID CMD
  1 mysqld
320 ps -A -o pid,cmd
\end{lstlisting}

In this example, we see that the image \lstinline{mariadb} only has only process (\lstinline{mysqld}). This way we know (without looking at the image name) that the container is a MySQL servier.

\subsubsection{Identifying Host Operating System}\label{subsubsection:identify-host-os}
It is also important to look for information about the host. This can be very useful to identify and use relevant exploits.

\medskip

Because containers use the kernel of the host, we can use the kernel version to identify information about the host. Let's take a look at the following example running on an Ubuntu host.
\begin{lstlisting}[caption={\lstinline{/etc/os-release} and \lstinline{uname} differ.},captionpos=b]
(host)$ docker run -it --rm alpine:latest cat /etc/os-release
...
PRETTY_NAME="Alpine Linux v3.10"
...
(host)$ docker run -it --rm alpine:latest uname -rv
5.0.0-36-generic #39~18.04.1-Ubuntu SMP Tue Nov 12 11:09:50 UTC 2019
\end{lstlisting}

We are running an Alpine Linux container, which we see when we look in the \lstinline{/etc/os-release} file. However, when we look at the kernel version (using the \lstinline{uname} command), we see that we are using an Ubuntu kernel. That means that we are most likely running on an Ubuntu host.

\medskip

We also see the kernel version number (in this case \lstinline{5.0.0-36-generic}). This can be used to see if the system is vulnerable to kernel exploits, because some kernel exploits may be used to escape the container.

\subsubsection{Checking Capabilities}
Once we have a clear picture what kind of system we are working with, we can do some more detailed reconnaissance. One of the most important things to look at are the kernel capabilities (see \autoref{protection-mechanisms:subsection:capabilities}) of the container. We can do this by looking at \lstinline{/proc/self/status}\footnote{\lstinline{/proc/self/} refers to \lstinline{/proc} of the current process}. This file contains multiple lines that contain information about the granted capabilities.

\begin{lstlisting}[caption={Capabilities of process in container},captionpos=b]
(cont)# grep Cap /proc/self/status
CapInh:	00000000a80425fb
CapPrm:	00000000a80425fb
CapEff:	00000000a80425fb
CapBnd:	00000000a80425fb
CapAmb:	0000000000000000
\end{lstlisting}

We see five different values:
\begin{itemize}
    \item \lstinline{CapInh}: The inheritable capabilities are the capabilities that a child process is allowed to get.
    \item \lstinline{CapPrm}: The permitted capabilities are the maximum capabilities that a process can use.
    \item \lstinline{CapEff}: The currently effective capabilities.
    \item \lstinline{CapBnd}: The capabilities that are permitted in the call tree.
    \item \lstinline{CapAmb}: Capabilities that non-root child processes can inherit.
\end{itemize}

We are interested in the \lstinline{CapEff} value, because that value represents the current capabilities. The capabilities are represented as a hexadecimal value. Every capability has a value and the \lstinline{CapEff} value is the combination of the values of granted capabilities. We can use the \lstinline{capsh} tool to get a list of capabilities from a hexadecimal value (this can be on a different system).

\begin{lstlisting}[caption={\lstinline{capsh} shows capabilities},captionpos=b]
(host)$ capsh --decode=00000000a80425fb
0x00000000a80425fb=cap_chown,cap_dac_override,cap_fowner,cap_fsetid,cap_kill,cap_setgid,cap_setuid,cap_setpcap,cap_net_bind_service,cap_net_raw,cap_sys_chroot,cap_mknod,cap_audit_write,cap_setfcap
\end{lstlisting}

We can use this to check if there are any capabilities that can be used to escape the Docker container (see \autoref{misconfigurations:subsection:capabilities}).

\subsubsection{Checking for Privileged Mode}

As stated before, if the container runs in privileged mode it gets all capabilities. This makes it easy to check if we are running a process in a container in privileged mode. \lstinline{0000003fffffffff} is the representation of all capabilities.

\begin{lstlisting}[caption={\lstinline{capsh} shows privileged capabilities},captionpos=b]
(host)$ docker run -it --rm --privileged ubuntu:latest grep CapEff /proc/1/status
CapEff:	0000003fffffffff
(host)$ capsh --decode=0000003fffffffff 
0x0000003fffffffff=cap_chown,cap_dac_override,cap_dac_read_search,cap_fowner,cap_fsetid,cap_kill,cap_setgid,cap_setuid,cap_setpcap,cap_linux_immutable,cap_net_bind_service,cap_net_broadcast,cap_net_admin,cap_net_raw,cap_ipc_lock,cap_ipc_owner,cap_sys_module,cap_sys_rawio,cap_sys_chroot,cap_sys_ptrace,cap_sys_pacct,cap_sys_admin,cap_sys_boot,cap_sys_nice,cap_sys_resource,cap_sys_time,cap_sys_tty_config,cap_mknod,cap_lease,cap_audit_write,cap_audit_control,cap_setfcap,cap_mac_override,cap_mac_admin,cap_syslog,cap_wake_alarm,cap_block_suspend,cap_audit_read
\end{lstlisting}

If we find a privileged container, we can easily escape it (as shown in \autoref{subsection:privileged}).

\subsubsection{Checking Volumes}\label{subsubsection:volumes}
Volumes, the directories that are mounted from the host into the container, are the persistent data of the container. This persistent data might contain sensitive information, that is why it is important to check what directories are mounted into the container.

We can do this by looking at the mounted filesystem locations.
\begin{lstlisting}[caption={The (very abbreviated) contents of \lstinline{/proc/mounts} in a Docker container.},captionpos=b,label={listing:proc-mounts}]
(host)$ docker run -it --rm -v /tmp:/host/tmp ubuntu cat /proc/mounts
overlay / overlay...
...
/dev/mapper/ubuntu--vg-root /host/tmp...
/dev/mapper/ubuntu--vg-root /etc/resolv.conf...
/dev/mapper/ubuntu--vg-root /etc/hostname ext4...
/dev/mapper/ubuntu--vg-root /etc/hosts...
...
\end{lstlisting}

Every line contains information about one mount. We see many lines (which are abbreviated or omitted from \autoref{listing:proc-mounts}). We see the root OverlayFS mount at the top and to what path it points on the host (some path in \lstinline{/var/lib/docker/overlay2/}). We also see which directories are mounted from the root file system on the host (which in this case is the LVM logical volume \lstinline{root} which is represented in the file system as \lstinline{/dev/mapper/ubuntu--vg-root}). In the command we can see that \lstinline{/tmp} on the host is mounted as \lstinline{/host/tmp} in the container and in \lstinline{/proc/mounts} we see that \lstinline{/host/tmp} is mounted. We unfortunately do not see what path on the host is mounted, only the path inside the container.

We now know this is an interesting path, because its contents need to be saved. During a penetration test, this would be a directory to pay extra attention to.

\subsubsection{Searching for the Docker Socket}


\subsubsection{Checking Network Configuration}
We should also look at the network of the container. We should look at which containers are in the same network and what the container is able to reach. To do this, we will most likely need to install some tools. Even the most basic networking tools (e.g.\ \lstinline{ping}) are removed from most Docker images, because very few containers will need them.

By default all containers get an IPv4 address in subnet \lstinline{172.17. 0.0/16}. It is possible to find the address (without installing anything) of a container you have access to by looking at \lstinline{/etc/hosts/} file. Docker will add a line that resolves the hostname of to the IPv4 address to \lstinline{/etc/hosts}.

\begin{lstlisting}[caption={Last line of \lstinline{/etc/hosts} in Docker.},captionpos=b]
(host)$ docker run -it --rm alpine tail -n1 /etc/hosts
172.17.0.2 e0e6b96367db
\end{lstlisting}


We can look at the Docker network by using \lstinline{nmap} (which we will have to install ourselves).
\begin{lstlisting}[caption={\lstinline{nmap} scan inside container.},captionpos=b]
(host)$ docker run -it --rm ubuntu bash
(cont)# apt update
...
(cont)# apt install nmap
...
(cont)# nmap -sn -PE 172.17.0.0/16
...
Nmap scan report for 172.17.0.1
Host is up (0.000044s latency).
MAC Address: 02:42:5F:92:ED:72 (Unknown)
Nmap scan report for 172.17.0.3
Host is up (0.000027s latency).
MAC Address: 02:42:AC:11:00:03 (Unknown)
\end{lstlisting}

We see that we can reach two containers, \lstinline{172.17.0.1} and \lstinline{172.17.0.2}. The former being the host itself and the latter being another docker. It is possible to capture the traffic of that container by using a \lstinline{ARP} man-in-the-middle attack (see \autoref{subsection:arp-spoofing}).


\subsection{Penetration Testing on a Host Running Docker}\label{subsection:testing-host}
When testing a host system with Docker installed on it, we are interested in bugs and misconfigurations that allow us to use Docker to access sensitive data or escalate our privileges within the system. In this section we will look at different steps we can take to gather information about the system and the configuration of Docker. This will tell us if it is possible to perform a Docker daemon (see \autoref{attacker-model:daemon-attacks}).

\subsubsection{Docker Version}\label{subsubsection:version}
The first step we take if we are testing a system that has Docker installed, is checking the Docker version. Docker does not need to be running and we do not need any special permissions (i.e. Docker permissions) to check the version of Docker\footnote{The version is hardcoded as string in the Docker client binary.}.

\begin{lstlisting}[caption={Show Docker version.},captionpos=b]
(host)$ docker -v
Docker version 19.03.5, build 633a0ea838
\end{lstlisting}

Once we have the Docker version, we should check for any CVEs (see \autoref{section:bugs} and \autoref{appendix:CVE-List}) that are available for the version of the Docker installation on the host.

\subsubsection{Who is allowed to use Docker?}
Docker permissions are defined by the permission bits on the Docker socket (i.e.\ \lstinline{/var/run/docker.sock}). By default, the owner (\lstinline{root}) and the group (\lstinline{docker}) have read and write permissions. Meaning that \lstinline{root} and every user in the \lstinline{docker} group are allowed to interact with the Docker socket.

We can see who is in the \lstinline{docker} group by looking in \lstinline{/etc/group}.
\begin{lstlisting}
$ grep docker /etc/group
docker:x:999:jvrancken
\end{lstlisting}
We see that only \lstinline{jvrancken} is part of the \lstinline{docker} group. It might also be interesting to look at which users have \lstinline{sudo} rights (in \lstinline{/etc/sudoers}). Users without \lstinline{sudo} but with Docker permissions still need to be considered \lstinline{sudo} users (see \autoref{subsection:docker-permissions}).

\hfill

It is possible that the Docker socket has permissions that give anybody permission to interact with Docker. Some people set the permissions to \lstinline{666} (i.e.\ read and write for all users). Giving all users read and write permission to the Docker socket allows them to use Docker.

\hfill

It is also possible that the \lstinline{setuid} bit is set on the Docker client. In that case, we are also able to use Docker (as the owner of the Docker client).

\begin{lstlisting}[caption={Permissions without and wit the \lstinline{setuid} bit.},captionpos=b]
(host)$ ls -l $(which docker)
-rwxr-xr-x 1 root root 88965248 nov 13 08:28 /usr/bin/docker
(host)# chmod +s $(which docker)
(host)$ ls -l $(which docker)
-rwsr-sr-x 1 root root 88965248 nov 13 08:28 /usr/bin/docker
\end{lstlisting}

\subsubsection{Configuration}\label{subsubsection:configuration}
Docker is configured using multiple files. The most important being the way the Docker daemon is started. Most systems will have a service manager that manages daemon processes. On many modern Linux distributions that is a task of \lstinline{systemd}. On other Linux systems the configuration file \lstinline{/etc/docker/daemon.json}\footnote{\url{https://docs.docker.com/engine/reference/commandline/dockerd/}} is used (and defaults might be set in \lstinline{/etc/default/docker}). These files will also tell us if the Docker API is available over TCP which, if not configured correctly, can be very dangerous (see \autoref{subsubsection:remote-access}).

\hfill

We can also look for user configuration files, that might contain secrets and sensitive data. See \autoref{subsection:config-files} for more information.

\subsubsection{Available Images \& Containers}
We should check which images and containers (both running and stopped) are available on the host. This will tell us more about the system we are testing.

\lstinline{docker images -a} will list all available images (including intermediate images) and \lstinline{docker ps -a} will list all (running and stopped) containers.

\begin{lstlisting}[caption={Listing all images and containers available.},captionpos=b]
(host)$ docker images -a
REPOSITORY  TAG     IMAGE ID        CREATED      SIZE
mariadb     latest  c1c9e6fba07a    2 weeks ago  355MB
ubuntu      latest  775349758637    4 weeks ago  64.2MB
alpine      3       965ea09ff2eb    6 weeks ago  5.55MB
alpine      latest  965ea09ff2eb    6 weeks ago  5.55MB
centos      latest  0f3e07c0138f    2 months ago 220MB
(host)$ docker ps -a --no-trunc --format="{{.Names}} {{.Command}} {{.Image}}"
database "docker-entrypoint.sh mysqld" mariadb:latest
\end{lstlisting}

\hfill

We should also look at the environment variables that have been passed to the containers, because environment variables are used to pass information (including passwords and secrets) to a container when it is created. Using \lstinline{docker inspect} we can see information about containers. Including the set environment variables.
\begin{lstlisting}[caption={List environment variables passed to Docker container.},captionpos=b]
(host)$ docker run --rm -e MYSQL_ROOT_PASSWORD=supersecret --name=database mariadb:latest
(host)$ docker inspect database | jq -r '.[0].Config.Env'
[
  "MYSQL_ROOT_PASSWORD=supersecret",
...
\end{lstlisting}

\hfill

The containers might have volumes. Those volumes tell us more about where sensitive and important data might be. We can also list the volumes using \lstinline{docker inspect}.
\begin{lstlisting}[caption={List bindmounts into Docker container.},captionpos=b]
(host)$ docker inspect database | jq -r '.[0].HostConfig.Binds'
[
  "/tmp/database/:/var/lib/mysql/"
]
\end{lstlisting}

\subsubsection{\texorpdfstring{\lstinline{iptables}}{iptables} Rules}
As we saw in \autoref{subsection:iptables}, Docker will bypass the host \lstinline{iptables} rules. Using \lstinline{iptables -vnL} and \lstinline{iptables -t nat -vnL} we can see the rules of the default tables \lstinline{mangle} and \lstinline{nat}, respectively. It is important that all firewall rules regarding Docker containers are set in the \lstinline{DOCKER-USER} chain in \lstinline{mangle}, because all Docker traffic will first pass the \lstinline{DOCKER-USER} chain.



\section{Automation Tools}\label{section:tools}
Most security assessments are time restricted. Large, complex systems need to be assessed in a short amount of time. There are tools that automate part of the assessment process. Instead of taking every step manually, these tools scan systems automatically and systematically to find known vulnerabilities and possible weak spots in a system. Because Docker is a popular ecosystem, there exist many different tools and scanners. Created by both companies (including Docker itself) and individuals interested in the security of Docker.

\medskip

The advantage of these tools is that they save a lot of time and effort, because they can look at every part of a system in a systematic way. However, while the tools save time, they do miss the precision and detail that manual testing brings. Many security flaws are complex and might only be vulnerable under specific circumstances. Manual examination and testing might reveal new vulnerabilities or vulnerable circumstances, while automated testing will only look for known vulnerabilities.

As we will see, most tools have a specific vulnerability or part of the system they scan. No single tool will automate all parts of an assessment.

That is why using tools can help us, but can never fully replace manual examination.

\medskip
In this section we will look at the different types of tools that are available to automate parts of what we looked at in \autoref{section:identify-vulnerabilities}.

\subsection{Host Configuration Scanners}
The tools described in this section are run on a host running Docker (see \autoref{subsection:testing-host}). They check for issues in the configuration of Docker, available images and available containers.

\subsubsection{Docker Bench for Security}\label{tools:bench}
Docker itself has released a scanner (called Docker Bench for Security\footnote{\url{https://github.com/docker/docker-bench-security}}) that is based on the CIS Docker Benchmark. It is meant to run on a host running Docker. The scanner checks whether the Docker configuration, images and containers on the host follow every guideline in the CIS Docker Benchmark. Some guidelines might be irrelevant to every host (e.g.\ guidelines relating to Docker Swarm). These are skipped by Docker Bench for Security.

\medskip

Docker Bench for Security solves the biggest problem of the CIS Docker Benchmark: its length. The CIS Docker Benchmark is a long document, which makes it hard to use (as discussed in \autoref{futurework:CIS}). Because Docker Bench for Security automatically checks all guidelines, we only need to look at the guidelines that do not pass the check. This makes it a helpful tool during a security assessment.

\subsubsection{Dockscan}
Dockscan\footnote{\url{https://github.com/kost/dockscan}} checks a host and the running containers for misconfigurations (not every misconfiguration is security related). It is quite old (the last change is made in august 2016) and as such less useful during a penetration test.\ dockscan scans for the following misconfigurations:
\begin{itemize}
    \item The number of changed but not persistent files of running containers.
    \item Empty passwords in containers (similar to \autoref{subsection:CVE-2019-5021}).
    \item The number of processes running inside a container.
    \item Whether a SSH server is running inside a container.
    \item If a non-stable version of Docker is installed.
    \item The use of insecure registries.
    \item Whether memory limits have been set for containers.
    \item Whether IPv4 traffic forwarding is enabled in Docker.
    \item Whether a mirror registry is used.
    \item If the AUFS storage driver is used.
\end{itemize}

\subsection{Docker Image Analysis Tools}\label{subsection:image-analysis-tools}
Most Docker security analysis tools focus on static analysis of Docker images. They look for software and libraries inside the images and match these against known vulnerability databases. Some also look for sensitive information (e.g.\ passwords) that might be stored inside the image. In \autoref{appendix:static-analysis-list} you will find a list of available Docker image analysis tools.

Although these tools are more useful from a defensive perspective (e.g.\ scanning images for problems before they are deployed), they might reveal vulnerabilities or sensitive information during a penetration test.

\subsection{Exploitation Tools}\label{subsection:offensive-tools}
There are tools that specifically focus on the exploitation of vulnerabilities. These tools focus on escaping containers or escalating privileges on the host. They can be useful during a penetration test, because they will automate exploitation of specific vulnerabilities.

\subsubsection{Break out of the Box}
Break out of the Box\footnote{\url{https://github.com/brompwnie/botb}} (BOtb) is a tool that identifies and exploits common container escape vulnerabilities. It is able to do the following escapes:

\begin{itemize}
    \item If BOtB finds the Docker socket mounted inside the container (which we manually do in \autoref{subsubsection:searching-socket}), BOtB can escape the container using the same technique we discuss in \autoref{subsection:api}.

    \item BOtB is able to escape containers using CVE--2019--5736 (see \autoref{CVE-2019-5736}).

    \item BOtB is able to identify sensitive information in environment variables (see \autoref{pentest:container:env-vars}).

    \item If the container is running in privileged mode, BOtB tries to escape using the same vulnerability\cite{TrailOfBits-Docker-Escape}\footnote{It should be noted that privileged mode is not needed for this container escape to work (as discussed in \autoref{CAP_SYS_ADMIN}).} we looked at in \autoref{CAP_SYS_ADMIN}.
\end{itemize}

\subsubsection{Metasploit}
Metasploit\footnote{\url{https://www.metasploit.com/}} is an exploitation framework (not only for Docker). It has some modules specific to Docker:

\begin{itemize}
    \item Linux Gather Container Detection\cite{Metasploit-Linux-Gather-Container-Detection}, checks whether it is running inside a container (similar to the checks we look at in \autoref{subsection:detection}).
    \item Multi Gather Docker Credentials Collection\cite{Docker-Credentials-Metasploit}, collects all \lstinline{.docker/config.json} files in the home directories of users (see \autoref{config-files:docker-config-json}).
    \item Unprotected TCP Socket Exploit\cite{Metasploit-Unprotected-TCP-Socket}, gets \lstinline{root} access to a remote host which exposes its Docker API over TCP by creating a container with the host filesystem mounted as a volume (see \autoref{subsection:api} and specifically \autoref{subsubsection:remote-access}).
\end{itemize}

\subsubsection{Harpoon}
Harpoon\footnote{\url{https://github.com/ProfessionallyEvil/harpoon}} is a simple tool that can identify whether it is running inside a container by looking at the \lstinline{cgroup} (see \autoref{subsubsection:detection:cgroup}) and tries to find and escape using a mounted Docker socket (see \autoref{subsection:api}).

