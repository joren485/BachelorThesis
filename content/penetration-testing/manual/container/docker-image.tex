\subsubsection{Identifying Docker Image}

The next step is to identify the operating system (and maybe the Docker Image) of the container.

All modern Linux distributions have a file \lstinline{/etc/os-release}\footnote{Although this file is introduced by \lstinline{systemd}, even systems that explicitly do not use \lstinline{systemd} contain it.} that contains information about the running operating system.
\begin{lstlisting}
(host)$ docker run -it --rm centos:latest cat /etc/os-release
...
PRETTY_NAME="CentOS Linux 8 (Core)"
...
\end{lstlisting}

\hfill

To get a better idea of what a container is supposed to do, we can look at the processes. Because containers should only have a singular task (e.g.\ running a database), they should only have one running process. 

\begin{lstlisting}
(host)$ docker run --rm --env MYSQL_RANDOM_ROOT_PASSWORD=true --name=database mariadb:latest
...
(host)$ docker exec database ps -A -o pid,cmd
PID CMD
  1 mysqld
320 ps -A -o pid,cmd
\end{lstlisting}

In this example, we see that the image \lstinline{mariadb} only has only process (\lstinline{mysqld}). This way we know (without looking at the image name) that the container is a MySQL servier.
