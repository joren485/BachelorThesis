\subsubsection{Identifying Operating System}

The next step is to identify the operating system (and maybe the Docker Image) of the container.

All modern Linux distributions have a file \lstinline{/etc/os-release}\footnote{Although this file was introduced by \lstinline{systemd}, operating systems that explicitly do not use \lstinline{systemd} (e.g.\ Void Linux) do use \lstinline{/etc/os-release}.} that contains information about the running operating system.
\begin{lstlisting}[caption={CentOS container \lstinline{/etc/os-release}.},captionpos=b]
(host)$ docker run -it --rm centos:latest cat /etc/os-release
...
PRETTY_NAME="CentOS Linux 8 (Core)"
...
\end{lstlisting}

To get a better idea of what a container is supposed to do, we can look at the processes. Because containers should only have a singular task (e.g.\ running a database), they should only have one running process.

\begin{lstlisting}[caption={A container only has one process.},captionpos=b]
(host)$ docker run --rm -e MYSQL_RANDOM_ROOT_PASSWORD=true --name=database mariadb:latest
...
(host)$ docker exec database ps -A -o pid,cmd
PID CMD
  1 mysqld
320 ps -A -o pid,cmd
\end{lstlisting}

In this example, we see that the image \lstinline{mariadb} only has one process\footnote{We also see our process listing all processes (with process id 320).} (\lstinline{mysqld}). This way we know that the container is a MySQL server and is probably (based on) the default MySQL Docker image (\lstinline{mariadb}).
