\subsubsection{Checking Capabilities}
Once we have a clear picture what kind of system we are working with, we can do some more detailed reconnaissance. One of the most important things to look at are the kernel capabilities (see \autoref{protection-mechanisms:subsection:capabilities}) of the container. We can do this by looking at \lstinline{/proc/self/status}\footnote{\lstinline{/proc/self/} refers to \lstinline{/proc} of the current process.}. This file contains multiple lines that contain information about the granted capabilities.

\begin{lstlisting}[caption={Capabilities of process in container.},captionpos=b]
(cont)# grep Cap /proc/self/status
CapInh:	00000000a80425fb
CapPrm:	00000000a80425fb
CapEff:	00000000a80425fb
CapBnd:	00000000a80425fb
CapAmb:	0000000000000000
\end{lstlisting}

We see five different values:
\begin{itemize}
    \item \lstinline{CapInh}: The inheritable capabilities are the capabilities that a child process is allowed to get.
    \item \lstinline{CapPrm}: The permitted capabilities are the maximum capabilities that a process can use.
    \item \lstinline{CapEff}: The currently effective capabilities.
    \item \lstinline{CapBnd}: The capabilities that are permitted in the call tree.
    \item \lstinline{CapAmb}: Capabilities that non-root child processes can inherit.
\end{itemize}

We are interested in the \lstinline{CapEff} value, because that value represents the current capabilities. The capabilities are represented as a hexadecimal value. Every capability has a value and the \lstinline{CapEff} value is the combination of the values of granted capabilities. We can use the \lstinline{capsh} tool to get a list of capabilities from a hexadecimal value (this can be on a different system).

\begin{lstlisting}[caption={\lstinline{capsh} shows capabilities.},captionpos=b]
(host)$ capsh --decode=00000000a80425fb
0x00000000a80425fb=cap_chown,cap_dac_override,cap_fowner,cap_fsetid,cap_kill,cap_setgid,cap_setuid,cap_setpcap,cap_net_bind_service,cap_net_raw,cap_sys_chroot,cap_mknod,cap_audit_write,cap_setfcap
\end{lstlisting}

We can use this to check if there are any capabilities that can be used to escape the Docker container (see \autoref{misconfigurations:subsection:capabilities}).

\subsubsection{Checking for Privileged Mode}

As stated before, if the container runs in privileged mode it gets all capabilities. This makes it easy to check if we are running a process in a container in privileged mode. \lstinline{0000003fffffffff} is the representation of all capabilities.

\begin{lstlisting}[caption={\lstinline{capsh} shows privileged capabilities.},captionpos=b]
(host)$ docker run -it --rm --privileged ubuntu:latest grep CapEff /proc/1/status
CapEff:	0000003fffffffff
(host)$ capsh --decode=0000003fffffffff
0x0000003fffffffff=cap_chown,cap_dac_override,cap_dac_read_search,cap_fowner,cap_fsetid,cap_kill,cap_setgid,cap_setuid,cap_setpcap,cap_linux_immutable,cap_net_bind_service,cap_net_broadcast,cap_net_admin,cap_net_raw,cap_ipc_lock,cap_ipc_owner,cap_sys_module,cap_sys_rawio,cap_sys_chroot,cap_sys_ptrace,cap_sys_pacct,cap_sys_admin,cap_sys_boot,cap_sys_nice,cap_sys_resource,cap_sys_time,cap_sys_tty_config,cap_mknod,cap_lease,cap_audit_write,cap_audit_control,cap_setfcap,cap_mac_override,cap_mac_admin,cap_syslog,cap_wake_alarm,cap_block_suspend,cap_audit_read
\end{lstlisting}

If we find a privileged container, we can easily escape it (as shown in \autoref{subsection:privileged}).
