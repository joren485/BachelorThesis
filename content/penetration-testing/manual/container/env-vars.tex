\subsubsection{Reading Environment Variables}\label{pentest:container:env-vars}
The environment variables are a way to communicate information to containers when they are started. When a container is started, environment variables are passed to it. These variables often contain passwords and other sensitive information.

\medskip

We can list all the environment variables that are set inside a Docker using the \lstinline{env} command (or by looking at the \lstinline{/proc/pid/environ} file of a process).

\begin{lstlisting}[caption={Listing all environment variables in a container}, captionpos=b]
(host)$ docker run --rm -e MYSQL_ROOT_PASSWORD=supersecret --name=database mariadb:latest
(host)$ docker exec -it database bash
(cont)# env
...
MYSQL_ROOT_PASSWORD=supersecret
...
\end{lstlisting}

It should be noted that this is not a misconfiguration. Using environment variables is the intended way to pass sensitive information to a Docker at runtime. However, during a black box penetration test, the sensitive information stored in the environment variables might be useful.

