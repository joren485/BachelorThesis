\subsubsection{Identifying Users}
The first step we should take is to see if we are a privileged user and identify other users. We can see our current user by using \lstinline{id} and see all users by looking at \lstinline{/etc/passwd}.
\begin{lstlisting}[caption={User enumeration.},captionpos=b]
(cont)# id
uid=0(root) gid=0(root) groups=0(root)
(cont)# cat /etc/passwd
root:x:0:0:root:/root:/bin/bash
...
test:x:1000:1000:,,,:/home/test:/bin/bash
\end{lstlisting}

Wee see that our current user is \lstinline{root} (the user id is 0) and that there are two users (besides the default users in Linux). By default, containers run as \lstinline{root}. That is great from an attackers perspective, because it allows us full access to everything inside the container. A well configured container most likely does not run as \lstinline{root} (see \autoref{subsection:non-root-user}).
