\subsubsection{Identifying Host Operating System}\label{subsubsection:identify-host-os}
It is also important to look for information about the host. This can be very useful to identify and use relevant exploits.

\medskip

Because containers use the kernel of the host, we can use the kernel version to identify information about the host. Let's take a look at the following example running on an Ubuntu host.
\begin{lstlisting}[caption={\lstinline{/etc/os-release} and \lstinline{uname} differ.},captionpos=b]
(host)$ docker run -it --rm alpine:latest cat /etc/os-release
...
PRETTY_NAME="Alpine Linux v3.10"
...
(host)$ docker run -it --rm alpine:latest uname -rv
5.0.0-36-generic #39~18.04.1-Ubuntu SMP Tue Nov 12 11:09:50 UTC 2019
\end{lstlisting}

We are running an Alpine Linux container, which we see when we look in the \lstinline{/etc/os-release} file. However, when we look at the kernel version (using the \lstinline{uname} command), we see that we are using an Ubuntu kernel. That means that we are most likely running on an Ubuntu host.

\medskip

We also see the kernel version number (in this case \lstinline{5.0.0-36-generic}). This can be used to see if the system is vulnerable to kernel exploits, because some kernel exploits may be used to escape the container.
