\subsection{Attack Scenario Detection}\label{subsection:detection}
In most security assessments and penetration tests it will be clear what kind of system (i.e.\ running as a Docker or not) we are attacking. In some cases, however, it might not be. A good example of this is getting remote code execution on a system during a black box penetration test. In that case, we might get a reverse shell and are able to execute commands, but do not know anything about the systems' internal workings. In such a case it is important to know if we are running in a Docker container or not.

In this section, we will look at fingerprinting a system to see whether we are in a Docker container.

\subsubsection{\texorpdfstring{\lstinline{/.dockerenv}}{/.dockerenv}}
\lstinline{/.dockerenv} is a file that is present in all Docker containers. It was used in the past by LXC\footnote{LXC used to be the engine that Docker used to create containers. It has now been replaced with \lstinline{containerd}.} to load the environment variables in the container. Currently it is always empty, because LXC is not used anymore. However, it is still (officially) used to identify whether a process is running in a Docker container\cite{Metasploit-Linux-Gather-Container-Detection}\cite{Removed-Dockerinit-Reference}.

\subsubsection{Control Group}\label{subsubsection:detection:cgroup}
To limit the resources of containers, Docker creates control groups for each container and a parent control group called \lstinline{docker}. If a process is started in a Docker container, that process will have to be in the control group of that container. We can verify this by looking at the \lstinline{cgroup} of the initial process (\lstinline{/proc/1/cgroups})\cite{Metasploit-Linux-Gather-Container-Detection}.

\begin{lstlisting}[caption={Process control group inside container\protect\footnotemark.},captionpos=b]
(cont)# cat /proc/1/cgroup
12:hugetlb:/docker/0c7a3b8...
11:blkio:/docker/0c7a3b8...
...
\end{lstlisting}
\footnotetext{Long lines have been abbreviated with ``$\ldots$''.}

If we look at a host, we do not see the same \lstinline{/docker/} parent control group.
\begin{lstlisting}[caption={Process control groups on the host.},captionpos=b]
(cont)# cat /proc/1/cgroup
12:hugetlb:/
11:blkio:/
...
\end{lstlisting}

In some systems that are using Docker (e.g.\ orchestration software), the parent control group has another name (e.g. \lstinline{kubepod} for Kubernetes).

\subsubsection{Running Processes}\label{subsubsection:processes}
Containers are made to run one process, while host systems run many processes. Processes on host systems have one root process (with process id 1) to start all necessary (child) processes. On most Linux systems that process is either \lstinline{init} or \lstinline{systemd}. We would never see \lstinline{init} or \lstinline{systemd} in a container, because the container only runs one process and not not a full operating system. That is why the amount of processes and the process with pid 1 is a good indicator whether we are running in a container.

\subsubsection{Available Libraries and Binaries}\label{subsubsection:binaries}
Docker images are made as small as possible. Many processes do not need a fully operational Linux system, they need only part of it. That is why developers often remove libraries and binaries that are not needed for their specific application from their Docker images. If we see a lot of missing packages, binaries or libraries it is a good indicator that we are running in a container.

\hfill

The \lstinline{sudo} package is a good example of this. This package is crucial on many Linux distributions, because it enables a way for non-\lstinline{root} users to execute commands as \lstinline{root}. However, in a Docker container the \lstinline{sudo} package does not make a lot of sense. If a process needs to run something as \lstinline{root}, the process should be run as \lstinline{root} in the container. That is why \lstinline{sudo} is often not installed in Docker images.

