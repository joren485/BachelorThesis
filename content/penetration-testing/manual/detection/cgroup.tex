\subsubsection{Control Group}\label{subsubsection:detection:cgroup}
To limit the resources of containers, Docker creates control groups for each container and a parent control group called \lstinline{docker}. If a process is started in a Docker container, that process will have to be in the control group of that container. We can verify this by looking at the \lstinline{cgroup} of the initial process (\lstinline{/proc/1/cgroups})\cite{Metasploit-Linux-Gather-Container-Detection}.

\begin{lstlisting}[caption={Process control group inside container\protect\footnotemark.},captionpos=b]
(cont)# cat /proc/1/cgroup
12:hugetlb:/docker/0c7a3b8...
11:blkio:/docker/0c7a3b8...
...
\end{lstlisting}
\footnotetext{Long lines have been abbreviated with ``$\ldots$''.}

If we look at a host, we do not see the same \lstinline{/docker/} parent control group.
\begin{lstlisting}[caption={Process control groups on the host.},captionpos=b]
(cont)# cat /proc/1/cgroup
12:hugetlb:/
11:blkio:/
...
\end{lstlisting}

In some systems that are using Docker (e.g.\ orchestration software), the parent control group has another name (e.g. \lstinline{kubepod} for Kubernetes).
