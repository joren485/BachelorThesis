\section{Manually Identifying Vulnerabilities}\label{section:identify-vulnerabilities}
During a penetration test we will get access to different systems. How we get that access depends on the type of assessment we are performing. During a white box assessment, we will most likely get full access to all systems before the assessment starts. During a black box assessment, on the other hand, we get access by exploiting vulnerabilities in systems to get a foothold (e.g.\ command execution). In this section we will discuss how we can manually identify the vulnerabilities we looked at in \autoref{chapter:vulnerabilities} once we have access to a system.

We will look at two different perspectives: inside a Docker container (\autoref{subsection:testing-container}) and from a host running Docker (\autoref{subsection:testing-host}). We will first look at detecting from which perspective we are attacking (\autoref{subsection:detection}). For both perspectives (container and host) we will then look at steps we can take (i.e.\ commands we can execute) to get information about the system and identify vulnerabilities.

\medskip

We will mostly focus on the misconfigurations, because although the security related bugs might have a high impact, they are all mitigated with one simple line of advice: ``Keep your systems up to date''. Checking whether a system is vulnerable to a known bug is also a lot easier than misconfigurations, because almost all Docker bugs are dependent on the version of Docker being out of date (i.e.\ the Docker version tells us what Docker is vulnerable to).
\subsection{Detect If We Are Running in a Container}\label{subsection:detection}
In most security assessments and penetration tests it will be clear what kind of system (i.e.\ running inside a container or not) we are attacking. In some cases, however, it might not be.

A good example of this, is finding a remote code execution vulnerability on a system during a black box penetration test. This allows us to execute arbitrary commands on a system that we know nothing about. In such a case it is important to know if we are running in a Docker container or not.

\medskip

In this section, we will look at steps that show us whether we are in a Docker container. These steps are in descending order of ease and certainty. If we know we are inside a container, we can perform reconnaissance inside the container (see \autoref{subsection:testing-container}). If we know we are not running inside a container, we can perform reconnaissance on the host (see \autoref{subsection:testing-host}).

\subsubsection{\texorpdfstring{\lstinline{/.dockerenv}}{/.dockerenv}}
\lstinline{/.dockerenv} is a file that is present in all Docker containers. It was used in the past by LXC\footnote{LXC used to be the engine that Docker used to create containers. It has now been replaced with \lstinline{containerd}.} to load the environment variables in the container. Currently it is always empty, because LXC is not used anymore. However, it is still (officially) used to identify whether a process is running in a Docker container\cite{Metasploit-Linux-Gather-Container-Detection}\cite{Removed-Dockerinit-Reference}.

\subsubsection{Control Group}\label{subsubsection:detection:cgroup}
To limit the resources of containers, Docker creates control groups for each container and a parent control group called \lstinline{docker}. If a process is started in a Docker container, that process will have to be in the control group of that container. We can verify this by looking at the \lstinline{cgroup} of the initial process (\lstinline{/proc/1/cgroups})\cite{Metasploit-Linux-Gather-Container-Detection}.

\begin{lstlisting}[caption={Process control group inside container\protect\footnotemark.},captionpos=b]
(cont)# cat /proc/1/cgroup
12:hugetlb:/docker/0c7a3b8...
11:blkio:/docker/0c7a3b8...
...
\end{lstlisting}
\footnotetext{Long lines have been abbreviated with ``$\ldots$''.}

If we look at a host, we do not see the same \lstinline{/docker/} parent control group.
\begin{lstlisting}[caption={Process control groups on the host.},captionpos=b]
(cont)# cat /proc/1/cgroup
12:hugetlb:/
11:blkio:/
...
\end{lstlisting}

In some systems that are using Docker (e.g.\ orchestration software), the parent control group has another name (e.g. \lstinline{kubepod} for Kubernetes).

\subsubsection{Running Processes}\label{subsubsection:processes}
Containers are made to run one process, while host systems run many processes. Processes on host systems have one root process (with process id 1) to start all necessary (child) processes. On most Linux systems that process is either \lstinline{init} or \lstinline{systemd}. We would never see \lstinline{init} or \lstinline{systemd} in a container, because the container only runs one process and not not a full operating system. That is why the amount of processes and the process with pid 1 is a good indicator whether we are running in a container.

\subsubsection{Available Libraries and Binaries}\label{subsubsection:binaries}
Docker images are made as small as possible. Many processes do not need a fully operational Linux system, they need only part of it. That is why developers often remove libraries and binaries that are not needed for their specific application from their Docker images. If we see a lot of missing packages, binaries or libraries it is a good indicator that we are running in a container.

\hfill

The \lstinline{sudo} package is a good example of this. This package is crucial on many Linux distributions, because it enables a way for non-\lstinline{root} users to execute commands as \lstinline{root}. However, in a Docker container the \lstinline{sudo} package does not make a lot of sense. If a process needs to run something as \lstinline{root}, the process should be run as \lstinline{root} in the container. That is why \lstinline{sudo} is often not installed in Docker images.


\subsection{Testing from Container}
If we have only access to a container, we are mostly are going to look for ways to escape it or see what we can reach from the container. In this section we will look at what we should look at and target.

\hfill

Many Docker images are stripped from unnecessary tools, binaries and libraries to make the image smaller. However, we might need those tools during a penetration test. If we are \lstinline{root} in a container, we are most likely able to install the necessary tooling. If we only have access to a non-\lstinline{root} user, it might not be possible to install anything. In that case, we will have to work with what is available to us.

\subsubsection{Identifying Users}\label{subsubsection:user-enumeration}
The first step we should take is to see if we are a privileged user and identify other users. We can see our current user by using \lstinline{id} and see all users by looking at \lstinline{/etc/passwd}.
\begin{lstlisting}[caption={User enumeration.},captionpos=b]
(cont)# id
uid=0(root) gid=0(root) groups=0(root)
(cont)# cat /etc/passwd
root:x:0:0:root:/root:/bin/bash
...
test:x:1000:1000:,,,:/home/test:/bin/bash
\end{lstlisting}

Wee see that our current user is \lstinline{root} (the user id is 0) and that there are two users (besides the default users in Linux). By default, containers run as \lstinline{root}. That is great from an attackers perspective, because it allows us full access to everything inside the container. A well configured container most likely does not run as \lstinline{root} (see \autoref{subsection:non-root-user}).

\subsubsection{Identifying Operating System}\label{subsubsection:idenitfy-container-os}
The next step is to identify the operating system (and maybe the Docker Image) of the container.

All modern Linux distributions have a file \lstinline{/etc/os-release}\footnote{Although this file was introduced by \lstinline{systemd}, operating systems that explicitly do not use \lstinline{systemd} (e.g.\ Void Linux) do use \lstinline{/etc/os-release}.} that contains information about the running operating system.
\begin{lstlisting}[caption={CentOS container \lstinline{/etc/os-release}.},captionpos=b]
(host)$ docker run -it --rm centos:latest cat /etc/os-release
...
PRETTY_NAME="CentOS Linux 8 (Core)"
...
\end{lstlisting}

To get a better idea of what a container is supposed to do, we can look at the processes. Because containers should only have a singular task (e.g.\ running a database), they should only have one running process.

\begin{lstlisting}[caption={A container only has one process.},captionpos=b]
(host)$ docker run --rm -e MYSQL_RANDOM_ROOT_PASSWORD=true --name=database mariadb:latest
...
(host)$ docker exec database ps -A -o pid,cmd
PID CMD
  1 mysqld
320 ps -A -o pid,cmd
\end{lstlisting}

In this example, we see that the image \lstinline{mariadb} only has one process\footnote{We also see our process listing all processes (with process id 320).} (\lstinline{mysqld}). This way we know that the container is a MySQL server and is probably (based on) the default MySQL Docker image (\lstinline{mariadb}).

\subsubsection{Identifying Host Operating System}
It is also important to look for information about the host. This can be very useful to identify and use relevant exploits.

\hfill

Because containers use the kernel of the host, we can use the kernel version to identify information about the host. Let's take a look at the following example running on an Ubuntu host.
\begin{lstlisting}[caption={\lstinline{/etc/os-release} and \lstinline{uname} differ.},captionpos=b]
(host)$ docker run -it --rm alpine:latest cat /etc/os-release
...
PRETTY_NAME="Alpine Linux v3.10"
...
(host)$ docker run -it --rm alpine:latest uname -rv
5.0.0-36-generic #39~18.04.1-Ubuntu SMP Tue Nov 12 11:09:50 UTC 2019
\end{lstlisting}

We are running an Alpine Linux container, which we see when we look in the \lstinline{/etc/os-release} file. However, when we look at the kernel version (using the \lstinline{uname} command), we see that we are using an Ubuntu kernel. That means that we are most likely running on an Ubuntu host.

\hfill

We also see the kernel version number (in this case \lstinline{5.0.0-36-generic}). This can be used to see if the system is vulnerable to kernel exploits, because some kernel exploits may be used to escape the container.

\subsubsection{Checking Capabilities}
Once we have a clear picture what kind of system we are working with, we can do some more detailed reconnaissance. One of the most important things to look at are the kernel capabilities (see \autoref{protection-mechanisms:subsection:capabilities}) of the container. We can do this by looking at \lstinline{/proc/self/status}\footnote{\lstinline{/proc/self/} refers to \lstinline{/proc} of the current process}. This file contains multiple lines that contain information about the granted capabilities.

\begin{lstlisting}[caption={Capabilities of process in container},captionpos=b]
(cont)# grep Cap /proc/self/status
CapInh:	00000000a80425fb
CapPrm:	00000000a80425fb
CapEff:	00000000a80425fb
CapBnd:	00000000a80425fb
CapAmb:	0000000000000000
\end{lstlisting}

We see five different values:
\begin{itemize}
    \item \lstinline{CapInh}: The inheritable capabilities are the capabilities that a child process is allowed to get.
    \item \lstinline{CapPrm}: The permitted capabilities are the maximum capabilities that a process can use.
    \item \lstinline{CapEff}: The currently effective capabilities.
    \item \lstinline{CapBnd}: The capabilities that are permitted in the call tree.
    \item \lstinline{CapAmb}: Capabilities that non-root child processes can inherit.
\end{itemize}

We are interested in the \lstinline{CapEff} value, because that value represents the current capabilities. The capabilities are represented as a hexadecimal value. Every capability has a value and the \lstinline{CapEff} value is the combination of the values of granted capabilities. We can use the \lstinline{capsh} tool to get a list of capabilities from a hexadecimal value (this can be on a different system).

\begin{lstlisting}[caption={\lstinline{capsh} shows capabilities},captionpos=b]
(host)$ capsh --decode=00000000a80425fb
0x00000000a80425fb=cap_chown,cap_dac_override,cap_fowner,cap_fsetid,cap_kill,cap_setgid,cap_setuid,cap_setpcap,cap_net_bind_service,cap_net_raw,cap_sys_chroot,cap_mknod,cap_audit_write,cap_setfcap
\end{lstlisting}

We can use this to check if there are any capabilities that can be used to escape the Docker container (see \autoref{misconfigurations:subsection:capabilities}).

\subsubsection{Checking for Privileged Mode}

As stated before, if the container runs in privileged mode it gets all capabilities. This makes it easy to check if we are running a process in a container in privileged mode. \lstinline{0000003fffffffff} is the representation of all capabilities.

\begin{lstlisting}[caption={\lstinline{capsh} shows privileged capabilities},captionpos=b]
(host)$ docker run -it --rm --privileged ubuntu:latest grep CapEff /proc/1/status
CapEff:	0000003fffffffff
(host)$ capsh --decode=0000003fffffffff 
0x0000003fffffffff=cap_chown,cap_dac_override,cap_dac_read_search,cap_fowner,cap_fsetid,cap_kill,cap_setgid,cap_setuid,cap_setpcap,cap_linux_immutable,cap_net_bind_service,cap_net_broadcast,cap_net_admin,cap_net_raw,cap_ipc_lock,cap_ipc_owner,cap_sys_module,cap_sys_rawio,cap_sys_chroot,cap_sys_ptrace,cap_sys_pacct,cap_sys_admin,cap_sys_boot,cap_sys_nice,cap_sys_resource,cap_sys_time,cap_sys_tty_config,cap_mknod,cap_lease,cap_audit_write,cap_audit_control,cap_setfcap,cap_mac_override,cap_mac_admin,cap_syslog,cap_wake_alarm,cap_block_suspend,cap_audit_read
\end{lstlisting}

If we find a privileged container, we can easily escape it (as shown in \autoref{subsection:privileged}).

\subsubsection{Identifying Volumes}

\subsubsection{Searching for the Docker Socket}
It is quite common for the Docker Socket to be mounted into containers. For example if we want to have a container that monitors the health of all other containers. However, this is very dangerous (as discussed in \autoref{subsection:api}). We can search for the socket using two techniques. We either look at the mounts (like in \autoref{subsubsection:volumes}) or we try to look for files with names similar to \lstinline{docker.sock}.

\begin{lstlisting}[caption={\lstinline{docker.sock} in \lstinline{/proc/mounts}.},captionpos=b,label={listing:socket-proc-mounts}]
(host)$ docker run -it --rm -v /var/run/docker.sock:/var/run/docker.sock ubuntu grep docker.sock /proc/mounts
tmpfs /run/docker.sock tmpfs rw,nosuid,noexec,relatime,size=792244k,mode=755 0 0
\end{lstlisting}

In \autoref{listing:socket-proc-mounts}, we mount \lstinline{/var/run/docker.sock} into the container as \lstinline{/var/run/docker.sock} and look at \lstinline{/proc/mounts}. We can see that the \lstinline{docker.sock} is mounted at \lstinline{/run/docker.sock} (it is not actually mounted at \lstinline{/var/run/docker.sock} because \lstinline{/var/run/} is a symlink to \lstinline{/run/}).

\medskip

\begin{lstlisting}[caption={Running \lstinline{find} to search for \lstinline{docker.sock}.},captionpos=b, label={listing:socket-find}]
(host)$ docker run -it --rm -v /var/run/docker.sock:/var/run/docker.sock ubuntu find . -name "docker.sock" /
/run/docker.sock
\end{lstlisting}

In \autoref{listing:socket-find}, we mount \lstinline{/var/run/docker.sock} into the container and search for files called ``\lstinline{docker.sock}''.

\subsubsection{Checking Network Configuration}
\todo[inline]{Reachable containers}
\todo[inline]{ARP Spoofing other containers}


\subsection{Testing from Host}

\subsubsection{Docker Version}
\todo[inline]{Docker version, does not require docker permissions}

\subsubsection{Running Containers}
\todo[inline]{Cronjobs}

\subsubsection{Docker Socket}

\subsubsection{Docker Daemon Configuration Files}
\todo[inline]{systemctl cat docker}
\todo[inline]{/etc/default/docker}
\todo[inline]{daemon.json}

\subsubsection{Docker Group}
\todo[inline]{Which users are in the docker group}

\subsubsection{\texorpdfstring{\lstinline{setuid}}{setuid} bit}

\subsubsection{\texorpdfstring{iptables}{iptables}}

\subsubsection{TCP Listener}

\subsubsection{Configuration Files}

