\subsubsection{Who is Allowed to Use Docker?}\label{subsubsection:docker-permissions-host}
Because having access to Docker is equivalent to having \lstinline{root} permissions, the users that are allowed to use Docker are interesting targets. If there is a way to become one of those users, we will essentially have access to everything on the host.

As discussed in \autoref{background:docker-socket}, every user with read an write access to the Docker socket (i.e. \lstinline{/var/run/docker.sock}) has permissions to use Docker. That is why the first thing we should do is see which users have read and write access to the Docker socket. This is shown in \autoref{listing:docker-socket-permissions}.

By default, \lstinline{root} and every user in the \lstinline{docker} group has read and write permissions to the socket.

\medskip

We can see who is in the \lstinline{docker} group by looking in \lstinline{/etc/group}.
\begin{lstlisting}[caption={See what users are in the \lstinline{docker} group.},captionpos=b]
$ grep docker /etc/group
docker:x:999:jvrancken
\end{lstlisting}
We see that only \lstinline{jvrancken} is part of the \lstinline{docker} group. It might also be interesting to look at which users have \lstinline{sudo} rights (in \lstinline{/etc/sudoers}). Users without \lstinline{sudo} but with Docker permissions still need to be considered \lstinline{sudo} users (see \autoref{subsection:docker-permissions}).

\medskip

It is also possible that the \lstinline{setuid} bit is set on the Docker client. In that case, we are also able to use Docker (as discussed in \autoref{subsubsection:setuid}).
\begin{lstlisting}[caption={Permissions without and with the \lstinline{setuid} bit.},captionpos=b]
(host)$ ls -l $(which docker)
-rwxr-xr-x 1 root root 88965248 nov 13 08:28 /usr/bin/docker
(host)# chmod +s $(which docker)
(host)$ ls -l $(which docker)
-rwsr-sr-x 1 root root 88965248 nov 13 08:28 /usr/bin/docker
\end{lstlisting}
