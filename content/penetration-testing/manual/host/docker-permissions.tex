\subsubsection{Who is Allowed to use Docker?}\label{subsubsection:docker-permissions-host}
Docker permissions are defined by the permission bits on the Docker socket (i.e.\ \lstinline{/var/run/docker.sock}). By default, the owner (\lstinline{root}) and the group (\lstinline{docker}) have read and write permissions. Meaning that \lstinline{root} and every user in the \lstinline{docker} group are allowed to interact with the Docker socket.

We can see who is in the \lstinline{docker} group by looking in \lstinline{/etc/group}.
\begin{lstlisting}[caption={See what users are in the \lstinline{docker} group.},captionpos=b]
$ grep docker /etc/group
docker:x:999:jvrancken
\end{lstlisting}
We see that only \lstinline{jvrancken} is part of the \lstinline{docker} group. It might also be interesting to look at which users have \lstinline{sudo} rights (in \lstinline{/etc/sudoers}). Users without \lstinline{sudo} but with Docker permissions still need to be considered \lstinline{sudo} users (see \autoref{subsection:docker-permissions}).

\hfill

It is possible that the Docker socket has permissions that give anybody permission to interact with Docker. Some people set the permissions to read and write for all users (i.e. \lstinline{666}). Giving all users read and write permission to the Docker socket allows them to use Docker.

\hfill

It is also possible that the \lstinline{setuid} bit is set on the Docker client. In that case, we are also able to use Docker (see \autoref{subsubsection:setuid}).

\begin{lstlisting}[caption={Permissions without and with the \lstinline{setuid} bit.},captionpos=b]
(host)$ ls -l $(which docker)
-rwxr-xr-x 1 root root 88965248 nov 13 08:28 /usr/bin/docker
(host)# chmod +s $(which docker)
(host)$ ls -l $(which docker)
-rwsr-sr-x 1 root root 88965248 nov 13 08:28 /usr/bin/docker
\end{lstlisting}
