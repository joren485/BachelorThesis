\subsubsection{Available Images \& Containers}
We should check which images and containers (both running and stopped) are available on the host. This will tell us more about the system we are testing.

\lstinline{docker images -a} will list all available images (including intermediate images) and \lstinline{docker ps -a} will list all (running and stopped) containers.

\begin{lstlisting}[caption={Listing all images and containers available.},captionpos=b]
(host)$ docker images -a
REPOSITORY  TAG     IMAGE ID        CREATED      SIZE
mariadb     latest  c1c9e6fba07a    2 weeks ago  355MB
ubuntu      latest  775349758637    4 weeks ago  64.2MB
alpine      3       965ea09ff2eb    6 weeks ago  5.55MB
alpine      latest  965ea09ff2eb    6 weeks ago  5.55MB
centos      latest  0f3e07c0138f    2 months ago 220MB
(host)$ docker ps -a --no-trunc --format="{{.Names}} {{.Command}} {{.Image}}"
database "docker-entrypoint.sh mysqld" mariadb:latest
\end{lstlisting}

\hfill

We should also look at the environment variables that have been passed to the containers, because environment variables are used to pass information (including passwords and secrets) to a container when it is created. Using \lstinline{docker inspect} we can see information about containers. Including the set environment variables.
\begin{lstlisting}[caption={List environment variables passed to Docker container.},captionpos=b]
(host)$ docker run --rm -e MYSQL_ROOT_PASSWORD=supersecret --name=database mariadb:latest
(host)$ docker inspect database | jq -r '.[0].Config.Env'
[
  "MYSQL_ROOT_PASSWORD=supersecret",
...
\end{lstlisting}

\hfill

The containers might have volumes. Those volumes tell us more about where sensitive and important data might be. We can also list the volumes using \lstinline{docker inspect}.
\begin{lstlisting}[caption={List bindmounts into Docker container.},captionpos=b]
(host)$ docker inspect database | jq -r '.[0].HostConfig.Binds'
[
  "/tmp/database/:/var/lib/mysql/"
]
\end{lstlisting}
