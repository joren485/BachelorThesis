\chapter{Introduction}
Secura, a company specializing in digital security, performs security assessments for clients. In these assessments, Secura evaluates vulnerable parts of the private and public network of their clients. They would like to improve those assessments by also looking into containerization software their clients may be running.

\hfill

Containerization software allows developers to package software into easily reproducible packages.
It removes the tedious process of installing the right dependencies to run software, because the dependencies and necessary files are neatly isolated in the container. This also allows multiple versions of the same software to run simultaneous on a server, because every instance runs in its own container.

\hfill

This thesis will focus on Docker, because it is the de facto industry standard for containerization software. It will focus on Linux, because Docker is developed for Linux (although a Windows version does exist\footnote{Docker on Windows actually runs inside a Linux virtual machine.}).

\hfill

This bachelor thesis will first describe necessary background information about containerization, Docker and penetration testing. I will then go into more detail about specific vulnerabilities and misconfigurations that are of interest during a security assessment. Finally, I will describe how a penetration tester can detect and use those vulnerabilities and misconfigurations during security assessments.

We will first look at some notations (\autoref{chapter:notation}) and the necessary background information (\autoref{chapter:background}). We will then go into more detail about the attack surface (\autoref{section:attack-surface-models}), specific vulnerabilities (\autoref{section:vulnerabilities}) and misconfigurations (\autoref{section:misconfigurations}). Finally, we will describe how these can be used during a penetration test (\autoref{chapter:pentesting}).
