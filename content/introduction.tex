\chapter{Introduction}
\todo[inline]{Rewrite this}
Secura, a company specializing in digital security, performs security assessments for clients. In these assessments, Secura evaluates the private and public networks of their clients. They would like to improve those assessments by also looking into containerization software their clients may be running.

\medskip

Containerization software allows developers to package software into easily reproducible packages. It removes the tedious, repetitive and error-prone process of installing the right dependencies to run software, because the dependencies (e.g.\ libraries and files) are neatly packaged in the container. These containers run isolated from each other and the host they run on.

\medskip

This thesis will focus on the de facto industry standard for containerization software, Docker. It will focus on Linux, because Docker is developed for Linux (although a Windows version does exist\footnote{Docker on Windows actually runs inside a Linux virtual machine.}). Throughout this thesis we will look at practical examples, so a good understanding of Linux is recommended.

\medskip

We will first look at the necessary background information (\autoref{chapter:background}). We will then go into more detail about the attack surface (\autoref{chapter:attack-surface-models}), misconfigurations (\autoref{section:misconfigurations}) and specific security related bugs (\autoref{section:bugs}). Finally, we will describe how these can be used during a penetration test (\autoref{chapter:pentesting}) and which tools might be useful to automate part of the process (\autoref{section:tools}).
