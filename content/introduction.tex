\chapter{Introduction}
Secura, a company specializing in digital security, performs security assessments for clients. In these assessments, Secura evaluates the security of the systems and applications of their clients. During these assessments, Secura encounters systems that use Docker, the de facto industry standard for containerization software. They would like to improve those assessments by better understanding how to test the security of systems that use Docker. This will help them perform better security assessments and make better recommendations to their clients. The goal of this research is to provide a methodology that penetration testers should use when testing the security of systems that use Docker.

\medskip

We will first introduce the necessary concepts (\autoref{chapter:notation}) and background information on containerization software and Docker (\autoref{chapter:background}). We will then go into more detail about the attacker models (\autoref{chapter:attack-surface-models}) that we should consider when thinking about containers. In \autoref{chapter:vulnerabilities} we look at vulnerabilities, both misconfigurations (\autoref{section:misconfigurations}) and security related bugs (\autoref{section:bugs}), that exist in Docker. We will map these to relevant guidelines from a best practices guide that is used by companies like Secura, the CIS Docker Benchmark. We will discuss how the vulnerabilities can be identified during a penetration test (\autoref{chapter:pentesting}). Most importantly, this research contributes a checklist of questions that penetration testers should ask themselves when they systems that use Docker (\autoref{chapter:checklist}). For each question, a simple way to answer the question and a reference to the relevant section in this thesis is given. Finally, we will look at out of scope but interesting ideas to extend this research (\autoref{chapter:futurework}), other research about the security of Docker (\autoref{chapter:relatedwork}) and the takeaways of this thesis from both an offensive and a defensive perspective (\autoref{chapter:conclusions}).

We will focus on Linux, because Docker is developed for Linux (although non-Linux Docker versions do exist\footnote{Docker on non-Linux systems runs inside a Linux virtual machine.}). Throughout this thesis we will look at practical examples, so a good understanding of Linux is helpful.
