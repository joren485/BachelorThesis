\section{Protection Mechanisms}

To significantly reduce that risk that (future) vulnerabilities and misconfigurations pose to a system with Docker, there are multiple protections built into Docker and the Linux kernel itself. In this section, we will look at the most well-known and important protections.

\hfill

It should be noted that because these protections add complexity and features, some vulnerabilities focus solely on bypassing one or more protections. A good example of this is CVE--2019--5021 (see \autoref{CVE-2019-5021}).

\subsection{Capabilities}\label{capabilities}
To allow or disallow a process to use specific privileged functionality, the Linux kernel has capabilities\footnote{See the \lstinline{man page} of \lstinline{capbilities}}. A capability is a granular way of giving certain privileges to processes. A capability allows a process to perform privileged action without giving the process full \lstinline{root} rights. For example, if we want a process to be able to create its own packets but not read sensitive files we give it the \lstinline{CAP_NET_RAW} capability.

\hfill

By default every Docker container is started with minimum capabilities. The default capabilities can be found in the Docker code\footnote{\url{https://github.com/moby/moby/blob/master/oci/caps/defaults.go}}. It is possible to add or remove capabilities at runtime using the \lstinline{--cap-add} and \lstinline{--cap-drop}\cite{More-Secure-Non-Root-Container} arguments.

\subsection{Secure Computing Mode}
Secure Computing Mode (seccomp for short), like capabilities, is a built-in way to limit the privileged functionality that a process is allowed to use. Where capabilities limit functionality (like reading privileged files), Secure Computing Mode limits specific \lstinline{syscalls}. This allows for very granular security control. It does this by using whitelists of \lstinline{syscalls} (called profiles).
To setup a strict, but still functional seccomp profile requires very specific knowledge of which \lstinline{syscalls} are used by a program. This makes it quite complex to setup

\hfill

The default seccomp profile that processes in Docker containers get is available in the source code\footnote{\url{https://github.com/moby/moby/blob/master/profiles/seccomp/default.json}}. To pass a custom seccomp profile the \lstinline{--security-opt seccomp} can be used. 


\subsection{AppArmor}
AppArmor (which stands for Application Armor) is a Linux kernel module that allows application-specific limitations of files and system resources. 

\hfill

Docker adds a default AppArmor profile to every container. This is profile generated at runtime based on a template\footnote{\url{https://github.com/moby/moby/blob/master/profiles/apparmor/template.go}}.

\hfill

\lstinline{bane}\footnote{\url{https://github.com/genuinetools/bane}} is a program that generates AppArmor profiles for containers.

\subsection{SELinux}
SELinux (which stands for Security-Enhanced Linux) is a set of changes to the Linux kernel that support system-wide access control for files and system resources. It is available by default on some Linux distributions.

\hfill

Docker does not enable SELinux support by default, but it does provide a SELinux policy\footnote{\url{https://www.mankier.com/8/docker_selinux}}.

\subsection{Non-root user in containers}
By default processes in Docker containers are executed as \lstinline{root} (the \lstinline{root} user of that \lstinline{namespace}), because the process is isolated from the host system. However, as we will see there exist many ways to escape containers. Most of those ways require \lstinline{root} privileges. That is why it is recommended to run processes in containers using non-root. If the container gets compromised in any way, the attacker cannot escape because the user is non-root.
