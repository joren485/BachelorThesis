\subsection{CVE--2019--5021}\label{subsection:CVE-2019-5021}
The Docker image for Alpine Linux (one of the most used base images) had a problem where the password of the \lstinline{root} user in the container is left empty. In Linux it is possible to disable a password and to leave it blank. A disabled password cannot be used, but a blank password equals an empty string. This allows non-\lstinline{root} users to gain \lstinline{root} rights by supplying an empty string.

\hfill

It is still possible to use the vulnerable images (\lstinline{alpine:3.3}, \lstinline{alpine:3.4} and \lstinline{alpine:3.5}).
\begin{lstlisting}[caption={The Docker image of Alpine Linux 3.5 has an empty password.},captionpos=b]
(host)$ docker run -it --rm alpine:3.5 cat /etc/shadow
root:::0:::::
...
(host)$ docker run -it --rm alpine:3.5 sh
(cont)# apk add --no-cache linux-pam shadow
...
(cont)# adduser test
...
(cont)# su test
Password:
(cont)$ su root
(cont)#
\end{lstlisting}

\subsubsection*{Side note about the CVSS score of CVE--2019--5021}

This vulnerability has a CVSS score of 9.8 (and a 10 in CVSS 2)\footnote{\url{https://nvd.nist.gov/vuln/detail/CVE-2019-5021}} out of a maximum score of 10. Such a high CVSS score means that this is considered an extremely high-risk vulnerability. But in actuality, this vulnerability is only risky in very specific cases.

An empty \lstinline{root} password sounds very dangerous, but it really is not that dangerous in an isolated environment (e.g.\ a container) that runs as \lstinline{root} (inside the container) by default. This vulnerability will only be dangerous in very specific cases.

For example, if we create a Docker image based on \lstinline{alpine:3.5} that uses a non-\lstinline{root} user by default. If an attacker finds a way to execute code in the container, this vulnerability will allow them to escalate their privileges from the non-\lstinline{root} user to \lstinline{root}, but the attacker will still need to find a way to escape the container. Being able to execute code as \lstinline{root} will help them with escaping the container, but it does not guarantee it. This example shows that this vulnerability is dangerous, but only in a scenario where it is chained using other vulnerabilities.
