\subsection{\texorpdfstring{\lstinline{iptables}}{iptables} Bypass}\label{subsection:iptables}
The Linux kernel has a built-in firewall, called \lstinline{Netfilter} which can be configured with a program called \lstinline{iptables}. This firewall consists of multiple chains of rules which are stored in tables. Each table has a different purpose. For example, there is a \lstinline{nat} table for address translation and a \lstinline{filter} table for traffic filtering (which is the default).
Each table has chains of ordered rules which also have a different purpose. For example, there are the \lstinline{OUTPUT} and \lstinline{INPUT} chains in the \lstinline{filter} table that are meant for all outgoing and incoming traffic, respectively.
It is possible to configure these rules using a program called \lstinline{iptables}. All Linux based firewalls (e.g. \lstinline{ufw}) use \lstinline{iptables} as their backend.

\medskip

When the Docker daemon is started, it sets up its own chains and rules to create isolated networks. The way it sets up its rules completely bypasses other in the firewall (because they are setup before the other rules) and by default the rules are quite permissive. This is by design, because the network stack of the host and the container are separate, including the firewall rules. Users of Docker might be under the impression that firewall rules set by the host are applicable to everything running on the host (including containers). This is not the case for Docker containers and could lead to unintended exposed ports.

It is, however, a bit counterintuitive, because we would assume that if a firewall rule is set on the host, it would apply to everything running on that host (including containers).

\medskip

We will look at the following example of bypassing a firewall rule with Docker.

\begin{lstlisting}[caption={Bypass \lstinline{iptables} firewall rules using Docker.},captionpos=b,label={listing:iptables-bypass}]
(host)# iptables -A OUTPUT -p tcp --dport 80 -j DROP
(host)# iptables -A FORWARD -p tcp --dport 80 -j DROP
(host)$ curl http://httpbin.org/get
curl: (7) Failed to connect to httpbin.org port 80: Connection timed out
(host)$ docker run -it --rm ubuntu /bin/bash
(cont)# apt update
...
(cont)# apt install curl
...
(cont)# curl http://httpbin.org/get
{
  "args": {},
  "headers": {
    "Accept": "*/*",
    "Host": "httpbin.org",
    "User-Agent": "curl/7.58.0"
  },
...
  "url": "https://httpbin.org/get"
}
\end{lstlisting}

In \autoref{listing:iptables-bypass} we first setup rules to drop all outgoing (including forwarded) traffic on port \lstinline{80} (the standard \lstinline{HTTP} port). Then, we try to request a webpage (\lstinline{http://httpbin.org/get}) on the host. As expected, the HTTP service is not reachable for us. If we then try to make the exact same request in a container, it works.

\medskip

The CIS Docker Benchmark does not cover this problem. It, however, does have guidelines that ensures this problem exists. Guideline 2.3 (Ensure Docker is allowed to make changes to iptables) recommends that the Docker daemon is allowed to change the firewall rules. Guideline 5.9 (Ensure that the host's network \lstinline{namespace} is not shared) recommends to not use the \lstinline{--network=host} argument, to make sure the container is put into a separate network stack. 

These are a good recommendations, because following them removes the need to configure a containerized network stack ourselves. However, it also isolates the firewall rules of the host from the containers.
