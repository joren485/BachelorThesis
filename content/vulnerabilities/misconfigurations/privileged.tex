\subsection{\texorpdfstring{\lstinline{--privileged}}{--privileged} Flag}\label{subsection:privileged}

Docker has a special privileged mode~\cite{Docker-in-Docker-blog}. This mode is enabled if a container is created with the \lstinline{--privileged} flag and it enables access to all host devices and kernel capabilities. This is a powerful mode and enables some useful features (e.g.\ building Docker images inside a Docker container). The downside of privileged mode is that all functionality of the kernel allows an attacker inside the container to escape and access the host.

\medskip

An example of this, is abusing a feature in \lstinline{cgroups}~\cite{CGroup-Docs}. Whenever a \lstinline{cgroup} is released due to an absence of any running processes, it is possible to run a command (called a \lstinline{release_agent}). It is possible to define such a \lstinline{release_agent} in a privileged docker. If the \lstinline{cgroup} is released, the command is run on the host~\cite{TrailOfBits-Docker-Escape}.

\medskip

We can look at a proof of concept of this attack developed by security researcher Felix Wilhelm~\cite{Felix-Wilhem-Tweet}.
\begin{lstlisting}[numbers=left, escapechar=@, caption={Privileged container escape using \lstinline{cgroups}.},captionpos=b,label={listing:privileged-escape}]
(host)$ docker run -it --rm --privileged ubuntu:latest bash
(cont)# d=`dirname $(ls -x /s*/fs/c*/*/r* |head -n1)`@\label{line:privileged:d}@
(cont)# mkdir -p $d/w@\label{line:privileged:w}@
(cont)# echo 1 >$d/w/notify_on_release@\label{line:privileged:enable_release_agent}@
(cont)# t=`sed -n 's/.*\perdir=\([^,]*\).*/\1/p' /etc/mtab`@\label{line:privileged:t}@
(cont)# touch /o
(cont)# echo $t/c >$d/release_agent@\label{line:privileged:release_agent}@
(cont)# printf '#!/bin/sh\nps >'"$t/o" >/c@\label{line:privileged:c}@
(cont)# chmod +x /c
(cont)# sh -c "echo 0 >$d/w/cgroup.procs"@\label{line:privileged:procs}@
(cont)# sleep 1
(cont)# cat /o
\end{lstlisting}

The proof of concept in \autoref{listing:privileged-escape} is a bit hard to read, because it uses a lot of \lstinline{Bash} syntax to abbreviate the commands. We will go over the commands line by line to see what each line does.

On \autoref{line:privileged:d}, the first \lstinline{cgroup} with a \lstinline{release_agent} is added to variable \lstinline{d}. A subgroup \lstinline{w} is added to the \lstinline{cgroup} of \lstinline{d} (\autoref{line:privileged:w}) and the execution of the \lstinline{release_agent} is enabled for \lstinline{w} (\autoref{line:privileged:enable_release_agent}). The location of the container filesystem on the host filesystem is added to variable \lstinline{t} (\autoref{line:privileged:t}). A script (\lstinline{/c}), containing only the line ``\lstinline{ps > $t/o}'', is created (\autoref{line:privileged:release_agent}) and is added as the \lstinline{release_agent} (\autoref{line:privileged:c}). A process is started to add itself to the \lstinline{w} (by writing ``0'' to the \lstinline{cgroup.procs} file of \lstinline{w}) on \autoref{line:privileged:procs}. After the \lstinline{release_agent} (\lstinline{/c}) is executed, we can see all the processes on the host in \lstinline{/o}.

\medskip

The \lstinline{--privileged} flag is covered by two CIS Docker Benchmark guidelines. Guideline 5.4 (Ensure that privileged containers are not used) recommends to not create containers with privileged mode. Guideline 5.22 (Ensure that docker exec commands are not used with the privileged option) recommends to not execute commands in running containers (with \lstinline{docker exec}) in privileged mode.
