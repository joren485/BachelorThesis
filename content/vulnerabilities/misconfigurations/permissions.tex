\subsection{Docker Permissions}\label{subsection:docker-permissions}
A very common (and notorious) misconfiguration is giving unprivileged users access to Docker, which allows them to create, start and otherwise interact with Docker containers (through the Docker daemon). This is very dangerous because this allows the unprivileged users to access all files as \lstinline{root}. The Docker documentation says\footnote{\url{https://docs.docker.com/engine/security/security/}}:
\begin{quote}
First of all, only trusted users should be allowed to control your Docker daemon. This is a direct consequence of some powerful Docker features. Specifically, Docker allows you to share a directory between the Docker host and a guest container; and it allows you to do so without limiting the access rights of the container. This means that you can start a container where the /host directory is the / directory on your host; and the container can alter your host filesystem without any restriction.
\end{quote}

In short, because the Docker daemon runs as \lstinline{root}, if a user adds a directory as a volume to a container, that file is accessed as \lstinline{root}. There a few ways for unprivileged users to access Docker. In this section we will look at those.

\subsubsection{\texorpdfstring{\lstinline{docker}}{docker} group}
Every user in the \lstinline{docker} group is allowed to use Docker (see \autoref{background:docker-socket} on why this is). This allows simple access management of Docker usage. Sometimes a system administrator does not want to do proper access management and adds every user to the \lstinline{docker} group, because that allows everything to run smoothly. This misconfiguration, however allows every user to access every file on the system, as illustrated in \autoref{listing:docker-group}.

\medskip

Let's say we want the password hash of user \lstinline{admin} on a system where we do not have \lstinline{sudo} privileges, but we are a member of the \lstinline{docker} group.

\begin{lstlisting}[caption={Docker \lstinline{group} exploit example.},captionpos=b,label={listing:docker-group}]
(host)$ sudo -v
Sorry, user unpriv may not run sudo on host.
(host)$ groups | grep -o docker
docker
(host)$ docker run -it --rm -v /:/host ubuntu:latest bash
(cont)# grep admin /host/etc/shadow
admin:$6$VOSV5AVQ$jHWxAVAUgl...:18142:0:99999:7:::
\end{lstlisting}

In \autoref{listing:docker-group} we first check our permissions. We do not have \lstinline{sudo} permissions, but we are a member of the \lstinline{docker} group. This allows us to create a container with \lstinline{/} mounted as volume and access any file as \lstinline{root}. This includes the file storing password hashes (i.e.\ \lstinline{/etc/passwd}).

\medskip

This is covered by the CIS Docker Benchmark guideline 1.2.2 (Ensure only trusted users are allowed to control Docker daemon).

\subsubsection{World Read and Writable Docker Socket}
By default, only \lstinline{root} and every user in the \lstinline{docker} group have access to Docker, because they have read and write access to the Docker socket. However, some administrators administrators set the permissions to read and write for all users (i.e. \lstinline{666}), giving all users access to the Docker daemon.

\begin{lstlisting}[caption={All users can use Docker if they have read and write access to the Socket},captionpos=b,label={listing:read-write-socket}]
(host)$ groups | grep -o docker
(host)$ ls -l /var/run/docker.sock
srw-rw-rw- 1 root docker 0 Dec 20 13:16 /var/run/docker.sock
(host)$ docker run -it --rm -v /:/host ubuntu:latest bash
(cont)# grep admin /host/etc/shadow
admin:$6$VOSV5AVQ$jHWxAVAUgl...:18142:0:99999:7:::
\end{lstlisting}

In \autoref{listing:read-write-socket}, we see that we are not a member of the Docker group, but because every user has read and write access (i.e.\ the read and write permissions are set for \lstinline{other}) we are still able to use Docker.

\subsubsection{\texorpdfstring{\lstinline{setuid}}{setuid} bit}\label{subsubsection:setuid}
Another way system administrators might skip proper access management is to set the \lstinline{setuid} bit on the \lstinline{docker} binary.

\medskip

The \lstinline{setuid} bit is a permission bit in Unix, that allows users to run binaries as the owner (or group) of the binary instead of themselves.
This is very useful in specific cases. For example, users should be able to change their own passwords, but should not be able to read password hashes of other users. That is why the \lstinline{passwd} binary (which is used to change a users password) has the \lstinline{setuid} bit set. A user can change their password, because \lstinline{passwd} is run as \lstinline{root} (the owner of \lstinline{passwd}) and, of course, \lstinline{root} is able to read and write to the password file. In this case the \lstinline{setuid} bit is not a security issue, because \lstinline{passwd} asks for the user's password itself and only writes to specific entries in the password file.

\medskip

If a system is misconfigured by having the \lstinline{setuid} bit set for the \lstinline{docker} binary, a user will be able to execute Docker as \lstinline{root} (the owner of \lstinline{docker} binary). Just like before, we can easily recreate this attack.

\begin{lstlisting}[caption={Docker \lstinline{setuid} exploit example.},captionpos=b, label={listing:setuid}]
(host)$ sudo -v
Sorry, user unpriv may not run sudo on host.
(host)$ groups | grep -o docker
(host)$ ls -halt /usr/bin/docker
-rwsr-xr-x 1 root root 85M okt 18 17:52 /usr/bin/docker
(host)$ docker run -it --rm -v /:/host ubuntu:latest bash
(cont)# grep admin /host/etc/shadow
admin:$6$VOSV5AVQ$jHWxAVAUgl...:18142:0:99999:7:::
\end{lstlisting}

In \autoref{listing:setuid} we see that we are not a part of the \lstinline{docker} group, but we can still run \lstinline{docker} because the \lstinline{setuid} bit (and the execute bit for all users) is set.

\medskip

This is not covered by the CIS Docker Benchmark guidelines. There are multiple guidelines about correct file and directory permissions, but none cover the binaries.
