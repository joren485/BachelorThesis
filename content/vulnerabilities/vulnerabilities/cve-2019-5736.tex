\subsection{CVE--2019--5736}
A very serious vulnerability was discovered in runC (which is used by Docker) that allows containers to overwrite the runC binary on the host\cite{CVE-2019-5736-DragonSector}\cite{CVE-2019-5736-Github}\cite{CVE-2019-5736-Twistlock}. Whenever a Docker is created or when \lstinline{docker exec} is used, a runC process is run. This runC process bootstraps the container. It creates all the necessary restrictions and then executes the process that needs to run in the container. The researches found that it is possible to make runC execute itself in the container, by telling the container to start \lstinline{/proc/self/exe} which during the bootstrap is symlinked to the runC binary. If this happens, \lstinline{/proc/self/exe} in the container will point to the runC binary on the host. The root user in the container is then able to replace the runC host binary using that reference. The next time runC is executed (a container is created or \lstinline{docker exec} is run), the overwritten binary is run instead.
This, of course, is very dangerous because it allows a malicious container to execute code on the host.
