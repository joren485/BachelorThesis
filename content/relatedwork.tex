\chapter{Related Work}\label{chapter:relatedwork}
A lot has been written about Security and Docker. Most of it focuses on the defensive perspective, summarizing existing material or on specific parts of the Docker ecosystem.

\medskip

In their 2018 paper, A. Martin et al.\ review and summarize the Docker ecosystem, its vulnerabilities and relevant literature~\cite{Docker-Ecosystem-Vulnerability-Analysis}.

A comparison of OS-level virtualization technologies (e.g.\ containers) is given in~\cite{Security-OS-level-Virtualization}.

An in-depth look at the security of the Linux features (e.g. \lstinline{namespaces}) is given in~\cite{Analysis-Docker-Security}.

A more flexible Docker image hardening technique using SELinux policies is proposed in~\cite{DockerPolicyModules}.

In~\cite{Defense-Docker-Escape} Z. Jian and L. Chen look at a Linux \lstinline{namespace} escape and look at defenses to protect against such an escape.

Memory denial of service attacks from the container to the host and possible protections against it are described in~\cite{Securing-Docker-Containers-from-DOS}.

A quick overview of penetration testing of Docker environments is given in~\cite{Research-Pentesting-Docker-Environment}.

In~\cite{Study-Vulnerabilities-Docker-Hub} the authors show the results of their publicly available Docker image scan. They have looked at 356218 images and have identified and analyzed vulnerabilities within them.

The research in~\cite{To-Docker-Not-To-Docker} looks at the security implications of practical use-cases of using a Docker environment.

The National Computing Center (NCC) group has published multiple papers on the security of Docker, both from a defensive~\cite{Understanding-and-Hardening-Linux-Containers} and offensive~\cite{Abusing-Containers} perspective.
