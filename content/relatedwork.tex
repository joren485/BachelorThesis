\chapter{Related Work}\label{chapter:relatedwork}
A lot has been written about Security and Docker. Most of it focuses on defensive perspective, summarizing existing material or very specific parts of the Docker ecosystem.

In their 2018 paper, A. Martin et al review and summarize the Docker ecosystem, vulnerabilities and literature about the security of Docker\cite{Docker-Ecosystem-Vulnerability-Analysis}. A comparison of OS-level virtualization (e.g.\ containers) technologies is given in\cite{Security-OS-level-Virtualization}. An in-depth look at the security of the Linux features (e.g. \lstinline{namespaces}) is given in\cite{Analysis-Docker-Security}. A more flexible Docker image hardening technique using SELinux policies is propose in\cite{DockerPolicyModules}. In\cite{Defense-Docker-Escape} Z. Jian and L. Chen look at a Linux \lstinline{namespace} escape and looks at defenses to protect from the attack. Memory denial of service attacks from the container to the host and possible protections are described in\cite{Securing-Docker-Containers-from-DOS}. A very quick overview of penetration testing Docker environments is given in\cite{Research-Pentesting-Docker-Environment}. In\cite{Study-Vulnerabilities-Docker-Hub} the authors show the results of their publicly available Docker image scan. They have looked at 356218 images and identified and analyzed vulnerabilities within the images.\cite{To-Docker-Not-To-Docker} looks at the security implications of practical use-cases of using a Docker environment. The NCC group has published multiple papers on the security of Docker, both from a defensive\cite{Understanding-and-Hardening-Linux-Containers} and offensive\cite{Abusing-Containers} perspective.
