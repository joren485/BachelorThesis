\chapter{Conclusions}\label{chapter:conclusions}
Using containers creates more secure environments, because it isolates software. However, using containers also increases the attack surface and risks, because containerization software also adds extra layers of abstraction and complexity. This poses challenges for both attackers and defenders of Docker systems. We will look at the findings of this bachelor thesis from both perspectives.

\section{Takeaways from an Offensive Perspective}

\begin{itemize}
    \item When inside a container, an attacker wants to escape and see what other containers it can reach. As we saw in \autoref{chapter:vulnerabilities} there are many ways to escape a Docker container.

    \item When on a host that runs Docker, an attacker wants to access privileged information through the Docker daemon. Because being able to use Docker is equivalent to having \lstinline{root} permissions, an attacker wants to find a way to get access to Docker and exploit it.

    \item Misconfigurations are more interesting from an offensive perspective, because they are harder to fix. Security bugs are simply fixed by updating Docker.

    \item Use the checklist provided in \autoref{chapter:checklist} to manually and systematically identify weak spots and misconfigurations that are present in containers or hosts with a Docker installation.

    \item Docker Bench for Security (see \autoref{tools:bench}) is a tool that automates the checking of every guideline in the CIS Docker Benchmark (which is too long to check manually) to see if there are obvious and interesting configuration mistakes in a Docker installation.

\end{itemize}

\section{Takeaways from a Defensive Perspective}

\begin{itemize}
    \item Docker is a very complex ecosystem. Understanding how it works internally, is crucial to building stable, secure systems.

    \item Keep Docker up to date to minimize the risk of software bugs.

    \item Know what are the best practices to configure Docker and be aware of the and common misconfigurations.

    \item Best practice guideline lists (e.g.\ CIS Docker Benchmark) are a good start, but are most likely not completely fitted to your use-case, very long and incomplete.

    \item (Automatically) analyze Docker images using static image analysis tools to discover misconfigurations and bugs, before the image is deployed.
\end{itemize}
