\chapter{Conclusions}\label{chapter:conclusions}
\todo[inline]{Luko: security takeaway 4 voelt niet echt als een takeaway maar als een doe dit, het valt vooral op in de rest van het rijtje}
\todo[inline]{Rewrite}
\todo[inline]{Geert: Better story in the bullet points}
\todo[inline]{Geert: A paragraph per bullet point}
\todo[inline]{CIS Benchmark does not cover everything}
\todo[inline]{Idea: Do not rely solely on guideline lists like CIS, because they are not cover everything}
\todo[inline]{Tools are not really useful, except for Bench. There are no tools that replace manual assessments.}
Containers help people create more secure environments, because it isolates software. However, using containers also increases the attack surface and risks, because containerization software also adds extra layers of abstraction and complexity. This poses challenges for both attackers and defenders of Docker systems. We will look at the findings of this bachelor thesis from both perspectives.

\section{Takeaways from an Offensive Perspective}

\begin{itemize}
    \item When inside a container, an attacker wants to escape and see what other containers it can reach. As we saw in \autoref{chapter:vulnerabilities} there are many ways to escape a Docker container.

    \item When on a host that runs Docker, an attacker wants to access privileged information through the Docker daemon. Because being able to use Docker is equivalent to having \lstinline{root} permissions, an attacker wants to find a way to get access to Docker and exploit it.

    \item Misconfigurations are more interesting from an offensive perspective, as these are harder to fix. Security bugs are simply fixed by updating Docker.

    \item Use the checklist provided in \autoref{chapter:checklist} to manually and systematically identify weak spots and misconfigurations that are present in containers or hosts with a Docker installation.

    \item Docker Bench for Security (see \autoref{tools:bench}) is a tool that automates the checking of every guideline in the CIS Docker Benchmark (as manually checking every guideline in the CIS Docker Benchmark takes a long time). It audits a host system for guidelines that are not followed and generates a report on its findings. This is useful to see if there are obvious and interesting configuration mistakes in a Docker installation.

\end{itemize}

\section{Takeaways from a Defensive Perspective}

\begin{itemize}
    \item Docker is not just one program, but a whole ecosystem. Understanding how it works internally is crucial to building stable, secure systems.

    \item Keep Docker up to date to minimize the risks of software bugs.

    \item Know what the best practices to configure Docker are and be aware of common misconfigurations.

    \item Best practice guideline lists (e.g.\ CIS Docker Benchmark) are a good start, but are not well fitted to every use-case, long and incomplete.

    \item Analyze Docker images (automatically) using static image analysis tools (see \autoref{subsection:image-analysis-tools}) to discover misconfigurations and bugs, before the images are deployed.
\end{itemize}
