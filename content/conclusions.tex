\chapter{Conclusions}\label{chapter:conclusions}
Containers help people create more secure environments, because it isolates software. However, using containers also increases the attack surface and risks, because containerization software also adds extra layers of abstraction and complexity. This poses challenges for both attackers and defenders of Docker systems. We will look at the findings of this research from both perspectives.

\section{Takeaways from an Offensive Perspective}
When performing a penetration test, it is important to be aware of the following points.
\begin{itemize}
    \item \textbf{Be aware of the attacker models described in \autoref{chapter:attack-surface-models}.}\\
        As we saw in \autoref{chapter:attack-surface-models} there are two attacker models: container escapes which focus on escaping the isolation of a container and Docker daemon attacks which focus on using an installation of Docker on a host to gain access to privileged data. It is important to know during an penetration test which one is relevant.

    \item \textbf{Misconfigurations are more interesting than security related software bugs.}\\
        We looked at many vulnerabilities in \autoref{chapter:vulnerabilities}. We looked at both misconfigurations and bugs. Both the misconfigurations and the bugs pose a danger. However, the misconfigurations are more interesting to a attacker, because they are harder to fix. Software bugs are easily fixed by using the latest version of Docker, while misconfigurations require changing the way Docker is used.

    \item \textbf{Do not solely rely on lists of guidelines.}\\
        Lists of guidelines (e.g.\ the CIS Docker Benchmark) are a good starting point to identify potential vulnerable parts of a system. However, as we saw with the CIS Docker Benchmark, they are not exhaustive.

    \item \textbf{Do not solely rely on tools to automate security assessments.}\\
        Tools (e.g.\ we looked at in \autoref{section:tools}) help automate penetration tests. They are useful because they save time and systematically look at target systems. They, however, fall short when it comes to identifying more complex vulnerabilities and covering larger parts of a system. Most tools are designed to scan for or exploit only one specific vulnerability. A detailed, manual assessment will take more time, but will also uncover more vulnerabilities and weak spots.

    \item \textbf{Use the checklist provided in \autoref{chapter:checklist}.}\\
        The checklist in \autoref{chapter:checklist} provides an attacker with three lists of interesting questions about the reconnaissance and exploitation of a system using Docker. The first list is meant to check whether an attacker is running inside a container or on a host. The second and third list are meant to gather data and identify vulnerabilities inside containers and on host systems, respectively.
\end{itemize}

\pagebreak

\section{Takeaways from a Defensive Perspective}
Although, this research focuses on an offensive perspective on Docker, it can be used to harden and secure a system that uses Docker. When designing or maintaining a system that uses Docker it is important to keep the following points in mind.
\begin{itemize}
    \item \textbf{Using Docker adds a layer of isolation to your software.}\\
        Docker, like all containerization software, adds a layer of isolation. This adds security, because software is isolated from the host system.

    However, this also adds a layer of abstraction to the system. Instead of running software directly on a host, it runs inside of a container on a host. This layer of abstraction increases the attack surface of the system.

    \item \textbf{Always use the latest version of Docker.}\\
        As we saw in \autoref{chapter:vulnerabilities}, there are many vulnerabilities that pose a risk to systems that use Docker. It is possible to significantly reduce the risk of one type of vulnerability we looked at. Software bugs are easily fixed (by the user) by always using the latest version of Docker.

    \item \textbf{The checklist in \autoref{chapter:checklist} will help us look at a system like an attacker.}\\
        If we know how an attacker looks at our system, we can more easily identify the parts that an attacker would target. The checklist of questions in \autoref{chapter:checklist} will help us look at a system like an attacker.

    \item \textbf{Do not solely rely on lists of guidelines.}\\
        Lists of guidelines (e.g.\ the CIS Docker Benchmark) are a good starting point to harden a system. However, as we saw with the CIS Docker Benchmark, they are not exhaustive.
\end{itemize}
