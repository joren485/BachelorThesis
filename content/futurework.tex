\chapter{Future Work}\label{chapter:futurework}
This thesis looks at how to penetration test Docker systems. During the research and writing, I came across some interesting topics that go beyond the scope of this thesis.

\section{Orchestration Software}
In modern software deployment, containerization is only part of the puzzle. Large companies run a lot of different software and each instance needs to support many connections and a lot of computing power. That means that for many applications, multiple containers of the same image are run to handle everything. To manage all of those containers there is orchestration software. The most famous being Kubernetes and Docker Swarm.

It would be interesting to continue this research by looking at how we could perform penetration tests on orchestration software and how orchestration software impacts the security of systems.

\section{Docker on Windows}
This bachelor thesis looks at Docker on Linux, because Docker is developed for Linux. However, it is also possible to run Docker on Windows (sort of). Because Docker uses very specific kernel features from Linux, Docker on Windows runs in a Linux virtual machine. That way Windows users can still use Docker exactly as they would use it on Linux (because they practically are).

\medskip

Some of the vulnerabilities and misconfigurations that are described in this thesis might also be relevant on Windows. There are also vulnerabilities that are specific to Docker on Windows (e.g.\ CVE--2019--15752 and CVE--2018--15514).

\medskip

It would be interesting (and relevant to penetration testing) to continue this research by specifically looking at Docker on Windows.

\section{Comparison of Virtualization and Containerization}
This thesis looks at the security of Docker. As stated in the background, virtualization is another way to achieve isolation. A lot has been written about the comparison of virtualization and containerization\cite{Virtualization-vs-Containerization-to-Support-PaaS}\cite{Hypervisor-vs-Lightweight-Virtualization}\cite{Updated-Performance-Comparison-Virtual-Machines-Containers}.
However, it would be interesting to specifically compare the isolation and security that virtualization offers to the isolation and security that containerization offers.

\section{Condense the CIS Docker Benchmark}\label{futurework:CIS}
The CIS Docker Benchmark contains 115 guidelines with their respective documentation.
This makes it a 250+ page document. This is not practical for developers and engineers (the intended audience). It would be much more useful to have a smaller, better sorted list that only contains common mistakes and pitfalls to watch out for.

\medskip

The CIS Benchmarks do indicate the importance of each guideline.
With Level 1 indicating that the guideline is a must-have and Level 2 indicating that the guideline is only necessary if extra security is needed. However, only twenty-one guidelines are actually considered Level 2. All the other guidelines are considered Level 1. This still leaves the reader with a lot of guidelines that are considered must-have.

\medskip

It would be a good idea to split the document into multiple sections. The guidelines can be divided by their importance and usefulness. For example, a three section division can be made.

\medskip

The first section would describe obvious and basic guidelines that everyone should follow (and probably already does). This is an example of guidelines that would be part of this section:
\begin{itemize}
    \item 1.1.2: Ensure that the version of Docker is up to date
    \item 2.4: Ensure insecure registries are not used
    \item 3.1: Ensure that the docker.service file ownership is set to root:root
    \item 4.2: Ensure that containers use only trusted base images
    \item 4.3: Ensure that unnecessary packages are not installed in the container
\end{itemize}

\medskip

The second section would contain guidelines that are common mistakes and pitfalls. These guidelines would be the most useful to the average developer. For example:
\begin{itemize}
    \item 4.4 Ensure images are scanned and rebuilt to include security patches
    \item 4.7 Ensure update instructions are not use alone in the Dockerfile
    \item 4.9 Ensure that COPY is used instead of ADD in Dockerfiles
    \item 4.10 Ensure secrets are not stored in Dockerfiles
    \item 5.6 Ensure \lstinline{sshd} is not run within containers
\end{itemize}

\medskip

The last section would describe guidelines that are intended for systems that need extra hardening. For example:
\begin{itemize}
    \item 1.2.4 Ensure auditing is configured for Docker files and directories
    \item 4.1 Ensure that a user for the container has been created
    \item 5.4 Ensure that privileged containers are not used
    \item 5.26 Ensure that container health is checked at runtime
    \item 5.29 Ensure that Docker's default bridge ``\lstinline{docker0}'' is not used
\end{itemize}

\section{Docker Man-in-the-Middle}
In \autoref{subsection:arp-spoofing} we looked at performing a man-in-the-middle attack using ARP spoofing. It would be interesting to look at more complex man-in-the-middle attacks. For example, capturing all traffic of a webserver running in a Docker or modifying traffic.

\section{Docker Specific Penetration Testing Tool}
In \autoref{section:tools} we discuss multiple tools that automate part of security assessments. However, we see that some tools are not interesting from an attackers perspective and most tools focus on very specific vulnerabilities. It would be interesting to develop a new tool or extend an existing tool that focuses on the full spectrum of exploits and vulnerabilities of one or more attacker model (\autoref{chapter:attack-surface-models}) and not only specific vulnerabilities.
