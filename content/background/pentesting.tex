\section{Penetration Testing}
Penetration testing (or pentesting) is an simulated attack to test the security and discover vulnerabilities in systems. The goal of a penetration test is to find the weak points in a system to be able to fix and secure them, before a malicious actor finds them.

\hfill

A penetration test consists of five steps:
\begin{enumerate}
    \item Reconnaissance
    \item Exploitation
    \item Post-exploitation
    \item Exfiltration
    \item Clean Closure
\end{enumerate}

\begin{itemize}
    \item Black Box: The assessor does not get any information about the system they are going to test.  
    \item Grey Box: The assessor gets some information (e.g.\ credentials) about the system.
    \item White Box/Crystal Box:  The assessor gets all information about the system and it's internal working.
    \item Infrastructure Research: An assessment of the infrastructure of a system, without having knowledge of the internal workings.
    \item Configuration Research: A white box infrastructure research. The assessor gets information on the complete infrastructure of a system.
    \item Internal Assessment: An assessment of the internal network of a system. Most of the time the assessment has a clear goal (e.g.\ finding certain sensitive information).
    \item Red Teaming: An security assessment with a specific goal that takes weeks or months. The focus heavily lies on stealth.
    \item Social Engineering: An assessment of the security of the people interacting with a system. For example, sending phishing mails.
    \item Code Reviews: Reviewing the source code of a system.
    \item Design Review: Reviewing the design of a system, by looking at architecture design descriptions and interviewing engineers and developers. This is possible to do before the system is actually build.
\end{itemize}

\subsection{Methodology Secura}
