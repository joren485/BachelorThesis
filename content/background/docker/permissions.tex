\subsection{Docker Socket}\label{background:docker-socket}
The Docker daemon runs a API\footnote{\url{https://docs.docker.com/engine/api/v1.40/}} that is used by clients to communicate with the Docker daemon. For example, when a user uses the Docker client command, it actually makes an HTTP request to the API.\@ By default, the API listens on a UNIX socket accessible through \lstinline{/var/run/docker.sock}, but it is also possible to make it listen for TCP connections.

\medskip

Which users are allowed to interact with the Docker daemon is defined by the permissions of the Docker socket. To use a Unix socket a user needs to have both read and write permissions.

\begin{lstlisting}[caption={Default Docker socket permissions.},captionpos=b,label={listing:docker-socket-permissions}]
(host)$ ls -l /var/run/docker.sock
srw-rw---- 1 root docker 0 Dec 20 13:16 /var/run/docker.sock
\end{lstlisting}

\autoref{listing:docker-socket-permissions} shows the default permissions of \lstinline{/var/run/docker.sock}. As we can see, the owner of \lstinline{/var/run/docker.sock} is \lstinline{root} and the group is \lstinline{docker}. Both the owner and the group have read and write access to the socket. This means that \lstinline{root} and every user in the \lstinline{docker} group is allowed to communicate with the Docker daemon and as such use Docker.
