\section{Docker}\label{background:docker}
\todo[inline]{Although, Docker is a namespace it can have a different OS}
\todo[inline]{Jos: containerization platform?}
The concept of containerization has been around for a long time\footnote{\url{https://docs.freebsd.org/44doc/papers/jail/jail-9.html}}, but it only gained traction as a serious way to package, distribute and run software in the last few years. This is mostly because of Docker.

\medskip

Docker was released in 2013 and it does not only offer a containerization platform, but also a way to distribute the containers. This allows developers and organizations to create packages that have no dependencies (besides Docker itself, of course). This allows for much faster development and deployment, because dependencies and installation of software are no longer a concern.

\medskip

Docker also makes it possible to run multiple versions of the same software on the same host, without creating a dependencies nightmare. For example, if someone wants to run a Wordpress 4 website and Wordpress 5 website, they only need to create two Wordpress containers. Wordpress 5 depends on much newer libraries that might not be compatible with Wordpress 4. Because the containers are isolated from one another, their conflicting dependencies are not a problem.
\subsection{Docker Concepts}
Docker is made up of a few concepts: daemons, images, containers and \lstinline{Dockerfile}s. These will be covered in the following sections.

\subsubsection{Docker Daemon}
The daemon is a service (a privileged program that runs in the background) that runs (as \lstinline{root}\footnote{An experimental rootless mode is being worked on.}\footnote{\url{https://github.com/docker/engine/blob/master/docs/rootless.md}}) on the host. It manages all things related to Docker on that machine. For example, if a user needs to restart a container, the Docker daemon is the process that restarts the container. It is good to note that, because everything related to Docker is handled by the daemon and Docker has access to all resources of the host (because it runs as \lstinline{root}), being able to use Docker is equivalent to having \lstinline{root} access to the host.\footnote{\url{https://docs.docker.com/engine/security/security/}}

\subsubsection{Images}
A Docker image is a packaged directory structure. It is a set of layers. Each layer adding, changing or removing specific files or directories in the image. The first layer (called the base image) is either an existing image or nothing (referred to as \lstinline{scratch}). Each layer on top of that is a change to the layer before.

\subsubsection{Containers}
A container is a running instance of a Docker image. If we run software packaged as a Docker image, we create a container based on that image. If we want to run two instances of the same Docker image, we can create two containers.

\subsubsection{\texorpdfstring{\lstinline{Dockerfiles}}{Dockerfiles}}
A \lstinline{Dockerfile} describes what layers a Docker image consists of. It describes the steps to build the image. Let's look at a basic example:

\begin{lstlisting}[caption={A basic \lstinline{Dockerfile}.},label={listing:dockerfile-simple},captionpos=b]
FROM alpine:latest
LABEL maintainer="Joren Vrancken"
CMD ["echo", "Hello World"]
\end{lstlisting}

These three instructions tell the Docker engine how to create a new Docker image.
The full instruction set can be found in the \lstinline{Dockerfile} reference.\footnote{\url{https://docs.docker.com/engine/reference/builder/}}

\begin{enumerate}
    \item The \lstinline{FROM} instruction tells the Docker engine what to base the new Docker image on. Instead of creating an image from scratch (a blank image), we use an already existing image as our basis (in this case an image based on Alpine Linux).

    \item The \lstinline{LABEL} instruction sets a key-value pair for the image. There can be multiple LABEL instructions. These key-value pairs get packaged and distributed with the image.

    \item The \lstinline{CMD} instruction sets the default command that should be run when the container is started and which arguments should be passed to it.
\end{enumerate}

We can use this to create a new image and container from that image.
\begin{lstlisting}[caption={Creating a Docker container from a \lstinline{Dockerfile}.},label={listing:create-container},captionpos=b]
(host)$ docker build -t thesis-hello-world .
(host)$ docker run --rm --name=thesis-hello-world-container thesis-hello-world
\end{lstlisting}

We first create a Docker image (called \lstinline{thesis-hello-world}) using the \lstinline{docker build} command and then create and start a new container (called \lstinline{thesis-hello-world-container}) from that image.

\subsection{Data Persistence}\label{subsection:data-persistence}
Without additional configuration, a Docker container does not have persistent storage. Its storage is maintained when the container is stopped, but not when the container is removed. It is possible to mount a directory on the host in a Docker container. This allows the container to access files on the host and save them to that mounted directory.

\begin{lstlisting}[caption={Bind mount example.},label={listing:docker-volume},captionpos=b]
(host)$ echo test > /tmp/test
(host)$ docker run -it --rm -v /tmp:/host-tmp ubuntu:latest bash
(cont)# cat /host-tmp/test
test
(cont)# cat /tmp/test
cat: /tmp/test: No such file or directory
\end{lstlisting}

In \autoref{listing:docker-volume} the host \lstinline{/tmp} directory is mounted into the container as \lstinline{/host-tmp}. We can see that a file that is created on the host is readable by the container. We also see that the container does have its own \lstinline{/tmp} directory, which has no relation to \lstinline{/host-tmp}.

\subsection{Networking}
\todo[inline]{Jos: bridge? networking resources internal to the container?}
When a Docker container is created, the Docker daemon creates a network sandbox for that container and (by default) connects it to an internal bridge network. This gives the container its own networking resources such as an IPv4 address\footnote{IPv6 support is not enabled by default.}, routes and DNS entries. All outgoing traffic is routed through a bridge interface (by default).

\medskip

Incoming traffic (that is not part of an existing connection) is possible by routing traffic for specific ports from the host to the container.
Specifying which ports on the host are routed to which ports on the container is done when a container is created. If we, for example, want to expose port \lstinline{80} to the Docker image created from \autoref{listing:dockerfile-simple} we can execute the following commands.

\begin{lstlisting}[caption={Creating a Docker container with exposed port.},label={listing:docker-port},captionpos=b]
(host)$ docker build -t thesis-hello-world .
(host)$ docker run --rm -p 8000:80 --name=thesis-hello-world-container thesis-hello-world
\end{lstlisting}

The first command creates a Docker image using the \lstinline{Dockerfile} and we then create (and start) a container from that image. We ``publish'' port \lstinline{8000} on the host to port \lstinline{80} of the container. This means that, while the container is running, all traffic from port \lstinline{8000} on the host is routed to port \lstinline{80} inside the container.

\subsection{Docker Internals}\label{subsubsection:internals}
A Docker container actually is a combination of multiple features within the Linux kernel.
Mainly \lstinline{namespaces}, \lstinline{cgroups} and \lstinline{OverlayFS}.

\medskip

\lstinline{namespaces} are a way to isolate resources from processes. For example, if we add a process to a process \lstinline{namespace}, it can only see the processes in that \lstinline{namespace}. This allows processes to be isolated from each other. Linux supports the following \lstinline{namespaces} types\footnote{See the \lstinline{man page} of \lstinline{namespaces}.}:
\begin{itemize}
    \item \lstinline{Cgroup}: To isolate processes from \lstinline{cgroup} hierarchies.
    \item \lstinline{IPC}: Isolates the inter-process communication. This, for example, isolates shared memory regions.
    \item \lstinline{Network}: Isolates the network stack (e.g. IP addresses, interfaces, routes and ports).
    \item \lstinline{Mount}: Isolates mount points. When creating a new \lstinline{mount namespace}, existing mount points are copied from the current \lstinline{namespace}. New mount points are not propagated.
    \item \lstinline{PID}: Isolates processes from seeing process ids in other \lstinline{namespaces}. Processes in different \lstinline{namespaces} can have the same \lstinline{PID}.
    \item \lstinline{User}: Isolates the users and groups.
    \item \lstinline{UTS}: Isolates the host and domain names.
\end{itemize}

When the Docker daemon creates a new container, it creates a new \lstinline{namespace} of each type for the process that runs in the container. That way the container cannot view any of the processes, network interfaces and mount points of the host (by default it can communicate with other Docker containers, because it is connected to the internal Docker network). To the container it seems that it is actually running an entirely separate operating system.

A \lstinline{mount namespace} is very similar to a \lstinline{chroot}. A big difference is that a \lstinline{chroot} has a parent directory.

\medskip

Control groups (\lstinline{cgroups}) are a way to limit resources (e.g.\ CPU and RAM usage) to (groups of) processes and to monitor those processes.

\medskip

\lstinline{OverlayFS} is a (union mount) file system that allows combining multiple directories and present them as if they are one. This is used to show the multiple layers in a Docker image as a single root directory.

\subsection{Permissions}
\todo[inline]{Misconfig: Docker Socket}
\todo[inline]{Misconfig: Docker permissions}
\todo[inline]{Pentesting host}

\subsection{Protection Mechanisms}
To significantly reduce the risks that (future) vulnerabilities pose to a system with Docker, there are multiple protections built into Docker and the Linux kernel itself. In this section, we will look at the best known and most important protections.

\medskip

It should be noted that because these protections add complexity and features, some vulnerabilities focus solely on bypassing one or more protection mechanisms. For example, CVE--2019--5021 (see \autoref{subsection:CVE-2019-5021}).

\subsubsection{Capabilities}\label{protection-mechanisms:subsection:capabilities}
To allow or disallow a process to use specific privileged functionality, the Linux kernel has a feature called ``capabilities''. A capability is a granular way of giving certain privileges to processes. A capability allows a process to perform a privileged action without giving the process full \lstinline{root} rights. For example, if we want a process to only be able to create its own network packets, we only give it the \lstinline{CAP_NET_RAW} capability.

\medskip

By default, every Docker container is started with only the necessary minimum capabilities. The default capabilities can be found in the Docker code\footnote{\url{https://github.com/moby/moby/blob/master/oci/caps/defaults.go}}. It is possible to add or remove capabilities at runtime using the \lstinline{--cap-add} and \lstinline{--cap-drop}\cite{More-Secure-Non-Root-Container} arguments.

\subsubsection{Secure Computing Mode}
Secure Computing Mode (\lstinline{seccomp}), like capabilities, is a built-in way to limit the privileged functionality that a process is allowed to use. Where capabilities limit functionality (like reading privileged files), Secure Computing Mode limits specific \lstinline{syscalls}. This allows granular security control. It does this by using whitelists (called profiles) of \lstinline{syscalls}.
To setup a strict, but still functional seccomp profile requires specific knowledge of which \lstinline{syscalls} are used by a program.

\medskip

The default seccomp profile that processes in Docker containers get, is available in the source code\footnote{\url{https://github.com/moby/moby/blob/master/profiles/seccomp/default.json}}. To pass a custom seccomp profile the \lstinline{--security-opt seccomp} can be used.

\subsubsection{Application Armor}
Application Armor (AppArmor) is a kernel module that allows application-specific limitations of files and system resources.

Docker adds a default AppArmor profile to every container. This is a profile generated at runtime based on a template\footnote{\url{https://github.com/moby/moby/blob/master/profiles/apparmor/template.go}}.

\medskip

It is also possible to generate custom apparmor profiles. For example, with a tool like \lstinline{bane}\footnote{\url{https://github.com/genuinetools/bane}}.

\subsubsection{Security-Enhanced Linux}
Security-Enhanced Linux (SELinux) is a set of changes to the Linux kernel that support system-wide access control for files and system resources. It is available by default on some Linux distributions (e.g.\ Red Hat Linux based distributions).

\medskip

Docker does not enable SELinux support by default, but it does provide a SELinux policy\footnote{\url{https://www.mankier.com/8/docker_selinux}}.

\subsubsection{Non-\texorpdfstring{\lstinline{root}}{root} Users in Containers}\label{subsection:non-root-user}
Besides the protection mechanisms on the host, there are also protection mechanisms in Docker images. The most important protection mechanism that Docker image creators can implement is not running processes inside a container as \lstinline{root}.

By default, processes in Docker containers are executed as \lstinline{root} (the \lstinline{root} user of that \lstinline{namespace}), because the process is isolated from the host system. However, as we will see there exist many ways to escape containers. Most of those ways require \lstinline{root} privileges (inside the container). This is why it is recommended to run processes in containers using non-\lstinline{root}. If the container gets compromised in any way, the attacker cannot escape because the attacker does not have \lstinline{root} permissions.

\medskip

This is covered by CIS Docker Benchmark guidelines 4.1 (Ensure that a user for the container has been created) and 5.23 (Ensure that docker exec commands are not used with the user=root option).

\subsection{\texorpdfstring{\lstinline{docker-compose}}{docker-compose}}
\todo[inline]{Remove this section?}
\lstinline{docker-compose} is a wrapper program (a program that simplifies usage of another program) around Docker that can be used to specify Docker container configurations in files (called \lstinline{docker-compose.yaml}). These files remove the need to execute Docker commands with the correct arguments in the correct order. We only have to specify the necessary arguments once in the \lstinline{docker-compose.yaml} file.

\medskip

\autoref{listing:docker-compose-file} is an example of an \lstinline{docker-compose.yaml} file similar to configuration that I have used in a production environment. Docker containers in production environments need to have a lot of runtime configuration (e.g.\ environment variables, exposed ports and dependencies on other containers). Specifying everything in a single file simplifies and stores the runtime configuration process.
\begin{lstlisting}[caption={Example \lstinline{docker-compose.yaml}.},label={listing:docker-compose-file},captionpos=b]
---
version: "3"

services:
  postgres:
    image: "postgres:10.5"
    restart: "always"
    environment:
      PGDATA: "/var/lib/postgresql/data/pgdata"
    volumes:
      - "/dir/data/:/var/lib/postgresql/data/"

  nextcloud:
    image: "nextcloud:17-fpm"
    restart: "always"
    ports:
      - "127.0.0.1:9000:9000"
    depends_on:
      - "postgres"
    environment:
      POSTGRES_DB: "database"
      POSTGRES_USER: "user"
      POSTGRES_PASSWORD: "password"
      POSTGRES_HOST: "postgres"
    volumes:
      - "/dir/www/:/var/www/html/"
\end{lstlisting}

Similar functionality is also built into the Docker Engine, called Docker Stack. It also uses \lstinline{docker-compose.yaml}. Some features that are supported by \lstinline{docker-compose} are not supported by Docker Stack and vice versa.

\subsection{Registries}
Docker images are distributable through registries. A registry is a server (that anybody can host), that stores Docker images. When a client does not have a Docker image that it needs, it can contact a registry to download that image. Note that because registries are an easy way to distribute Docker images, they are an interesting attack vector.

\medskip

The most popular (and default) registry is Docker Hub, which is run by the Docker company itself. Anybody can create a Docker Hub account and start creating and publishing images that anybody can download.


