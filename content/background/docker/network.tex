\subsection{Networking}
\todo[inline]{Jos: bridge? networking resources internal to the container?}
When a Docker container is created, the Docker daemon creates a network sandbox for that container and (by default) connects it to an internal bridge network. This gives the container its own networking resources such as an IPv4 address\footnote{IPv6 support is not enabled by default.}, routes and DNS entries. All outgoing traffic is routed through a bridge interface (by default).

\medskip

Incoming traffic (that is not part of an existing connection) is possible by routing traffic for specific ports from the host to the container.
Specifying which ports on the host are routed to which ports on the container is done when a container is created. If we, for example, want to expose port \lstinline{80} to the Docker image created from \autoref{listing:dockerfile-simple} we can execute the following commands.

\begin{lstlisting}[caption={Creating a Docker container with exposed port.},label={listing:docker-port},captionpos=b]
(host)$ docker build -t thesis-hello-world .
(host)$ docker run --rm -p 8000:80 --name=thesis-hello-world-container thesis-hello-world
\end{lstlisting}

The first command creates a Docker image using the \lstinline{Dockerfile} and we then create (and start) a container from that image. We ``publish'' port \lstinline{8000} on the host to port \lstinline{80} of the container. This means that, while the container is running, all traffic from port \lstinline{8000} on the host is routed to port \lstinline{80} inside the container.
