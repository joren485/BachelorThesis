\subsection{Networking}
When a Docker container is created, the Docker daemon creates a network sandbox for that container and (by default) connects it to an internal network. This gives the container networking resources (e.g.\ an IPv4 address,\footnote{IPv6 support is not enabled by default.} routes and DNS entries) that are separate from the host. All incoming and outgoing traffic to the container is routed through an interface (by default) which is bridged\footnote{A bridge interface is an interface that connects the network connection of one interface to another.} to an interface on the host.

\medskip

Incoming traffic (that is not part of an existing connection) is possible by routing traffic for specific ports from the host to the container.
Specifying which ports on the host are routed to which ports on the container is done when a container is created. If we, for example, want to expose port \lstinline{80} to the Docker image created from \autoref{listing:dockerfile-simple} we can execute the following commands.

\begin{lstlisting}[caption={Creating a Docker container with exposed port.},label={listing:docker-port},captionpos=b]
(host)$ docker build -t thesis-hello-world .
(host)$ docker run --rm -p 8000:80 --name=thesis-hello-world-container thesis-hello-world
\end{lstlisting}

The first command creates a Docker image using the \lstinline{Dockerfile} and we then create (and start) a container from that image. We ``publish'' port \lstinline{8000} on the host to port \lstinline{80} of the container. This means that, while the container is running, all traffic from port \lstinline{8000} on the host is routed to port \lstinline{80} inside the container.

\medskip

By default, all Docker containers are added to the same internal network. This means that (by default) all Docker containers can reach each other over the network. This differs from the isolation Docker uses for other \lstinline{namespaces}. In the other \lstinline{namespaces}, Docker isolates containers from the host and from other containers. This difference in design can lead to dangerous misconfigurations, because developers may believe that Docker containers are completely isolated from each other (including the network).
