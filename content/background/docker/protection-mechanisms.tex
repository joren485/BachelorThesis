\subsection{Protection Mechanisms}
To significantly reduce the risks that (future) vulnerabilities pose to a system with Docker, there are multiple protections built into Docker and the Linux kernel itself. In this section, we will look at the best known and most important protections.

\medskip

It should be noted that because these protections add complexity and features, some vulnerabilities focus solely on bypassing one or more protection mechanisms. For example, CVE--2019--5021 (see \autoref{subsection:CVE-2019-5021}).

\subsubsection{Capabilities}\label{protection-mechanisms:subsection:capabilities}
To allow or disallow a process to use specific privileged functionality, the Linux kernel has a feature called ``capabilities''. A capability is a granular way of giving certain privileges to processes. A capability allows a process to perform a privileged action without giving the process full \lstinline{root} rights. For example, if we want a process to only be able to create its own network packets, we only give it the \lstinline{CAP_NET_RAW} capability.

\medskip

By default, every Docker container is started with only the necessary minimum capabilities. The default capabilities can be found in the Docker code\footnote{\url{https://github.com/moby/moby/blob/master/oci/caps/defaults.go}}. It is possible to add or remove capabilities at runtime using the \lstinline{--cap-add} and \lstinline{--cap-drop}\cite{More-Secure-Non-Root-Container} arguments.

\subsubsection{Secure Computing Mode}
Secure Computing Mode (\lstinline{seccomp}), like capabilities, is a built-in way to limit the privileged functionality that a process is allowed to use. Where capabilities limit functionality (like reading privileged files), Secure Computing Mode limits specific \lstinline{syscalls}. This allows granular security control. It does this by using whitelists (called profiles) of \lstinline{syscalls}.
To setup a strict, but still functional seccomp profile requires specific knowledge of which \lstinline{syscalls} are used by a program.

\medskip

The default seccomp profile that processes in Docker containers get, is available in the source code\footnote{\url{https://github.com/moby/moby/blob/master/profiles/seccomp/default.json}}. To pass a custom seccomp profile the \lstinline{--security-opt seccomp} can be used.

\subsubsection{Application Armor}
Application Armor (AppArmor) is a kernel module that allows application-specific limitations of files and system resources.

Docker adds a default AppArmor profile to every container. This is a profile generated at runtime based on a template\footnote{\url{https://github.com/moby/moby/blob/master/profiles/apparmor/template.go}}.

\medskip

It is also possible to generate custom apparmor profiles. For example, with a tool like \lstinline{bane}\footnote{\url{https://github.com/genuinetools/bane}}.

\subsubsection{Security-Enhanced Linux}
Security-Enhanced Linux (SELinux) is a set of changes to the Linux kernel that support system-wide access control for files and system resources. It is available by default on some Linux distributions (e.g.\ Red Hat Linux based distributions).

\medskip

Docker does not enable SELinux support by default, but it does provide a SELinux policy\footnote{\url{https://www.mankier.com/8/docker_selinux}}.

\subsubsection{Non-\texorpdfstring{\lstinline{root}}{root} Users in Containers}\label{subsection:non-root-user}
Besides the protection mechanisms on the host, there are also protection mechanisms in Docker images. The most important protection mechanism that Docker image creators can implement is not running processes inside a container as \lstinline{root}.

By default, processes in Docker containers are executed as \lstinline{root} (the \lstinline{root} user of that \lstinline{namespace}), because the process is isolated from the host system. However, as we will see there exist many ways to escape containers. Most of those ways require \lstinline{root} privileges (inside the container). This is why it is recommended to run processes in containers using non-\lstinline{root}. If the container gets compromised in any way, the attacker cannot escape because the attacker does not have \lstinline{root} permissions.

\medskip

This is covered by CIS Docker Benchmark guidelines 4.1 (Ensure that a user for the container has been created) and 5.23 (Ensure that docker exec commands are not used with the user=root option).
