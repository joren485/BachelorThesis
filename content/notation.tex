\chapter{Notation}\label{notation}
Throughout this thesis we will look at examples using shell commands. The following conventions are used to represent the different contexts in which the commands are executed.

\begin{itemize}
    \item If a command is executed directly on a host system, it is prefixed by ``\lstinline{(host)}''.
    \item If a command is executed inside a container, it is prefixed by ``\lstinline{(cont)}''.
    \item If a command is executed by an unprivileged user, it is prefixed by ``\lstinline{$}''.
    \item If a command is executed by a privileged user (i.e. \lstinline{root}), it is prefixed by ``\lstinline{#}''.
    \item Long and irrelevant output of commands is replaced by ``\ldots''.
\end{itemize}

In this example, an unprivileged user executes the command \lstinline{echo Hello, World!} on the host system.
\begin{lstlisting}[caption={Shell command notation example 1}, captionpos=b]
(host)$ echo Hello, World!
Hello, World!
\end{lstlisting}

\hfill

In this example, the \lstinline{root} user executes two commands to get system information. The contents of \lstinline{/proc/cpuinfo} are not shown.
\begin{lstlisting}[caption={Shell command notation example 2}, captionpos=b]
(cont)# uname -r
5.3.8-arch1-1
(cont)# cat /proc/cpuinfo
...
\end{lstlisting}
