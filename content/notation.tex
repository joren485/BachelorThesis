\chapter{Notation}\label{chapter:notation}
Throughout this thesis we will look at examples using Unix shell commands. The following conventions are used to represent the different contexts in which the commands are executed.

\begin{itemize}
    \item If a command is executed directly on a host system, it is prefixed by ``\lstinline{(host)}''.
    \item If a command is executed inside a container, it is prefixed by ``\lstinline{(cont)}''.
    \item If a command is executed by an unprivileged user, it is prefixed by ``\lstinline{$}''.
    \item If a command is executed by a privileged user (i.e. \lstinline{root}), it is prefixed by ``\lstinline{#}''.
    \item Long or irrelevant output of commands is replaced by ``\ldots''.
\end{itemize}

In \autoref{listing:notation:example1}, an unprivileged user executes the command ``\lstinline{echo Hello, World!}'' on a host system.
\begin{lstlisting}[caption={Shell command notation example 1.}, captionpos=b, label={listing:notation:example1}]
(host)$ echo Hello, World!
Hello, World!
\end{lstlisting}

\medskip

In \autoref{listing:notation:example2}, the \lstinline{root} user executes two commands to get system information. The content of \lstinline{/proc/cpuinfo} is omitted.
\begin{lstlisting}[caption={Shell command notation example 2.}, captionpos=b, label={listing:notation:example2}]
(cont)# uname -r
5.3.8-arch1-1
(cont)# cat /proc/cpuinfo
...
\end{lstlisting}

\medskip

We will look at many examples using shell commands. Although I prefer using non-abbreviated arguments and quoted values (see \autoref{listing:notation:example3}) to make it more clear what a command does, throughout this thesis we will use abbreviated non-quoted arguments (see \autoref{listing:notation:example4}) where possible to make the commands smaller and more readable.

\begin{lstlisting}[caption={Abbreviated non-quoted command example.}, captionpos=b, label={listing:notation:example3}]
(host)$ docker run --tty="true" --interactive="true" --rm --volume="/:/host/" ubuntu:latest
...
\end{lstlisting}

\begin{lstlisting}[caption={Non-abbreviated quoted command example.}, captionpos=b, label={listing:notation:example4}]
(host)$ docker run -it --rm -v /:/host/ ubuntu:latest /bin/bash}
...
\end{lstlisting}

\subsection*{Common Vulnerabilities and Exposures}
The Common Vulnerabilities and Exposures (CVE for short) system is a list of publicly known security related bugs. Every bug that is found gets a CVE identifier, which looks like CVE--2019--0000. The first number represents the year in which the vulnerability is found. The second number is an arbitrary number that is at least four digits long. The system is maintained by the Mitre Corporation. Organizations that are allowed to give out new CVE identifiers are called CVE Numbering Authorities (CNA for short). It is possible to read and search the full list on Mitre's website\footnote{\url{https://cve.mitre.org/}}, the United State's National Vulnerability Database\footnote{\url{https://nvd.nist.gov/}} and other websites like CVEDetails\footnote{\url{https://www.cvedetails.com/}}.

The severity (impact and likelihood of exploitation) of a CVE is determined by the Common Vulnerability Scoring System (CVSS for short) score. The CVSS scores of every CVE can be found in the National Vulnerability Database\footnote{\url{https://nvd.nist.gov/}} which is maintained by National Institute of Standards and Technology.
