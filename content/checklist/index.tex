\chapter{Docker Penetration Test Checklist}\label{chapter:checklist}
\todo[inline]{Rename checklist to something like methodology eveywhere}
\todo[inline]{Dave:you may add the text that Secura explisitly asked for this list}
\todo[inline]{Jos: What is your contribution? What is new?}
\todo[inline]{Jos: How long will it remain useful/ up to date?}
\todo[inline]{Jos: To what extent can this checklist be automated?}
\todo[inline]{Jos: What are problems with automation?}
\todo[inline]{autoref section:tools capital S}
In \autoref{chapter:vulnerabilities} and \autoref{chapter:pentesting} we looked at common vulnerabilities and how to identify them. In this chapter we will summarize those into a checklist consisting of steps.

This list is kept intentionally short and uses only Unix shell commands that can be run manually, to make it easy and quick to use. \autoref{section:tools} describes tools help automating certain enumeration or exploitation of vulnerabilities. These are however not necessary to use this checklist.

\medskip

The first steps (\autoref{section:checklist-detection}) are meant to detect whether we are running inside a container. If we know we are inside a container, we can look for vulnerabilities inside the container (see \autoref{section:checklist-vulnerabilities-container}). If we know we are not running inside a container, we can look for vulnerabilities on the host (see \autoref{section:checklist-vulnerabilities-host}).

\section{Are We Running in a Container?}\label{section:checklist-detection}
If the answer to any of the following questions is yes, we are most likely running inside a container. For detailed information, see \autoref{subsection:detection}.

\medskip

If we are running inside a container, see \autoref{section:checklist-vulnerabilities-container}. If not, please see \autoref{section:checklist-vulnerabilities-host}.

\begin{itemize}
    \item \textbf{Does \lstinline{/.dockerenv} exists?} (see \autoref{subsubsection:dockerenv})\\
    Execute ``\lstinline{ls /.dockerenv}'' to see if \lstinline{/.dockerenv} exists.

    \item \textbf{Does \lstinline{/proc/1/cgroup} contain ``\lstinline{/docker/}''?} (see \autoref{subsubsection:detection:cgroup})\\
    Execute ``\lstinline{grep '/docker/' /proc/1/cgroup}'' to find all lines in \lstinline{/proc/1/cgroup} containing ``\lstinline{/docker/}''.

    \item \textbf{Are there fewer than 5 processes?} (see \autoref{subsubsection:processes})\\
    Execute ``\lstinline{ps aux}'' to view all processes.

    \item \textbf{Is the process with process id 1 a common initial process?} (see \autoref{subsubsection:processes})\\
        Execute ``\lstinline{ps -p1}'' to view the process with process id 1 and check if it is a common initial process (.e.g.\ \lstinline{systemd} or \lstinline{init}).

    \item \textbf{Are common libraries and binaries not present on the system?} (see \autoref{subsubsection:binaries})\\
    We can use the \lstinline{which} command to find available binaries. For example, ``\lstinline{which sudo}'' will tell us if the \lstinline{sudo} binary is available.
\end{itemize}

\section{Finding Vulnerabilities in Containers}\label{section:checklist-vulnerabilities-container}
The following questions and steps are meant to identify interesting parts and weak spots inside containers. For detailed information, see \autoref{subsection:testing-container}.

\begin{itemize}
    \item \textbf{What is the current user?} (see \autoref{subsubsection:user-enumeration})\\
    Execute ``\lstinline{id}'' to see what the current user is and what groups it is in.

    \item \textbf{Which users are available on the system?} (see \autoref{subsubsection:user-enumeration})\\
    Read \lstinline{/etc/passwd} to see what users are available.

    \item \textbf{What is the operating system of the container?} (see \autoref{subsubsection:idenitfy-container-os})\\
    Read \lstinline{/etc/os-release} to get information about the operating system.

    \item \textbf{Which processes are running?} (see \autoref{subsubsection:idenitfy-container-os})\\
    Execute ``\lstinline{ps aux}'' to view all processes.

    \item \textbf{What is the host operating system?} (see \autoref{subsubsection:identify-host-os})\\
    Execute ``\lstinline{uname -a}'' to get information about the kernel and the underlying host operating system.

    \item \textbf{Which capabilities do the processes in the container have?} (see \autoref{subsubsection:container:capabilities})\\
    Get the current capabilities value by running ``\lstinline{grep CapEff /proc/self/status}'' and decode it with ``\lstinline{capsh --decode=value}\footnote{``value'' should look like \lstinline{00000000a80425fb}}''. \lstinline{capsh} can be run on a different system.

    \item \textbf{Is the container running in privileged mode?}
    \\(see \autoref{subsubsection:container:privileged})\\
    If the \lstinline{CapEff} value of the previous step equals \lstinline{0000003fffffffff}, the container is running in privileged mode and we are able to escape it (see \autoref{subsection:privileged}).

    \item \textbf{What volumes are mounted?} (see \autoref{subsubsection:volumes})\\
    Read \lstinline{/proc/mounts} to see all mounts including the volumes.

    \item \textbf{Is there sensitive information stored in environment variables?} (see \autoref{pentest:container:env-vars})\\
    The ``\lstinline{env}'' command will list all environment variables. We should check these for sensitive information.

    \item \textbf{Is the Docker Socket mounted inside the container?} (see \autoref{subsubsection:searching-socket})\\
        Check \lstinline{/proc/mounts} to see if \lstinline{docker.sock} (or some similar named socket) is mounted inside the container. \lstinline{/run/docker.sock} is a common mount point. If we find it, we can escape the container and interact with the Docker daemon on the host.

    \item \textbf{What hosts are reachable on the network?} (see \autoref{subsubsection:network-scan})\\
    If possible, use \lstinline{nmap} to scan the local network for reachable hosts. The IPv4 address of the container can be found in \lstinline{/etc/hosts}.
\end{itemize}

\section{Finding Vulnerabilities on the Host}\label{section:checklist-vulnerabilities-host}
The following questions and steps are meant to identify interesting parts and weak spots on hosts running Docker. For detailed information, see \autoref{subsection:testing-host}.

\begin{itemize}
    \item \textbf{What is the version of Docker?} (see \autoref{subsubsection:version})\\
    Run ``\lstinline{docker --version}'' to find the version of Docker. We will need to check if there are any known software related bugs (\autoref{section:bugs}) in this version of Docker. We can find relevant CVEs in the National Vulnerability Database\footnote{\url{https://nvd.nist.gov/}}.

    \item \textbf{Run Docker Bench for Security} (see \autoref{tools:bench})\\
        Run Docker Bench for Security\footnote{\url{https://github.com/docker/docker-bench-security}} to quickly see what CIS Docker Benchmark guidelines are not being followed.

    \item \textbf{Which users are able to interact with the Docker socket?} (see \autoref{subsubsection:docker-permissions-host})\\
    Execute ``\lstinline{ls -l /var/run/docker.sock}'' to see the owner and group of \lstinline{/var/run/docker.sock} and which users have read and write access to it. The owner and every user in the group is allowed to use Docker.

    \item \textbf{Who is in the \lstinline{docker} group?} (see \autoref{subsubsection:docker-permissions-host})\\
    Check which users are in the group identified in the previous step (by default \lstinline{docker}) by executing ``\lstinline{grep docker /etc/group}''.

    \item \textbf{Is the \lstinline{setuid} bit set on the Docker client binary?} (see \autoref{subsubsection:docker-permissions-host})\\
    Check the permissions (including whether the \lstinline{setuid} bit is set) of the Docker binary by executing ``\lstinline{ls -l $(which docker)}''.

    \item \textbf{What images are available?} (see \autoref{subsubsection:images-containers})\\
    List the available images by running ``\lstinline{docker images -a}''.

    \item \textbf{What containers are available?} (see \autoref{subsubsection:images-containers})\\
    List all containers (running and stopped) by running ``\lstinline{docker ps -a}''.

    \item \textbf{How is the Docker daemon started?} (see \autoref{subsubsection:configuration})\\
    Check configuration files (e.g. \lstinline{/usr/lib/systemd/system/docker.service} and \lstinline{/etc/docker/daemon.json}) for information on how the Docker daemon is started.

    \item \textbf{Do any \lstinline{docker-compose.yaml} files exist?} (see \autoref{subsection:config-files} and \autoref{subsubsection:configuration})\\
    Find all \lstinline{docker-compose.yaml} files using ``\lstinline{find / -name "docker-compose.*"}''.

    \item \textbf{Do any \lstinline{.docker/config.json} files exist?} (see \autoref{subsection:config-files} and \autoref{subsubsection:configuration})\\
    Read the \lstinline{config.json} files in all directories by running ``\lstinline{cat /home/*/.docker/config.json}''.

    \item \textbf{Are the \lstinline{iptables} rules set for both the host and the containers?} (see \autoref{subsubsection:iptables-host})\\
    List the \lstinline{iptables} by running ``\lstinline{iptables -vnL}'' and ``\lstinline{iptables -t filter -vnL}''.
\end{itemize}
