\begin{abstract}
Penetration testers encounter many different systems during assessments. Penetration testers encounter systems using Docker more and more often, because of the popularity of Docker in recent years. This research discusses Docker from a security perspective and looks at how a penetration tester should assess the security of systems that use Docker.

\medskip

We introduce two attacker models: container escapes and Docker daemon attacks. These attacker models are generalizations of attacks from a certain perspective. We discuss container escapes, an attacker model where the attacker takes the perspective of a process running inside a container. We also discuss Docker daemon attacks, an attacker model where the attacker takes the perspective of a process running on a host with Docker installed.

\medskip

We look at known vulnerabilities in Docker. Specifically, we look at misconfigurations and security related software bugs. We provide practical examples of how to exploit the misconfigurations the and what the resulting impact could be. We find that misconfigurations are more interesting than the software bugs, because software bugs are far easier to fix for a user.

We map these vulnerabilities to relevant CIS Docker Benchmark (a best practices guide about the use of Docker) guidelines. We see that not all misconfigurations are covered by the CIS Docker Benchmark.

\medskip

Additionally, we describe how to identify the relevant attacker model during a penetration test. After that we describe how to manually perform reconnaissance and identify vulnerabilities on systems that use Docker. We do this for both attacker models.

We take a look at tools that might automate the identification and exploitation of vulnerabilities. We, however, find that no tool fully automates and replaces manual assessments.

\medskip

We conclude by presenting a checklist that summarizes the research as questions that a penetration tester should ask about a target system using Docker during an assessment. For each question, a simple way to answer the question and a reference to the relevant section in this thesis is given. This checklist helps penetration testers test the security of systems that use Docker.
\end{abstract}
