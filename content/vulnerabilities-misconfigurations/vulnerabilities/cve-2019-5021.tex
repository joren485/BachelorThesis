\subsection{CVE--2019--5021}\label{subsection:CVE-2019-5021}
One of the most used base images (the Docker image for Alpine Linux) had a problem where the password of the \lstinline{root} user is left empty. In Linux it is possible to disable a password (what should have happened) and to leave it blank. A disabled password cannot be used, but a blank password equals an empty input. This allows non-root users to gain root rights by supplying a blank password.

\hfill

It is still possible to use the vulnerable images.
\begin{lstlisting}
(host)$ docker run -it --rm alpine:3.5 cat /etc/shadow
root:::0:::::
(host)$ docker run -it --rm alpine:3.5 sh
(cont)# apk add --no-cache linux-pam shadow
...
(cont)# adduser test
...
(cont)# su test
Password:
(cont)$ su root
(cont)#
\end{lstlisting}

\subsubsection*{Side note about the CVSS score of CVE--2019--5021}

This vulnerability has a CVSS score of 9.8 (and a 10 in CVSS 2)\footnote{\url{https://nvd.nist.gov/vuln/detail/CVE-2019-5021}}. The CVSS scores are out of 10, meaning this is seen as an extremely high-risk vulnerability. But in actuality, this vulnerability is only risky in very specific cases. ``Empty \lstinline{root} password'' sounds very dangerous, but it really is not that dangerous in an isolated container that runs root by default. Only in the very specific case that a process in a container runs as a non-root user and their is some vulnerability or misconfiguration that allows \lstinline{root} to escape the container and an attacker can get control of the process in the container is this dangerous. In other words, this vulnerability is actually not likely to be used in the wild and most likely needs to be combined with another vulnerability or misconfiguration to be able to do damage.
