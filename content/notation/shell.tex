\section{Unix Shell Commands}
The following conventions are used to represent the different contexts in which commands are executed.

\begin{itemize}
    \item If a command is executed directly on a host system, it is prefixed by ``\lstinline{(host)}''.
    \item If a command is executed inside a container, it is prefixed by ``\lstinline{(cont)}''.
    \item If a command is executed by an unprivileged user, it is prefixed by ``\lstinline{$}''.
    \item If a command is executed by a privileged user (i.e. \lstinline{root}), it is prefixed by ``\lstinline{#}''.
    \item Long or irrelevant output of commands is replaced by ``\ldots''.
    \item In order to improve legibility, commands shown use abbreviated command arguments (where possible) and quoted argument values.
\end{itemize}

\medskip

In \autoref{listing:notation:example1}, an unprivileged user executes a command on a host system.
\begin{lstlisting}[caption={Shell command notation example 1.}, captionpos=b, label={listing:notation:example1}]
(host)$ echo "Hello, World!"
Hello, World!
\end{lstlisting}

\medskip

In \autoref{listing:notation:example2}, the \lstinline{root} user executes two commands to get system information. The content of \lstinline{/proc/cpuinfo} is omitted.
\begin{lstlisting}[caption={Shell command notation example 2.}, captionpos=b, label={listing:notation:example2}]
(cont)# uname -r
5.3.8-arch1-1
(cont)# cat /proc/cpuinfo
...
\end{lstlisting}
