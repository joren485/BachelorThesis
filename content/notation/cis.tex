\section{The CIS Docker Benchmark}
The Center for Internet Security (CIS) is a not-for-profit organization that provides best practice solutions for digital security. For example, they provide security hardened virtual machine images\footnote{\url{https://www.cisecurity.org/cis-hardened-images/}} that are configured for optimal security.

\medskip

The CIS Benchmarks\footnote{\url{https://cisecurity.org/cis-benchmarks/}} are guidelines and best practices on security on many different types of software. These guidelines are freely available for anyone and can be found on their site. Companies (e.g. Secura) use the CIS Benchmarks as a baseline to assess the security and configuration of systems that use Docker.

\medskip

They also provide guidelines on Docker.\footnote{Only the community edition (Docker CE). It does not cover the enterprise edition (Docker EE).} The latest version (1.2.0, published 29 July 2019) contains 115 guidelines. These are sorted by topic (e.g. Docker daemon and configuration files). In \autoref{appendix:CIS-Benchmark-Example} you will find an example guideline from the latest CIS Docker Benchmark.

In \autoref{chapter:vulnerabilities} we will look at different Docker related vulnerabilities. We will map those to guidelines in the CIS Docker Benchmark. We will also look at a tool that automatically checks if a host follows all guidelines in \autoref{tools:bench}.

In \autoref{futurework:CIS} we look at possible improvements to the CIS Docker Benchmark.
