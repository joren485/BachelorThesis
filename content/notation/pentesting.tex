\section{Penetration Testing}
\todo[inline]{Jos: add an architecture picture}
Penetration testing (pentesting for short) is an simulated attack to test the security of systems and applications. The goal of a penetration test is to find the weak points in a system in order able to fix and secure them.

Companies, such as Secura, perform penetration tests for clients. The result of such a penetration test is a report detailing the weaknesses of the client's systems and applications. This gives the client insight into how to secure their systems and the weaknesses an attacker might target.

A typical penetration test is performed in phases (called a \emph{kill chain}):
\begin{enumerate}
    \item Reconnaissance: Gather data about the target system or application. These can be gathered actively (i.e.\ with interaction with the target) or passively (i.e.\ without interaction with the target).
    \item Exploitation: The gathered data is used to identify weak spots and vulnerabilities. These are attacked and exploited to gain (unprivileged) access.
    \item Post-exploitation: After successful exploitation and gaining a foothold, a persistent foothold is established.
    \item Exfiltration: Once a persistent foothold has been established, sensitive data from the system is retrieved.
    \item Cleanup: Once the attack has been successful, all traces of the attack should be removed.
\end{enumerate}

\medskip

There are many types of assessments. Most tests differ in what information about the system the assessor gets from the system administrator or owner before the assessment starts or what kind of systems or applications are being tested. Below are some common assessments that companies, like Secura, perform:
\begin{itemize}
    \item Black Box Application / Infrastructure Test: The assessor does not get any information about the system that are in the assessment scope.

    \item Grey Box Application / Infrastructure Test: The assessor gets some information (e.g.\ credentials) about the systems in the assessment scope.

    \item Crystal Box Application / Infrastructure Test:  The assessor gets all available information about the system and its internal workings. Additionally, architects of the system may be interviewed. Crystal Box assessments are sometimes called a White Box assessment.

    \item Design Review: An assessment where the architecture, documentation and configuration of all systems within an environment are reviewed. No actual tests are performed during a design review.
    \item Internal Penetration Test: An assessment of the internal network of a client. Most of the time, the assessment has a clear goal (e.g.\ finding certain sensitive information).

    \item Red Teaming: An assessment that is similar to a real word targeted attack. This type of assessment relies heavily on stealth and includes all techniques that might be used by malicious actors to obtain sensitive information without being detected.

    \item Social Engineering: An assessment of the security of the people interacting with a system (e.g.\ employees of a company). For example, sending phishing mails or trying to get physical access to a building by impersonating an employee.

    \item Code Reviews: Reviewing the source code of an application.
\end{itemize}
