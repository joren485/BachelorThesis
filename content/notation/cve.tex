\section{Common Vulnerabilities and Exposures}
The Common Vulnerabilities and Exposures (CVE) system is a list of publicly known security related bugs. 

Every vulnerability that is found is given a CVE identifier, which looks like CVE--2019--0000. The first number represents the year in which the vulnerability is found. The second number is an arbitrary number of at least four digits. In practice the arbitrary number is implemented as a counter (e.g.\ the first CVE of a year gets CVE--YYYY--0001 and second gets CVE--YYYY--0002).

The system is maintained by the MITRE Corporation.\footnote{\url{https://cve.mitre.org/}} Organizations that are allowed to give out new CVE identifiers are called CVE Numbering Authorities (CNA for short). It is possible to read and search the full list on MITRE's website, the United State's National Vulnerability Database\footnote{\url{https://nvd.nist.gov/}} (NVD) and other websites like CVEDetails.\footnote{\url{https://www.cvedetails.com/}}

\medskip

The severity (impact and likelihood of exploitation) of a CVE is determined by the Common Vulnerability Scoring System (CVSS for short) score. The CVSS scores of every CVE can be found in the National Vulnerability Database.

\medskip

In \autoref{section:bugs} we will look at different security related bugs.
