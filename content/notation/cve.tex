\section{Common Vulnerabilities and Exposures}
The Common Vulnerabilities and Exposures (CVE) system is a list of publicly known security related bugs. Every bug that is found gets a CVE identifier, which looks like CVE--2019--0000. The first number represents the year in which the vulnerability is found. The second number is an arbitrary number that is at least four digits long. The system is maintained by the Mitre Corporation. Organizations that are allowed to give out new CVE identifiers are called CVE Numbering Authorities (CNA for short). It is possible to read and search the full list on Mitre's website\footnote{\url{https://cve.mitre.org/}}, the United State's National Vulnerability Database\footnote{\url{https://nvd.nist.gov/}} (NVD) and other websites like CVEDetails\footnote{\url{https://www.cvedetails.com/}}.

The severity (impact and likelihood of exploitation) of a CVE is determined by the Common Vulnerability Scoring System (CVSS for short) score. The CVSS scores of every CVE can be found in the National Vulnerability Database\footnote{\url{https://nvd.nist.gov/}} which is maintained by National Institute of Standards and Technology.

\medskip

In \autoref{section:bugs} we will look at different security related bugs.
