\chapter{Future Work}
\section{Kubernetes}
\section{Condense CIS Benchmarks}

The CIS Benchmarks for Docker contains 115 guidelines with their respective documentation.
This makes it a 250+ page document. This is not practical for developers and engineers (the intended audience). It would be much more useful to have a smaller, better sorted list that only contains common mistakes and pitfalls to watch out for.

\hfill

The CIS Benchmarks do indicate the importance of each guideline.
With Level 1 indicating that the guideline is a must-have and Level 2 indicating that the guideline is only necessary if extra security is needed. However, only twenty-one guidelines are actually considered Level 2. All the other guidelines are considered Level 1. This still leaves the reader with a lot of guidelines that are considered must-have.

\hfill

It would be a good idea to split the document into multiple sections. The guidelines can be divided by their importance and usefulness. For example, a three section division can be made.

\hfill

The first section would describe obvious and basic guidelines that everyone should follow (and probably already does). This is an example of guidelines that would be part of this section:
\begin{itemize}
    \item 1.1.2: Ensure that the version of Docker is up to date
    \item 2.4: Ensure insecure registries are not used
    \item 3.1: Ensure that the docker.service file ownership is set to root:root
    \item 4.2: Ensure that containers use only trusted base images
    \item 4.3: Ensure that unnecessary packages are not installed in the container
\end{itemize}

\hfill

The second section would contain guidelines that are common mistakes and pitfalls. These guidelines would be the most useful to the average developer. For example:
\begin{itemize}
    \item 4.4 Ensure images are scanned and rebuilt to include security patches
    \item 4.7 Ensure update instructions are not use alone in the Dockerfile
    \item 4.9 Ensure that COPY is used instead of ADD in Dockerfiles
    \item 4.10 Ensure secrets are not stored in Dockerfiles
    \item 5.6 Ensure \verb|sshd| is not run within containers
\end{itemize}

\hfill

The last section would describe guidelines that are intended for systems that need extra hardening. For example:
\begin{itemize}
    \item 1.2.4 Ensure auditing is configured for Docker files and directories
    \item 4.1 Ensure that a user for the container has been created
    \item 5.4 Ensure that privileged containers are not used
    \item 5.26 Ensure that container health is checked at runtime
    \item 5.29 Ensure that Docker's default bridge ``\verb|docker0|'' is not used
\end{itemize}
