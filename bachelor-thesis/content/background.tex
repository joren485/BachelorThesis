\chapter{Background}

\section{Containerization Software}
\subsection{Docker}
\todo[inline]{Talk about: Docker is not a security model}
\subsection{docker-compose}
\todo[inline]{Secrets in docker-compose can be easily found, even without docker group permissions}
\subsection{Docker Hub}
\todo[inline]{Everybody can create an image}
\todo[inline]{Official images not always standard images}

\section{Penetration Testing}
\subsection{Methodology Secura}

\section{CIS Benchmarks}
The Center for Internet Security (or CIS for short) is a non-profit organization that provides best practice solutions for digital security. For example, they provide security hardened virtual machine images that are configured for optimal security.

\hfill

The CIS Benchmarks are guidelines and best practices on security on many different types of software. These guidelines are freely available for anyone and can be found on their site\footnote{\url{https://www.cisecurity.org/cis-benchmarks/}}.

\hfill

They also provide guidelines on Docker\footnote{\url{https://www.cisecurity.org/benchmark/docker/}}. The latest version of the benchmarks (1.2.0) contain 115 guidelines. These are sorted by topic (e.g. Docker daemon and configuration files). In \hyperref[appendix:a]{the appendix} you will find an example CIS benchmark taken from the latest CIS benchmarks.

\section{Related Work}
